\chapter{Závěr}
Esport neustále roste a rozvíjí se do odvětví jako sázení, analytika či podnikání. Jedná se o nový trend, který zajímá nejvíce mladou populaci a hráče her.
Díky propojení esportu a sázkových kanceláří vzniká potřeba pro kvalitní a přesné modely, díky kterým lze výsledky zápasů předpovídat a stanovovat tak
vhodné kurzy. V profesionální scéně lze modely využít na identifikaci významných prediktorů pro výhru zápasu. Hráči a týmy se pak mohou pomocí dat
a modelů mohou soustředit pouze na charakteristiky, které jejich šanci na výhru zvyšují.

Vytvořené modely v bakalářské práci byli sestavené jak pro individuální hráče, tak pro dva referenční týmy a pro všechny týmy v datovém souboru. Z modelů
je patrné, že nejdůležitější prediktory se týkají počtu zabití nepřátel a počtu úmrtí hráče. Ty hrají významnou roli jak v modelech pro hráče, tak pro tým.
Rozdíl ve významnosti prediktorů se nachází mezi profesionálními týmy a amatérskými týmy. Z modelů lze odvodit důraz na individuální charakteristiky hráčů
v profesionálních týmech. V týmech amatérských lze pak pozorovat důraz na různé charakteristiky hráčů a na charakteristiky zápasu. 

Model by bylo možné rozšířit více podrobnými a specifickými charakteristikami hráčů. Modely zároveň nepočítají s tím, že se hráči v týmu mohou měnit a každý hráč má jiné
charakteristiky a délku hraní v týmu. Předpoklad pro aktuální modely je takový, že je známé, na jaké mapě se hraje a jaký tým začíná na jaké straně. 
V praxi je tento fakt známý až velmi krátce před začátkem zápasu. K modelům by bylo proto vhodné přidat predikci mapy, která se bude mezi dvěma týmy hrát.