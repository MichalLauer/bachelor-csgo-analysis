\chapter{Závěr}
Bakalářská práce se zabírala predikcí výher zápasů ve hře \ac{CSGO} pro hráče i pro týmy. Významnost modelů byla zjištěna pomocí Waldova testu a hladiny významnosti
$\alpha = 0.05$. Predikce modelů byla vyhodnocena pomocí matice záměn a statistik Přesnost, Senzitivita a Specificita.

Pro práci s modely bylo prve nutné data spojit do jednoho datového souboru. Modely byly vytvořené pomocí trénovací množiny dat, která činila 80\% ze 
spojeného datového souboru. Validační množina pak tvořila zbylých 20\% dat. Ta byla použita k tvorbě matice záměn.

Modely pro hráče byly vytvořené přes všechny kategorie map a modely pro mapu Mirage a Vertigo mezi sebou byli porovnané. Modely se lišily hlavně tím, zda se hodí pro 
predikci výher či proher. Model pro mapu Mirage je díky své vyšší specificitě vhodný pro identifikaci proher. Model pro mapu Vertigo má naopak vyšší senzitivitu
a hodí se spíše pro predikci výhry hráče. Každý model má také jiné významné prediktory, což značí, že každá se specifikuje jiným stylem hraním.

Modely pro celé týmy byly vytvořeny tři. Celkový model měl významnou většinu prediktorů. Jeho přesnost, senzitivita i specificita jsou velmi podobné a model díky tomu
predikuje stejně úspěšně jak výhry, tak prohry. Model pro referenční tým Astralis má významné pouze dva prediktory \textit{mean\_kills} a \textit{mean\_deaths}. Statistiky
z matice záměn naznačují, že se model hodí spíše k predikci výher. Model referenčního týmu Sprout má významné statistiky také pouze \textit{mean\_kills} a \textit{mean\_deaths}.
Z matice záměn lze usoudit, že se model hodí spíše k predikci výher týmu.