\chapter{Závěr}
Esport neustále roste a proniká do odvětví jako je sázení, analytika či podnikání. Jedná se o nový trend, který nejvíce zajímá mladou populaci a hráče her.
Propojením esportu a sázkových kanceláří vzniká potřeba pro kvalitní a přesné modely, díky kterým lze výsledky zápasů předpovídat a stanovovat tak
vhodné kurzy. Na profesionální scéně lze modely využít k identifikaci významných prediktorů pro výhru v zápase. Hráči a týmy se pak díky datům a modelům mohou soustředit
pouze na charakteristiky, které jejich šanci na výhru zvyšují.

Vytvořené modely v bakalářské práci byly sestavené pro individuální hráče, pro dva referenční týmy i pro všechny týmy v datovém souboru. Z modelů
je patrné, že nejdůležitější prediktory se týkají počtu zabití nepřátel a počtu úmrtí hráče. Ty hrají významnou roli jak v modelech pro hráče, tak i pro tým. 
Z modelů pro hráče lze usoudit, že nově přidaném mapy do hry jsou více vhodné pro týmy se silným individuálním výkonem hráčů. Čím déle je mapa ve hře, tím více
je vhodná pro strategické týmy. Rozdíl ve významnosti prediktorů se nachází mezi profesionálními a amatérskými týmy. Z druhého typu model lze odvodit důraz na individuální
charakteristiky hráčů v profesionálních týmech. V týmech amatérských lze pak pozorovat významnost charakteristik \textit{mean\_assists}, \textit{mean\_hs}, \textit{mean\_fkdiff},
\textit{team\_rank} a významnost interakce mezi mapou (\textit{map}) a počáteční stranou (\textit{starting\_ct})

Model by bylo možné rozšířit a přidat více podrobné a specifické charakteristiky hráčů. Modely navíc nepočítají s tím, že se hráči v týmu mohou obměňovat a mohou být
součástí týmu rozdílně dlouhou dobu. Aktuální modely považují za známé, na jaké mapě se zápas hraje a na jaké straně tým začíná. V praxi je však tento fakt známý
až velmi krátce před začátkem zápasu. K modelům by bylo proto vhodné přidat predikci mapy, která se bude mezi dvěma týmy hrát.
