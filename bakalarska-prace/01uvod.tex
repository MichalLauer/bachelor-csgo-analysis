\chapter{Úvod}
Esport je označení pro elektronický sport. Obsahuje všechny důležité oblasti jako klasický sport (např. turnaje, trénování, peníze, stadiony, či sázení) s tím rozdílem, že se hraje
na nějakém zařízení (počítač, konzole, mobil).
Esport je jedno z nejrychleji rostoucích odvětví v dnešní době. V roce 2021 se jeho tržní hodnota pohybovala kolem jedné miliardy dolarů - skoro
50\% nárůst oproti roku 2020. Lze předpovídat, že v roce 2024 esport překročí hodnotu 1,5 miliardy dolarů \cite{Gough2021}.
Dalo by se spekulovat, že za takový velký nárůst je zodpovědná aktuální pandemie. Většina populace, hlavně ta mladší, je nucena zůstat doma. Toto otevřelo dveře
se s esportem přirozeně seznámit a nějakým způsobem se ho účastnit (online divák, soutěžící, organizátor, fanoušek...). 
Hrají se různé kategorie her - střílečky, \acl{MOBA}\footnote{tzn. MOBA, kde hráči hrají v jedné online aréně proti sobě}, karetní hry, či \acl{BR}.

... Proč logistický model, již aplikované modely v reálném světě

... Proč Counter Strike, ale ne osobně ale objektivně

Finální cíl práce je vytvořit logistický model, který předpovídá výsledek zápasů. Tento model je vyhodnocen různými klasifikacemi pro vyhodnocení kvality.
Práce také popisuje grafické metody vizualizace dat a teorii k tvorbě a vyhodnocení logistických modelů. Také zde najdeme popis datového souboru a postup,
jakým byli vybráni nejvýznamnější prediktory - ať už statisticky, či čistě ze znalosti esportu. 
