\chapter{Úvod}
Esport je označení pro elektronický sport. Obsahuje všechny důležité oblasti jako klasický sport (např. turnaje, trénování, investice, stadiony, či sázení)
s tím rozdílem, že se hraje na nějakém zařízení (počítač, konzole, mobil).
Je to jedno z nejrychleji rostoucích odvětví v dnešní době. V roce 2021 se tržní hodnota esportu pohybovala kolem jedné miliardy dolarů - skoro
50\% nárůst oproti roku 2020. Lze předpovídat, že v roce 2024 esport překročí hodnotu 1,5 miliardy dolarů \cite{Gough2021}.
Dalo by se spekulovat, že za takový velký nárůst je zodpovědná aktuální pandemie. Většina populace je nucena zůstat doma. Toto otevřelo dveře
se s esportem přirozeně seznámit a nějakým způsobem se ho účastnit (online divák, soutěžící, organizátor, fanoušek...). 
Hrají se různé kategorie her
{\color{red}
jako např.
}
střílečky, \ac{MOBA}\footnote{tzn. MOBA, kde hráči hrají v jedné online aréně proti sobě}, karetní hry, \ac{FPS} či \ac{BR}.

Práce se zaměřuje primární na esportový titul \acf{CSGO}. Je to jeden z nejdéle hraných esportových titulů, boří mnohé divácké
rekordy\footnote{\scriptsize \url{https://www.invenglobal.com/articles/15619/csgo-major-breaks-viewership-records-overtakes-the-international}}
a je aktuálně nejhranějším \ac{FPS} esport titulem. \ac{CSGO} vyniká nejen detailní herní mechanikou, ale i bohatou a zajímavou historií. Hra
je unikátní i tím, že obsahuje mnoho různých módu\footnote{rozšíření, jak hru hrát. Každý mód má svá vlastní pravidla, mapy, či herní fanoušky}
a hráč může strávit mnoho hodin pouze objevováním komunitních serverů, hraním klasických zápasů či trénováním na offline mapách.

{\color{red}
Finální cíl práce je vytvořit logistický model, který předpovídá výsledek zápasů. Pro tvorbu kvalitního modelu bude kritické zvolit vhodné prediktory. Použitý data
set\footnote{\url{https://www.kaggle.com/mateusdmachado/csgo-professional-matches}} obsahuje čtyři souboru, které obsahují mnoho informací jak už o zápase (např. datum, 
výsledek zápasu, výsledek jednotlivých map, typ zápasu), hráčích (např. statistiky za zápas, statistiky za mapy, statistiky za team), tak o 
vývoji celého zápasu (především ekonomika týmu). V práci je tedy vytvořeno více specializovaných modelů pro každý vybraný týmů a následně
je pro každý tým vybrán nejlepší model. Výsledné modely jsou v závěru mezi sebou porovnány.
}

Logistický model je preferován kvůli své lehké interpretaci a 
dobré aplikaci v reálném životě. Výsledky, statistiky a pravděpodobnosti mohou být použity např. v sázkových kancelářích, kdy se výsledky modelu
dají využít na nejrůznější sázky 
{\color{red}
a lze předpovídat,
}
kdo vyhraje zápas, kdo vyhraje mapu, jaký hráč bude mít nejlepší statistiky, či zda si hráč koupí určitou zbraň.

{\color{red}
Práce je tedy rozdělená do tří částí. V první části je kladen důraz na esport, jeho vývoj, a na esportový titul \ac{CSGO}. Jsou zde také představená pravidla, podle kterých se
hra hraje. V druhé části jsou popsány popisné a statistické metody. Jsou zde definované jak grafické nástroje pro popis data setu, tak logistický regresní model. Třetí čas se
zaměřuje na praktickou tvorbu modelů, jejich interpretaci, a vzájemné porovnání.
}
