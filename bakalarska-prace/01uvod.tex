\chapter{Úvod}
Esport neboli elektronický sport se od klasického sportu příliš neliší. Taktéž vyžaduje tréninky, probíhají turnaje na stadionech a sazí se na něj. Rozdíl spočívá v tom, že
se hraje na nějakém zařízení --- na počítači, konzoli či mobilním telefonu. Esport je dnes jedním z nejrychleji rostoucích odvětví. V roce 2021 se jeho tržní hodnota pohybovala
okolo jedné miliardy dolarů, když narostla téměř o 50 \% oproti roku předchozímu. Lze předpokládat, že v roce 2024 hodnota esportu překročí hranici 1,5 miliardy dolarů. Výborné
prostředí pro organický růst esportu vytvořila především kombinace rozvoje počítačových her spolu s nástupem generace, jež je na práci s počítači zvyklá již od útlého dětství.
K výraznému nárůstu esportu však navíc přispěla pandemie koronaviru probíhající v letech 2019 --- 2022. Většina populace byla nucena zůstat doma, kde se přirozeně nabízela možnost
se s esportem seznámit a nějakým způsobem se ho účastnit, ať už jako online divák, soutěžící, organizátor, nebo fanoušek.

Práce se zaměřuje primárně na esportový titul \acf{CSGO}. Jedná se o jeden z nejdéle hraných esportových titulů, který boří mnohé divácké rekordy a aktuálně je nejhranějším
\ac{FPS} esport titulem. \ac{CSGO} vyniká nejen detailní herní mechanikou, ale i bohatou a zajímavou historií. Hra je unikátní i tím, že obsahuje mnoho různých
módů\footnote{Mód je rozšíření, jak lze hru hrát. Každý mód má svá vlastní pravidla, mapy i herní fanoušky}. Cílem každého týmu je pak buď zneškodnit bombu, nebo ji položit na
předem určené bomboviště. Zápas se může hrát na různých mapách a každý tým hraje buď za teriristy (cílem je položit bombu) nebo counter-terroristy (cílem je bombu zneškodnit).
K dosažení cíle se používají různé taktiky, zbraně či týmová souhra.

Cílem práce je vytvořit vícerozměrný logistický regresní model, který předpovídá výsledky zápasů. Pro vytvoření kvalitního modelu bude kritická vhodná volba
charakteristiky hráčů a jednotlivých týmů. Pro predikci jsou použity charakteristiky, které se nacházejí ve dvou samostatných datových
souborech obsahujících informace o zápase (např. datum, výsledek zápasu, výsledek jednotlivých map, typ zápasu) a hráčích (např. charakteristiky za zápas). 

Predikovat výhru hráče a týmu je důležité zejména pro oblast sázení, která je s esportem úzce spojena. Přesné predikce a kvalitní modely sázkovým kancelářím umožňují
stanovovat výhodné a profitabilní kurzy. Kurzy pak nemusí být vypsány pouze na výhru, či prohru, ale také např. na charakteristiky hráčů nebo hranou mapu. Logistický model je pro
predikci výhry či prohry preferován kvůli své snadné interpretaci.

Závěrečná práce je rozdělená do tří částí.První část se věnuje esportu, jeho vývoji a konkrétně také esportovému titulu \ac{CSGO}. Jsou zde představena pravidla, podle kterých se
hra hraje. V druhé části jsou představeny popisné a statistické metody. Jsou zde definovány grafické nástroje pro popis datového souboru, logistického
regresního modelu a evaluační nástroje pro model. Třetí část se zaměřuje na praktickou tvorbu modelů, jejich interpretaci a vzájemné porovnání.
