\chapter{Úvod}
Esport je označení pro elektronický sport. Obsahuje všechny důležité oblasti jako klasický sport (např. turnaje, trénování, investice, stadiony, či sázení)
s tím rozdílem, že se hraje na nějakém zařízení (počítač, konzole, mobil). Je to jedno z nejrychleji rostoucích odvětví v dnešní době. V roce 2021 se tržní hodnota
esportu pohybovala kolem jedné miliardy dolarů --- skoro 50\% nárůst oproti roku 2020. Lze předpovídat, že v roce 2024 esport překročí hodnotu 1,5 miliardy dolarů.
Dalo by se spekulovat, že za takový velký nárůst je zodpovědná pandemie koronaviru v letech 2019 --- 2022. Kombinace rozvoje počítačových her a generace, která je na
práci s počítačem zvyklá od mala, vzniklo výborné prostředí pro organický růst esportu. Většina populace byla nucena zůstat doma a to otevřelo dveře
se s esportem přirozeně seznámit a nějakým způsobem se ho účastnit (online divák, soutěžící, organizátor, fanoušek...). 

Práce se zaměřuje primárně na esportový titul \acf{CSGO}. Je to jeden z nejdéle hraných esportových titulů, boří mnohé divácké rekordy a je aktuálně nejhranějším
\ac{FPS} esport titulem. \ac{CSGO} vyniká nejen detailní herní mechanikou, ale i bohatou a zajímavou historií. Hra je unikátní i tím, že obsahuje mnoho různých
módů\footnote{rozšíření, jak hru hrát. Každý mód má svá vlastní pravidla, mapy, či herní fanoušky} a hráč může strávit mnoho hodin pouze objevováním komunitních
serverů, hraním klasických zápasů či trénováním na offline mapách.

Finálním cílem práce je vytvořit vícerozměrný logistický regresní model, který předpovídá výsledky zápasů. Pro tvorbu kvalitního modelu bude kritické zvolit vhodné
charakteristiky hráčů a jednotlivých týmů. Pro predikci jsou použity charakteristiky, které se nacházejí ve dvou samostatných datových
souborech, které podávají informace jak už o zápase (např. datum, výsledek zápasu, výsledek jednotlivých map, typ zápasu) a hráčích (např. charakteristiky za zápas). 

Predikovat výhru hráče a týmu je důležité zejména v oblasti sázení, která je s esportem úzce spojená. Přesné predikce a kvalitní modely sázkovým kancelářím umožňují
stanovovat výhodné a profitabilní kurzy. Kurzy pak nemusí být pouze na výhru či prohru, ale např. na charakteristiky hráčů či hranou mapu. Logistický model je 
pro predikci výhry či prohry preferován kvůli své snadné interpretaci.

Práce je rozdělená do tří částí. V první části je kladen důraz na esport, jeho vývoj, a na esportový titul \ac{CSGO}. Jsou zde také představená pravidla, podle kterých se
hra hraje. V druhé části jsou popsány popisné a statistické metody. Jsou zde definované grafické nástroje pro popis datového souboru, logistického
regresního modelu, a evaluační nástroje pro model. Třetí část se zaměřuje na praktickou tvorbu modelů, jejich interpretaci, a vzájemné porovnání.
