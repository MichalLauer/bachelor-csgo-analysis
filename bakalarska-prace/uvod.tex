\chapter*{Úvod}
\addcontentsline{toc}{chapter}{Úvod}
Esport je jedno z nejrychleji rostoucích odvětví v dnešní době. V roce 2021 se jeho tržní hodnota pohybovala kolem jedné miliardy dolarů - skoro
50\% nárůst oproti roku 2020. Dle portálu statista.com lze předpovídat, že v roce 2024 esport překročí hodnotu 1,5 miliardy dolarů \cite{Gough2021}.
Dalo by se spekulovat, že za takový velký nárůst je zodpovědná aktuální pandemie. Většina populace, hlavně ta mladší, je nucena zůstat doma. Toto otevřelo dveře
se s esportem přirozeně seznámit a nějakým způsobem se ho účastnit \textit{(online divák, soutěžící, organizátor, fanoušek...)}. Esport je vlastně sport,
akorát s počítačovými hrami. Hrají se různé kategorie her - střílečky \textit{(\ac{CS:GO}, Valorant)}, arény \textit{(\ac{LoL})}, či karetní hry
\textit{(\ac{HS})}.

Toto téma jsem si zvolil hlavně kvůli tomu, že se o oblast zajímám od mého mládí. Když jsem si vybíral téma na bakalářskou práci,
chtěl jsem propojit statistiku s něčím, co mě baví a naplňuje - toto je ideální kombinace. Zároveň bylo mím cílem vytvořit práci, která bude v dnešní době relevantní.
Zvolené téma je dle mého názoru velmi aktuální, avšak ne pro širokou veřejnost, nýbrž pouze pro lidi, co se o zajímají o esport či sázení.
Podobné logistické modely, avšak velmi složitější, mohou totiž sloužit například k vyhodnocení sázkových kurzů. Esport, tak jak klasický sport, je se sázením propojen. 

Finální cíl práce je vytvořit logistický model, který předpovídá výsledek zápasů. Tento model je vyhodnocen různými klasifikacemi pro vyhodnocení kvality.
Práce také popisuje grafické metody vizualizace dat a teorii k tvorbě a vyhodnocení logistických modelů. Také zde najdeme popis datového souboru a postup,
jakým byli vybráni nejvýznamnější prediktory - ať už statisticky, či čistě ze znalosti esportu. 
