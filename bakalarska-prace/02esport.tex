\chapter{Představení esportu}
\section{Historie esportu}
I přes fakt, že esport není obecně známý pojem mezi širokou veřejností, má přes 70 let bohaté historie. Za jeho počátky by se
daly považovat arkádové automaty, kde hráči z počátku soutěžili sami proti sobě. Největší rozvoj arkádových automatů se děl kolem 70 let minulého 
století. Nejen za tímto účelem byla 9. 2. 1982 založena \textit{\ac{TGNS}}. \ac{TGNS} měla na starosti nejen udržování výsledkové tabulky \textit{(ang. scoreboard)},
ale i tvorbu prvotních pravidel pro férovou hru. Za tímto účelem byla vydána kniha \textit{Twin Galaxies' Official Video Game \& Pinball Book of World Records}.

Na přelomu osmdesátých let minulého století se začal esport vyvíjet již více profesionálním směrem. V roce 1972 pořádala Standfordská Universita historicky první
esportový turnaj v arkádové hře \textit{Spacewar!}. Výherce si mohl odnést předplatné magazínu Rolling Stones. Dále v roce 1983 byl založen první esportový profesionální team,
který se nacházel ve Spojených státech. Všechno toto se stalo díky podnikateli Walteru Day, který je jak zakladatel společnosti \ac{TGNS} a založil již zmíněný
prvních esportový team. Ač se Walter považuje za jednoho z hlavních pionýrů esportu, v roce 2010 \ac{TGNS} opustil kvůli své vášni pro hudbu.

Další důležitou kapitolou ve vývoji esportu je příchod internetu a výkonných počítačů. Hráči měli rychlejší sestavy, stolní počítače byli cenově dostupnější a díky tomu
se dostali k více lidem. Klesala cena hardwaru, vývoj nové technologie a her se zrychloval. Díky rozvoji počítačových sítí se mohli hrát LAN party\footnote{Hráči hrají v jedné
místnosti na lokální počítačové síti.} či organizovat BYOC turnaje\footnote{z ang. Bring Your Own Computer. Hráči si na akci donesou vlastní počítač}. Dále už esport potřeboval
jen čas a dnes má tržní hodnotu přes jednu miliardu amerických dolarů \cite{Gough2021}, \cite{Larch2019}.
\section{Zasazení do dnešní doby}
...

\section{Představení vybraného esportového titulu}
...

\section{Propojení bakalářské práce a esportového titulu}
...
