\chapter{Představení esportu}
\section{Historie esportu}
I přes fakt, že esport není obecně známý pojem mezi širokou veřejností, má přes 70 let bohaté historie. Za jeho počátky by se
daly považovat arkádové automaty, kde hráči z počátku soutěžili sami proti sobě. Největší rozvoj arkádových automatů se děl kolem 70 let minulého 
století. Nejen za tímto účelem byla 9. 2. 1982 založena \textit{\ac{TGNS}}. \ac{TGNS} měla na starosti nejen udržování výsledkové tabulky \textit{(ang. scoreboard)},
ale i tvorbu prvotních pravidel pro férovou hru. Za tímto účelem byla vydána kniha \textit{Twin Galaxies' Official Video Game \& Pinball Book of World Records}.

Na přelomu osmdesátých let minulého století se začal esport vyvíjet již více profesionálním směrem. V roce 1972 pořádala Stanfordská Universita historicky první
esportový turnaj v arkádové hře \textit{Spacewar!}. Výherce si mohl odnést předplatné magazínu Rolling Stones. Dále v roce 1983 byl založen první esportový profesionální team,
který se nacházel ve Spojených státech. Všechno toto se stalo díky podnikateli Walteru Day, který je jak zakladatel společnosti \ac{TGNS} a založil již zmíněný
prvních esportový team. Ač se Walter považuje za jednoho z hlavních pionýrů esportu, v roce 2010 \ac{TGNS} opustil kvůli své vášni pro hudbu.

Další důležitou kapitolou ve vývoji esportu je příchod internetu a výkonných počítačů. Hráči měli rychlejší sestavy, stolní počítače byli cenově dostupnější a díky tomu
se dostali k více lidem. Klesala cena hardwaru, vývoj nové technologie a her se zrychloval. Díky rozvoji počítačových sítí se mohli hrát LAN party\footnote{Hráči hrají v jedné
místnosti na lokální počítačové síti.} či organizovat BYOC turnaje\footnote{z ang. Bring Your Own Computer, kde si hráči si na akci donesou vlastní počítač}. Dále už esport 
potřeboval jen čas na organický růst a dnes má tržní hodnotu přes jednu miliardu amerických dolarů \cite{Gough2021}, \cite{Larch2019}.

\section{Zasazení do dnešní doby}
Jak již bylo zmíněno, esport je v dnešní době téměř miliardová záležitost. Díky pandemii, která trvá již třetím rokem, si esport ještě přilepšil. Dle průzkumu \cite{Gough2021a}
z října roku 2020 si 73 \% dotázaných myslelo, že se úroveň zájmu \textit{(ang. level of investment)} a obchodní činnost \textit{(ang. deal activity)} v Q4 2020 a Q1 2021
zvětší. Respondenti, kteří se průzkumu zúčastnili, jsou považování za \uv{industry professionals}. Tento průzkum byl následné podpořen růstem že tržní hodnoty esportu a mezi lety 2019 a 2020
vzrostla o téměř 50 \% \cite{Gough2021}.

K takto prudkému růstu tržní hodnoty esportu z velké části přispěla právě pandemie. Mladá generace byla nucena zůstat doma, což dovolilo i esportem nedotčeným jedincům do
tohoto světa proniknout. Větší zájem o esport přinesl i větší tržby herním studiím, které začali do esportových turnajů investovat více peněz\cite{Professeur2021}\cite{liquipedia2021}.
S větším počtem diváku roste i marketingový potenciál, investiční příležitost a kariérní růst.

V dnešní době má esport mnoho titulů, proto představím jen ty nejvýznamnější. Největší esport rivalita je mezi herním titule \ac{LoL} a Dota 2. Oba tituly jsou žánru \ac{MOBA}, díky 
čemuž mají podobnou, avšak velmi rozdílnou fanouškovskou základnu. Historie mezi tituly je velmi složitá, avšak mimo rozsah této práce. Pro rozšíření znalosti mohu doporučit videa 
z youtubového kanálu theScore esport o tomto tématu - \href{https://www.youtube.com/watch?v=h9Zv_TiVzmg}{The~Story of Dota 2} a \href{https://www.youtube.com/watch?v=tHtfD-MnQK8}{The Story of League of Legends}.

Druhý dominantní žánr je \ac{FPS}. V této kategorii dominují hry \ac{CS:GO} a Valorant. Zde proti sobě hrají dva teamu, většinou složené z pěti hráčů. Každý hráč pak má v teamu různou roli, jako např. velitel
či odstřelovač. Jeden team má obvykle za úkol něco zničit \textit{(položit bombu, unést rukojmí)} a druhý team jim v tom musí zabránit \textit{(ochránit oblast proti bombě, záchrana rukojmí)}.

Poslední žánr, který zmíním, je \ac{BR}. V těchto hrách hraje buď každý hráč za sebe, ve dvojicích, nebo skupinách po čtyřech. Zde hráči padají na začátku kola na velkou mapu. Jejich úkolem je
získat tzn. \uv{loot} \textit{(vybavení)}, aby mohl porazit ostatní hráče a kolo sami, nebo s týmem vyhrát. Nacházejí se zde různé role, avšak trošku rozdílné oproti žánru \ac{FPS}. Hlavním titulem
této kategorie je hra Fortnite, která žánru dominuje. Stal se z ní jak esport titul, tak perfektní marketingové místo pro teenagery. Hráči si zde mohou koupit oblečky různých filmových či komiksových postav.
Pokud vychází nový film, ve hře se může objevit \uv{event} \textit{(událost)}, který daný film propaguje. Toto lze vidět například na propagaci \href{https://www.youtube.com/watch?v=TanGK9o_d24}{Avengers: Endgame}.

\section{Představení titulu Counter-Strike: Global Offensive}
\aclu{CS:GO} jak ho známe dnes, má bohatou a dlouho historii. Ne vždy se to ovšem jmenovalo stejně. Úplně první iterace hry se jmenovala čistě Counter-Strike a byl to pouze mód\footnote{upravení či rozšíření hry} do
hry Half-Life. Half-Life byl vyvinutí společností Valve, tehdy primárně společností zaměřenou na vyvíjej her. Mód byl vytvořen studenty vysoké školy, panem \textit{Minh Le} a \textit{Jess Cliffe}. Toto rozšíření začali programovat
v roce 1999. Jelikož mód byl neoficiálním rozšířením, Valve se o něj moc nezajímalo. Až po pěti betaverzích hry Counter-Strike si společnost Valve všimla rozšíření, její komunity, ale především jejich autorů. Minh a Jess
se v roce 2000 stali oficiálními zaměstnanci Valve, prodali \uv{duševní vlastnictví} módu Valve. Autoři, nově jako zaměstnanci Valve, roku 2000 vydávají první oficiální verzi hry Counter-Strike. I přes toto \uv{oficiální}
datum vydání je většina komunity přesvědčena, že výročí má \ac{CS:GO} v den svého úplně první vydání, a to 18. června 1999.

Hra je z žánru \aclu{FPS} a hraje se primárně online proti skutečným hráčům. Counter-Strike se v herní komunitě rychle rozrostl jeho jednoduchosti. Hra se dá velmi dobře popsat pořekadlem
\textit{\uv{Lehké hrát, těžké vypilovat} (ang. Easy to play, hard to master.)}. Hra má mechaniky\footnote{herní prvky či unikátní vlastnosti}, které jsou lehké na pochopení, ale velmi těžké na vypilování k dokonalosti.
Spolu s touto vlastností je hra vlastně velmi jednoduchá a hráč hraje buď za policisty, nebo za teroristy. Hráči tak mohli, a stále můžou, hru velmi lehce a rychle začít hrát - tento formát se totiž za posledních 20 let nezměnil. 

Hra tedy rostla zejména díky své komunitě. Hráči hru různě upravovali, přidávali další módy, typy her, zbraně, mapy či audiovizuální obsah. Tento trend se přenášel přes mnoho různých verzí hry. První velký \uv{průlom} udělala
verze 1.6, tedy Counter-Strike 1.6. Ta kvetla jak esportem, tak komunitním obsahem. Jen v České a Slovenské republice bylo několik herních serverů, na kterých se mohlo sejít sta tisíce hráčů. Např. na česko-slovenském 
herním portálu \textit{kotelna} hrálo celkem přes 1,5 milionu unikátních hráčů \cite{csko2021}. Hra byla populární nejen mezi \uv{casual} hráči, ale i profesionály.

Counter-Strike 1.6 je pionýrem esportu pro \ac{FPS} žánr. Za podpory Valve se hráli první major\footnote{turnaj pořádaný přímo Valve, který má největší prestiž} turnaje, kde hráči mohli ukázat svůj um za tehdy relativně velkou
sumu peněz. Hra se časem vyvíjela, hráči nalézali nové strategie či triky a Valve vydalo novou verzi - Counter-Strike: Source. Tato nová verze získala nepříliš pozitivní ohlas, jelikož velmi rozdělila herní komunitu. Představila 
nové mechaniky, staré mechaniky změnila a hráčům, zejména v esportu, se nechtělo učit něco úplně nového. Valve se rozhodlo sjednotit herní komunitu, a proto vydalo novou verzi hry - \aclu{CS:GO}

\ac{CS:GO} se snažilo sjednotit oba tábory - Counter-Strike 1.6 a Counter-Strike: Source. Hra vyšla 21. srpna 2012 a z počátku nebyla tolik úspěšná, ale díky přidání různých skinů\cite{Valve2013} na zbraně hra přilákala
úplně nové publikum. Díky novému a velkému publiku se začali hrát menší esportové turnaje právě ve hře \ac{CS:GO}, ke kterým se později přidali i profesionále z předchozích dvou verzí. Díky tomuto organickému růstu má
Counter-Strike velmi silnou komunitu, která se o hru i nadále stará. I přes netradiční interakci mezi Valve a herní komunitou hra stále roste. \ac{CS:GO} se díky své dlouhé historii, bohaté komunitě a různým možnostem,
jak hru hrát, dostala na špičku esportu. I přes několik titulů, které se s hrou snaží soutěžit, je hra stále největším a nejsledovanějším esport titulem v rámci \ac{FPS} žánru\cite{Henningson2020}.

\section{Propojení práce a titulu Counter-Strike: Global Offensive}
\subsection{Vysvětlení hry}
Jak již bylo zmíněno, \ac{CS:GO} hraje pět hráčů proti pěti \textit{(dále jen 5v5)}. Hra se většinou hraje online, avšak velké esportové turnaje se hrají offline, tedy v nějaké např. aréně. Hra má v základu 30 kol a po 
prvních patnácti se mění strany. Jedna strana jsou policisté \textit{(Counter-Terrorists či CT)}, kteří mají za úkol chránit \uv{bomboviště} - část mapy, která má vybouchnout. Naopak cíl Teroristů \textit{(T)} je právě
bombu položit a \uv{bomboviště} nechat vybouchnout. Vyhrává team, který první dosáhne 16 kol. Pokud ovšem po prvních 30 kolech je stav nerozhodný, tedy 15:15, hraje se prodloužení. Tento formát není standardizovaný pro
všechny turnaje, proto se budu zaměřovat čistě na turnaje, které pořádá Valve \textit{(již zmíněné a nejvíc prestižní Majory)}. Zde se hraje prodloužení ve formát Bo6, tedy kdo první získá 4 body, vyhraje zápas. Takto 
může jít zápas teoreticky do nekonečna. Nejdelší semi-profesionální zápas, který se ovšem neodehrál na Majoru, se stal mezi týmem exceL a XENEX\cite{hltv.org2015}. Zápas pokračoval do úctyhodných 88 kol.

V každém kole má tým určitý počet peněz. Každá hráč začíná polovinu \textit{(ted v první a šestnácté kolo)} s \$800. Finance každého hráče pak záleží na mnoha faktorech - kolik vyhrál jeho team kol v řadě, kolik nakoupil
zbraní, kolik zabil nepřítelů, kolik peněz dostane hráč za zabití či jak kolo skončí. V profesionálním teamu je velmi obtížné pracovat s financemi, jelikož hráči musí být s financemi na jedné stránce. V tuto chvíli přichází
na řadu tzn. In-Game Leader \textit{(velitel teamu)}. Tuto roli má většinou jeden hráč v každém teamu. Je to ta nejdůležitější role ze všech - má na starosti finance, rozhoduje kdy se koupí a kdy půjde tzn. eco 
\textit{(hráči nekoupí nic, aby ušetřili peníze)}, jaké se budou hrát mapy či jaká se půjde v daném kole strategie.V dnešní době k tomu In-Game Leader má i pomocníka - trenéra. Ten hru nehraje, ale pozoruje hráče a dává jim
různé typy a triky.

Role trenéra není nijak silně definovaná a každý esportový team má trošku jiného trenéra. V jednom případě může být trenér čistě jako podpora - pomáhá hráčům když se nedaří a řeší interní problémy. V jiném
teamu může ovšem mít velký zásah do hry, pomáhat In-Game Leaderovi se strategiemi, obelstění soupeře či sledováním předchozích zápasů pro kontinuální zlepšování teamu.
Další role v teamu jsou například Entry Fragger \textit{(má za úkol získat první zabití pro team)}, support \textit{(podporuje svůj team s pomocí různých granátů nebo se často pro svůj team obětuje)}, AWP hráč
\textit{(hráč je specifický tím, že hraje primárně s jednou zbraní - odstřelovací puškou AWP)} a Lurker \textit{(chodí po mapě sám a snaží se nepřítele odchytnout ze stran, které by nečekali)}

Zápasy se pak hrají ve formátech \uv{Best of}. \textit{Best of 3} například znamená, že se hrají tři mapy - kdo vyhraje dvě, vyhrál zápas. Turnaje se pak odehrávají v tradičních formátech, jako je pavouk.
Ten se charakterizuje tím, že vypadá jak pavučina, jde z leva a každý team může prohrát pouze jednou. Následně tu máme Upper/Lower bracket formát, který je v podstatě \uv{pavoučí formát}, akorát jsou zde dvě
sítě a každý team může prohrát maximálně jednou, jinak je vyřazen. Specifičtější formát pro \ac{CS:GO} je například swiss, který je složitější a mimo rozsah této práce.

Práce se zabývá právě predikcí daného zápasu. K tomuto lze použít např. hodnocení hráčů či umístění teamu na světovém žebříčku. Dále také záleží, na jaké mapě se zápas hraje. Když se budeme dívat na historické 
zápasy, musíme se dívat pouze na to, kdo zápas vyhrál - ne na rozdíl kol. Ten může být velmi zavádějící a zápas s výsledkem 16:7 mohl být více vyrovnaný než zápas s výsledkem 16:13. Toto jde vidět na zkušenostech teamu.
Zápas mezi velmi dobrými esportovými teamy může dopadnout 16:7, ale může být velmi těsný - oba dva týmy hráli dobře a o výsledku rozhodovali maličkosti. Pokud hrají dva méně zkušené týmy, zápas může skončit např. 16:13,
ale nemusí být vůbec blízko - jeden team může dělat zbytečné chyby \textit{(které by lepší team neudělal)}, které se normálně nedějí, což může vést k více těsnému výsledku i přes rozdílné hodnocení teamů.  