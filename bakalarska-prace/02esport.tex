\chapter{Představení esportu}
\section{Historie esportu}
I přes fakt, že esport není obecně známý pojem mezi širokou veřejností, má přes 70 let bohaté historie. Za jeho počátky by se
daly považovat arkádové automaty, kde hráči z počátku soutěžili sami proti sobě. Největší rozvoj arkádových automatů se děl kolem 70 let minulého 
století. Nejen za tímto účelem byla 9. 2. 1982 založena \textit{\ac{TGNS}}. \ac{TGNS} měla na starosti nejen udržování výsledkové tabulky \textit{(ang. scoreboard)},
ale i tvorbu prvotních pravidel pro férovou hru. Za tímto účelem byla vydána kniha \textit{Twin Galaxies' Official Video Game \& Pinball Book of World Records}.

Na přelomu osmdesátých let minulého století se začal esport vyvíjet již více profesionálním směrem. V roce 1972 pořádala Standfordská Universita historicky první
esportový turnaj v arkádové hře \textit{Spacewar!}. Výherce si mohl odnést předplatné magazínu Rolling Stones. Dále v roce 1983 byl založen první esportový profesionální team,
který se nacházel ve Spojených státech. Všechno toto se stalo díky podnikateli Walteru Day, který je jak zakladatel společnosti \ac{TGNS} a založil již zmíněný
prvních esportový team. Ač se Walter považuje za jednoho z hlavních pionýrů esportu, v roce 2010 \ac{TGNS} opustil kvůli své vášni pro hudbu.

Další důležitou kapitolou ve vývoji esportu je příchod internetu a výkonných počítačů. Hráči měli rychlejší sestavy, stolní počítače byli cenově dostupnější a díky tomu
se dostali k více lidem. Klesala cena hardwaru, vývoj nové technologie a her se zrychloval. Díky rozvoji počítačových sítí se mohli hrát LAN party\footnote{Hráči hrají v jedné
místnosti na lokální počítačové síti.} či organizovat BYOC turnaje\footnote{z ang. Bring Your Own Computer, kde si hráči si na akci donesou vlastní počítač}. Dále už esport 
potřeboval jen čas a dnes má tržní hodnotu přes jednu miliardu amerických dolarů \cite{Gough2021}, \cite{Larch2019}.
\section{Zasazení do dnešní doby}
Jak již bylo zmíněno, esport je v dnešní době téměř miliardová záležitost. Díky pandemii, která trvá již třetím rokem, si esport ještě přilepšil. Dle průzkumu \cite{Gough2021a}
z října roku 2020 si 73 \% dotázaných myslí, že se úroveň zájmu \textit{(ang. level of investment)} a obchodní činnost \textit{(ang. deal activity)} v Q4 2020 a Q1 2021
zvětší. Respondenti, kteří se průzkumu zúčastnili, jsou považování za \uv{industry professionals}. Tento průzkum potvrzuje fakt, že tržní hodnota esportu mezi lety 2019 a 2020
vzrostla o téměř 50 \% \cite{Gough2021}.

K takto prudkému růstu tržní hodnoty esportu z velké části přispěla právě pandemie. Mladá generace byla nucena zůstat doma, což dovolilo i esportem nedotčeným jedincům do
tohoto světa proniknout. Větší zájem o esport přinesl i větší tržby herním studiím, které začali do esportových turnajů investovat více peněz\cite{Professeur2021}\cite{liquipedia2021}.
S větším počtem diváku roste i marketingový potenciál, investiční příležitost a kariérní růst.

V dnešní době má esport mnoho titulů, proto představím jen ty nejvýznamnější. Největší esport rivalita je mezi herním titule \ac{LoL} a Dota 2. Oba tituly jsou žánru \ac{MOBA}, díky 
čemuž mají podobnou, avšak velmi rozdílnou fanouškovskou základnu. Historie mezi tituly je velmi složitá, avšak mimo rozsah této práce. Pro rozšíření znalosti mohu doporučit videa 
z youtubového kanálu theScore esport o tomto tématu - \href{https://www.youtube.com/watch?v=h9Zv_TiVzmg}{The~Story of Dota 2} a \href{https://www.youtube.com/watch?v=tHtfD-MnQK8}{The Story of League of Legends}.

Druhý dominantní žánr je \ac{FPS}. V této kategorii dominují hry \ac{CS:GO} a Valorant. Zde proti sobě hrají dva teamu, většinou složené z pěti hráčů. Každý hráč pak má v teamu různou roli, jako např. velitel
či odstřelovač. Jeden team má obvykle za úkol něco zničit \textit{(položit bombu, unést rukojmí)} a druhý team jim v tom musí zabránit \textit{(ochránit oblast proti bombě, záchrana rukojmí)}.

Poslední žánr, který zmíním, je \ac{BR}. V těchto hrách hraje buď každý hráč za sebe, ve dvojicích, nebo skupinách po čtyřech. Zde hráči padají na začátku kola na velkou mapu. Jejich úkolem je
získat tzn. \uv{loot} \textit{(vybavení)}, aby mohl porazit ostatní hráče a kolo sám, nebo s teamem vyhrát. Nacházejí se zde různé role, avšak trošku rozdílné oproti žánru \ac{FPS}. Hlavním titulem
této kategorie je hra Fortnite, která žánru dominuje. Stal se z ní jak esport titul, tak perfektní marketingové místo pro teenagery. Hráči si zde mohou koupit oblečky různých filmových či komixových postav.
Pokud vychází nový film, ve hře se může objevit \uv{event} \textit{(událost)}, který daný film propaguje. Toto lze vidět například na propagaci \href{https://www.youtube.com/watch?v=TanGK9o_d24}{Avengers: Endgame}.
\section{Představení vybraného esportového titulu}
...

\section{Propojení bakalářské práce a esportového titulu}
...
