\chapter{Představení esportu}
\section{Historie esportu}
Přestože esport mezi širokou veřejností není obecně známým pojmem, jeho bohatá historie se píše více než 70 let. Za počátky esportu lze považovat arkádové automaty, kde hráči
zpočátku soutěžili sami se sobou. Největší rozvoj arkádových automatů probíhal v 70. letech minulého století. Dne 9. 2. 1982 pak byla založena \ac{TGNS}. \ac{TGNS} je databáze
obsahující rekordy z arkádových her, obsahuje výsledkové tabulky a pravidla pro férovou hru. Z této databáze byla vytvořena kniha Twin Galaxies' Official
Video Game \& Pinball Book of World Records, která nustále vychází s novými záznamy a rekordy.

Na přelomu osmdesátých let minulého století se začal esport vyvíjet již více profesionálním směrem. V roce 1972 pořádala Stanfordská univerzita historicky první
esportový turnaj v arkádové hře Spacewar! Výherce si mohl odnést předplatné magazínu Rolling Stones. V roce 1983 byl v USA založen první profesionální esportový tým. Největší
zásluhu na tomto rozvoji měl podnikatel Walter Day, jenž založil jak \ac{TGNS}, tak i zmíněný první esportový tým. Ač se Day považuje za jednoho z hlavních pionýrů esportu, v
roce 2010 \ac{TGNS} opustil kvůli své vášni pro hudbu.

Další důležitá kapitola ve vývoji esportu se začala psát s příchodem internetu a výkonných počítačů. Hráči dostali možnost hrát na lepších počítačových sestavách, stolní počítače
se staly cenově dostupnějšími a díky tomu se zpřístupnily více lidem. Klesala cena hardwaru, vývoj nové technologie a her se zrychloval. Díky rozvoji počítačových sítí se mohly
odehrávat LAN\footnote{Hráči hrají v jedné místnosti na lokální počítačové síti.} party či organizovat BYOC\footnote{z ang. Bring Your Own Computer, kde si hráči na akci donesou
vlastní počítač} turnaje. Poté už esport potřeboval jen čas na organický růst.

\section{Zasazení do dnešní doby}
V dnešní době je esport téměř miliardovou záležitostí. Esportu pomohla i pandemie koronaviru probíhající v letech 2019 až 2022.
Dle průzkumu \cite{gough_future_2021} z října roku 2020 si 73 \% dotázaných odborníků na esport myslelo, že se úroveň zájmu o něj a obchodní činnost esportu v Q4 2020 a
Q1 2021 zvýší. Odborníci se nemýlili, neboť tržní hodnota esportu mezi lety 2019 a 2020 meziročně vzrostla o téměř 50 \% \cite{gough_global_2022}.

K takto prudkému růstu tržní hodnoty esportu z velké části přispěla právě pandemie. Mladá generace byla nucena zůstat doma, což dovolilo do tohoto světa proniknout i esportem
dosud netknutým jedincům . Větší zájem o esport přinesl i větší tržby herním studiím, která začala do esportových turnajů více investovat \cite{professeur_esea_2022},
\cite{liquipedianet_pgl_2021}. S větším počtem diváku roste i marketingový potenciál, investiční příležitosti a možnost kariérního růstu.

Jedním z dominantních žánrů počítačových her je žánr \acf{FPS}. V této kategorii mezi nejvýznamnější hry patří \ac{CSGO} a Valorant. V tomto žánru proti sobě hrají dva týmy,
většinou složené z pěti hráčů. Každý hráč pak má v týmu vlastní roli, např. velitele či odstřelovače. Jeden tým má obvykle za úkol něco zničit (položit bombu, unést rukojmí)
a druhý tým mu v tom musí zabránit (ochránit oblast proti bombě, zachránit rukojmí).

Dalším významným a zajímavý žánrem je žánr \acf{BR}. V těchto hrách hraje každý hráč buď sám za sebe, ve dvojici, nebo ve skupině po čtyřech. Zde hráči skákají z letadla na
začátku kola na velkou mapu. Jejich úkolem je získat vybavení, aby mohli porazit ostatní hráče a kolo vyhrát, ať už sami, nebo s týmem. Nabízejí se zde různé role, avšak trošku
rozdílné oproti žánru \ac{FPS}. Žánru BR dominuje hra Fortnite. Stal se z ní jak esportový titul, tak marketingové místo pro teenagery.
Hráči si zde mohou koupit oblečky různých filmových či komiksových postav. Pokud vychází nový film, ve hře se může objevit \uv{event} (událost),
jenž daný film propaguje. Takové nástroje byly například použity k propagaci filmu Avengers: Endgame\footnote{Trailer pro propagaci události:
\url{https://www.youtube.com/watch?v=TanGK9o_d24}}.

\section{Představení titulu Counter-Strike: Global Offensive}
\ac{CSGO}, jak ho známe dnes, má bohatou a dlouho historii. Ne vždy se ovšem hra jmenovala stejně. První iterace hry nesla čistě název Counter-Strike a jednalo se pouze o
mód\footnote{upravení či rozšíření hry} do hry Half-Life. Half-Life byl vyvinut společností Valve, která se primárně zaměřuje na vývoj her. Mód byl však
vytvořen studenty vysoké školy. Minh Le a Jess Cliffe toto rozšíření začali programovat v roce 1999. Protože mód nebyl oficiálním rozšířením, společnost Valve o něj neprojevovala
veliký zájem. Až po pěti betaverzích hry Counter-Strike si společnost Valve všimla rozšíření hry, její komunity, ale především jejích autorů. Autoři rozšíření se v roce 2000 stali
oficiálními zaměstnanci Valve a autorská práva k módu prodali. Ještě ve stejném roce pak pod záštitou Valve vydali první oficiální verzi hry Counter-Strike. Navzdory tomuto
\uv{oficiálnímu} datu je většina komunity přesvědčena, že výročí \ac{CSGO} připadá na den svého úplně prvního vydání, a to na 18. června 1999.

Hra je z žánru \ac{FPS} a hraje se primárně online proti skutečným hráčům. Counter-Strike se v herní komunitě rychle rozšířil díky své jednoduchosti.
Hrát tuto hru je snadné, podat kvalitní výkon už je však obtížnější. Hra má mechaniky\footnote{herní prvky či unikátní vlastnosti},
které jsou snadné na pochopení, ale je velmi těžké je vypilovat k dokonalosti. Spolu s touto vlastností je hra vlastně velmi jednoduchá a hráč hraje buď za policisty, nebo za
teroristy. Tento jednoduchý formát si hra v téměř nezměněné podobě drží již od roku 2000.

Hra rostla zejména díky své komunitě. Hráči hru různě upravovali, přidávali další módy, typy her, zbraně, mapy či audiovizuální obsah. Ačkoli takto bylo doplňováno a upravováno
mnoho verzí hry,  první velký průlom přinesla verze 1.6. Counter-Strike 1.6 vynikal v oblasti esportu i v rámci hráčské komunity. Jen v České a Slovenské republice bylo několik
herních serverů, na kterých se mohly sejít sta tisíce hráčů. Např. na česko-slovenském herním portálu Kotelna ji hrálo celkem přes 1,5 milionu unikátních hráčů
\cite{cskocs_kotelna_2022}. Hra byla populární nejen mezi obyčejnými hráči, ale i mezi profesionály.

Counter-Strike 1.6 se stal pionýrem esportu pro \ac{FPS} žánr. Za podpory Valve se hrály první major\footnote{turnaj pořádaný přímo Valve, který má největší prestiž} turnaje,
kde hráči mohli ukázat svůj um za tehdy relativně velkou sumu peněz. V dnešní době majory trhají světové rekordy a dívají se na ně miliony
diváků\footnote{\scriptsize \url{https://www.invenglobal.com/articles/15619/csgo-major-breaks-viewership-records-overtakes-the-international}}
Hra se časem vyvíjela, hráči nalézali nové strategie či triky a Valve vydalo novou verzi ---  Counter-Strike: Source. Tato nová verze neměla příliš pozitivní ohlasy,
jelikož velmi rozdělila herní komunitu. Představila nové mechaniky a naučené mechaniky změnila. Hráči zejména v esportu nebyli ochotní se učit něco úplně nového. Valve se proto
rozhodlo, že herní komunitu opět sjednotí, a vydalo tak novou verzi hry s názvem \ac{CSGO}.

\ac{CSGO} se snažilo sjednotit oba tábory z her 
Counter-Strike 1.6 a Counter-Strike: Source. Hra vyšla 21. srpna 2012. Zpočátku nebyla příliš úspěšná, ale díky přidání různých
skinů \cite{valve_counterstrike_2013} na zbraně nakonec přilákala úplně nové publikum. Začaly se hrát menší esportové turnaje právě ve hře \ac{CSGO}, ke kterým se
později přidali i profesionálové z předchozích dvou verzí. I přes nástup hry \ac{CSGO} mají obě předchozí verze hry stále poměrně aktivní hráčskou základnu. I přes netradiční
interakci mezi Valve a herní komunitou hra stále roste. \ac{CSGO} se díky své dlouhé historii, bohaté komunitě a různým možnostem, jak lze hru hrát, dostala na špičku
esportu. I přes několik titulů, které se s ní snaží soupeřit, je tato hra stále největším a nejsledovanějším esport titulem v rámci \ac{FPS} žánru \cite{henningson_history_2020}.

\newpage
\section{Propojení závěrečné práce s titulem}
Práce se zaměřuje na identifikování významných prediktorů a následně na vytvoření regresního modelu. Před jakoukoliv prací s daty je však nutné pochopit, jak se hra vlastně hraje
a jaká jsou její pravidla. Ve hře \ac{CSGO} hraje pět hráčů proti pěti (dále jen 5v5). Většinou se hraje online, avšak velké esportové turnaje se hrají offline, např. v aréně.
Hra má v základu 30 kol a po  prvních patnácti se mění strany. Jedna strana jsou policisté (counter-terrorists či ct), kteří mají za úkol chránit \uv{bomboviště} --- část mapy,
jež má vybouchnout. Naopak cílem teroristů (t) je bombu položit a \uv{bomboviště} nechat vybouchnout. Vyhrává tým, který první vyhraje 16 kol. Pokud je však po prvních 30 kolech
stav nerozhodný, tedy 15:15, hraje se prodloužení. Tento formát není standardizovaný pro všechny turnaje, proto jsou zmíněné pouze pravidla, která se týkají turnajů od společnosti
Valve (již zmíněné a nejvíc prestižní majory). Na nich se hraje prodloužení ve formátu Bo6. Kdo první získá 4 body, vyhraje zápas. Pokud je stav nerozhodný, tedy 3:3, zápas se znovu
prodlouží. Takto se zápas teoreticky může prodlužovat do nekonečna. Nejdelší semi-profesionální zápas, který se ale neodehrál na majoru, proběhl mezi týmem exceL a
XENEX \cite{professeur_hltvorg_2015}. Zápas dospěl do úctyhodného 88. kola.

V každém kole má tým určitý počet peněz. Každý hráč začíná polovinu zápasu (tedy první a šestnácté kolo) s 800 \$. Stav financí každého hráče pak záleží na mnoha faktorech.
Vliv má například to, kolik vyhrál jeho tým kol v řadě, kolik nakoupil zbraní, kolik zabil nepřátel, kolik peněz hráč za zabití dostane či jak skončí kolo. V profesionálním
týmu je práce s financemi obzvlášť obtížná, jelikož tým musí počítat nejen svoje finance, ale i finance nepřátelského týmu. Nejen v této oblasti je důležitá role tzv. in-game
leadera (velitele týmu). Jedná se o nejdůležitější roli ze všech a bývá přiřazena pouze jednomu hráči v týmu. Ten pak má na starosti např. finance týmu, rozhoduje, kdy se co
koupí a kdy se budou finance šetřit na další kola, jaké se budou hrát mapy či jaká bude v daném kole zvolena strategie. V současnosti mívá in-game leader k dispozici i trenéra.
Ten hru nehraje, ale pozoruje hráče a dává jim různé tipy a rady.

Role trenéra není nijak silně definována a v každém esportovém týmu plní trochu jinou roli. V jednom případě může trenér působit čistě jako podpora, pomáhá tedy hráčům, když
se jim nedaří, a řeší interní problémy. V jiném týmu může ovšem do hry zasahovat mnohem více, především pokud pomáhá in-game leaderovi se strategiemi, obelstěním soupeře či
sledováním předchozích zápasů pro kontinuální zlepšování týmu. Dalšími rolemi v týmu mohou být například entry fragger (má za úkol získat první zabití pro tým), support
(podporuje svůj tým za pomoci různých granátů), AWP hráč (hraje primárně s odstřelovací puškou) a lurker (chodí po mapě sám a snaží se nepřítele
odchytnout ze stran, ze kterých by to nečekal).

Zápasy se hrají ve formátech \uv{best of}. Best of 3 například znamená, že se hrají tři mapy. Kdo první vyhraje dvě mapy, vyhraje celý zápas. Turnaje se pak odehrávají v
tradičních formátech. Klasickým formátem je pavouk, kde týmy postupují na pavučině zleva doprava, přičemž každý tým může prohrát pouze jednou. Dalším možným formátem je
upper/lower bracket formát, který je pavoukovi podobný, avšak obsahuje dvě \uv{sítě}. Druhá prohra tu znamená vyřazení z turnaje. Specifičtějším formátem pro \ac{CSGO}
je například swiss, který počítá různé body a statistiky výsledných zápasů.