\chapter{Teoretická část}
V následující části jsou popsány jak teoretické metody pro vizualizaci dat, tak i tvar, forma a vyhodnocení logistického regresního modelu. 
Ke každé části, která se věnuje popisu dat pomocí nějakého grafu, je přidána praktická ukázka s popisem a praktickým vysvětlením.
vhodné.
Testovací citace: \cite{Hebak2015}, \cite{Kleinbaum2010}

\section{Vizualizace dat} \label{sec:vizualizace_dat}
\subsection{Bodový graf}
Bodový graf slouží pro zobrazení vztahu dvou kvantitativních proměnných. Z pravidla se vysvětlovaná proměnná dává na osu Y,
zatímco proměnná vysvětlující se nachází na ose X. Vysvětlovaná (nezávislá) proměnná je ta proměnná, která má být předvídaná.
Vysvětlující proměnná se naopak snaží vysvětlovanou proměnnou předpovědět či popsat. Propojením vysvětlované a vysvětlující proměnné
na bodovém grafu lze vidět např. sílu korelace nebo vztah mezi proměnnými (např. lineární, kvadratický, logaritmický).

\begin{figure}[H]
    \centering
    \includegraphics{../obrazky/bodovy_graf_mtcars.png}
    \caption{Bodový graf hmotnosti a míly za galon} 
    \label{fig:bodovy_graf_mtcars}
\end{figure}

Obrázek \ref{fig:bodovy_graf_mtcars} zobrazuje negativní korelaci mezi hmotností vozidla a mílemi ujetými za galon.

\subsection{Sloupcový graf}
Sloupcový graf slouží k zobrazení četnosti kategorií. Na jednu osu (z pravidla osu X) se položí možné kategorie. Na druhou osu
se pak položí sledovaná statistika.
Sledovat můžeme nejen četnost, ale i průměr či kteroukoli jinou statistiku, kterou bude možné na ose zobrazit.

{\color{red}
Nejčastější se pomocí sloupcového grafu porovnává absolutní četnost dané kategorie. Řazení kategorií se dále odvíjí podle toho, zda je
daná proměnná ordinální či nominální. V případě nominální proměnné se sloupce řadí podle absolutní četnosti dané kategorie. V případě ordinální
proměnné se zachovává přirozené řazení.
}

Příklad sloupcového grafu je zobrazen na obrázku, který porovnává průměrnou hrubou koňskou sílu s počtem válců.
Je na něm také vidět vztah, kdy s vyšším počtem válců stoupá průměrná koňská síla.

\begin{figure}[H]
    \centering
    \includegraphics{../obrazky/sloupcovy_graf_mtcars.png}
    \caption{Sloupcový graf počtu válců a průměrné hrubé koňské síly} 
    \label{fig:sloupcovy_graf_mtcars}
\end{figure}

\subsection{Histogram}

Histogram je speciální typ sloupcového grafu. Hlavní rozdíl je v tom, že popisuje rozdělení spojité proměnné a mezi sloupci není žádná mezera.
Pro histogram je třeba data sloučit do skupin \textit{(bins)} o určité šířce. Správný výběr počtu skupin je kritický, jelikož může velmi
silně ovlivnit interpretaci dat. Pokud se vybere příliš malý počet skupin, data se seskupí a může se ztratit důležitý vztah. Pokud se ovšem
vybere moc velký počet skupin, v datech bude obtížné najít nějaký obecný vztah či trend.

\begin{figure}[H]
    \centering
    \includegraphics{../obrazky/histogram_porovnani_mtcars.png}
    \caption{Porovnání histogramů s různým počtem skupin} 
    \label{fig:histogram_porovnani_mtcars}
\end{figure}

Pro vhodný počet skupin existuje mnoho způsobů. Nejznámější je takzvané Sturgesovo pravidlo, které se spočítá následujícím vztahem:

\begin{equation}
    \label{eq:sturgesovo_pravidlo}
    k \text{ } \dot{\mathbf{=}} \text{ } 1 + 3,3 * log_{10}(n)
\end{equation}

kde $k$ je přibližný počet skupin a $n$ je počet pozorování. Druhý parametr, který je pro tvorbu histogramu potřeba, je šířka skupiny.
Ta by měla být ideálně stejná pro všechny skupiny. Pokud tomu tak není, histogram může být zavádějící a čtenář mu nemusí plně rozumět.
Pro vypočtení počtu skupin má šířka skupiny následující tvar:

\begin{equation}
    \label{eq:sirka_histogramu}
    w = \frac{max(x) - min(x)}{k}
\end{equation}

kde $x$ je zobrazovaná proměnná, $k$ je počet skupin a $w$ je výsledná šířka intervalu. 
Uplatněním rovnic \ref{eq:sturgesovo_pravidlo} a \ref{eq:sirka_histogramu} na datový soubor z obrázku \ref{fig:histogram_porovnani_mtcars}
lze zobrazit representativnější sloupcový graf. Nutné je však podotknout, že není pravidlem se danými výpočty řídit a výsledný
sloupcový graf je nutné přizpůsobit jednotlivým datům.
\begin{figure}[H]
    \centering
    \includegraphics{../obrazky/histogram_mtcars_sturges.png}
    \caption{Histogram s počtem skupin dle Sturgesova pravidla} 
    \label{fig:histogram_mtcars_sturges}
\end{figure}

\subsection{Krabičkový graf}
\subsubsection{Five-number summary}
Five-number summary je číselná tabulka, která pomocí pěti různých čísel shrnuje seřazenou číselnou řadu. Základní statistický nástroj pro
vytvoření takové tabulky jsou kvantily. Hodnota $P$-tého percentilu označuje číslo, které rozděluje seřazenou číselnou řadu na dva intervaly. 
První interval obsahuje $P*100\%$ číselné řady a druhý analogicky $(1-P)*100\%$. Různé hodnoty percentilů mohou mít specifičtější pojmenování a značí se $Q_P$.
Percentil $P = 0,5$ se označuje jako medián a rozděluje seřazenou číselnou řadu na polovinu. Percentily, kde $P = 0,25$ nebo $P = 0,75$, se označují
jako kvartily a značí se $Q_{1}$ a $Q_{3}$. Oba tyto typy kvartilů jsou použité při tvorbě Five-number summary tabulky. Jako příklad je
uvedena následující tabulka

\begin{table}[H]
    \centering
    \begin{tabular}[t]{c|c|c|c|c}
        \hline
        $Q_{0} (Q_0)$ & $Q_{0,25} (Q_1) $ & $Q_{0,50}$ & $Q_{0,75} (Q_3)$ & $Q_{1,00}$\\
        \hline
        1,513 & 2,58125 & 3,325 & 3,61 & 5,424\\
        \hline
    \end{tabular}
    \caption{\label{tab:five-number_summary}Five-number summary tabulka hmotnosti vozidla (lb/1000)}
\end{table}

kde $Q_{0}$ a $Q_{1,00}$ označují minimum a maximum číselné řady. Kvartily $Q_{1}$, $Q_{2}$ (medián) a $Q_{3}$ jsou čísla, která rozděluji časovou řadu na na čtvrtiny. V prvním
případě, tedy $Q_1 = Q_{0,25}$, je 25\% čísel menší než 1,513 a 75\% dat větší. Pro kvantil $Q_3 = Q_{0,75}$ je 75\% čísel menších než 3,61 a 25\% větších. $Q_{0,50}$ označuje medián.

\subsubsection{Krabičkový graf}
Krabičkový graf je grafické zobrazení a rozšíření Five-number summary tabulky. Kromě grafického zobrazení 
pěti kvantilů  ukazuje odlehlé a extrémní hodnoty.
V Krabičkovém grafu se také nachází obdélník, který ukazuje mezikvartilové rozpětí (IQR), tedy prostředních 50 \% dat. V obdélníku se také nachází černá čára, která značí medián.
Z prostředního obdélníku vedou oběma směry čáry, jejichž konce značí hranici pro odlehlá pozorování. Pokud datový soubor neobsahuje žádná odlehlá pozorování, konec těchto čar
značí minimum a maximum datového souboru.
Pozorování, která jsou buď větší než horní hranice, nebo menší než spodní hranice, označujeme jako odlehlé nebo extrémní. 

\begin{align}
    \textit{Spodní hranice } &= Q_1 - 1,5 * IQR \\
    \textit{Horní hranice } &= Q_3 + 1,5 * IQR
\end{align}

Hodnoty, které spadají do intervalu $\langle Q_1 - 1,5IQR; Q_1 - 3IQR\rangle$ a $\langle Q_3 + 1,5IQR, Q_3 + 3IQR \rangle$ se nazývají jako odlehlé.
Hodnoty které leží mimo tento vztah, tedy hodnoty menší než $Q_1 - 3IQR$ nebo větší než $Q_3 + 3IQR$ se nazývají jako hodnoty extrémní a
v krabičkovém grafu jsou z pravidla vyznačeny nějakým speciálním znakem, např. kolečkem.
Díky grafickému zobrazení lze lehce porovnávat rozdělení jedné vysvětlované kvantitativní proměnné tříděné přes několik kategorií.

\begin{figure}[H]
    \centering
    \includegraphics{../obrazky/krabickovy_graf_mtcars.png}
    \caption{Krabičkový graf hmotnosti auta pro různý počet válců} 
    \label{fig:krabickovy_graf_mtcars}
\end{figure}

Průhledná kolečka v obrázku \ref{fig:krabickovy_graf_mtcars} v kategorii osmi válců značí odlehlé hodnoty, t.j. hodnoty
v intervalu $\langle Q_3 + 1,5IQR, Q_3 + 3IQR \rangle$.

{\color{red}
\subsection{Korelační matice}
\subsubsection{Korelace}
Korelace popisuje směr a sílu vztahu mezi dvěma proměnnými $x$ a $y$. Značí se $r$ a nabývá hodnot $\langle -1; 1 \rangle$.

\begin{equation}
r = \frac{n \sum_{i=1}^n x_i y_i - \sum_{i=1}^n x_i \sum_{i=1}^n y_i}
{\sqrt{(\overline{x^2} - \overline{x}^2)(\overline{y^2} - \overline{y}^2)}}
\end{equation}

Čím větší je absolutní hodnota korelace mezi proměnnou $x$ a $y$, tím lépe lze pomocí jedné proměnné vysvětlit druhou proměnnou. Kladnost, případně zápornost korelace
značí pak směr vztahu. Pokud je korelace záporná, tedy $r < 0$, s růstem jedné proměnné klesá proměnná druhá. Naopak při kladné korelace, tedy $r > 0$, s růstem
jedné proměnné roste i druhá. Pokud by se například změřila výška žáku ve třídě, kde proměnná $x$ bude jejich naměřená výška v centimetrech
a proměnná $y$ jejich naměřená výška v metrech, korelace mezi proměnnou $x$ a $y$ bude rovna jedné. To je z toho důvodu, že výška v centimetrech lze
perfektně vysvětlit výškou v měřenou v metrech. To platí i naopak.  

\subsubsection{Korelační matice}
Korelační matice je nástroj, díky kterému lze zobrazit korelaci mezi více jak dvěma páry proměnných. Matice může být zobrazená jako tabulka nebo
graf. Korelační matice je velmi užitečná v regresní analýze kvůli předpokladu nezávislosti proměnných. Pokud jsou při tvorbě modelu prediktory korelované,
vzniká problém tzn. multikolinearity. Pokud jsou prediktory vysoce korelované, zhoršuje to přesnost a vypovídací hodnotu koeficientů. \cite{Kleinbaum2010}
V takovém případě je např zvýšit počet pozorování nebo z modelu určité modelované prediktory odebrat. 

\begin{figure}[H]
    \centering
    \includegraphics{../obrazky/mtcars_korelace.png}
    \caption{Korelační matice} 
    \label{fig:mtcars_korelace}
\end{figure}

Graf korelační matice může mít mnoho podob. V příkladu obrázku \ref{fig:mtcars_korelace} je zobrazená korelační matice jako teplotní mapa. Z obrázku je možné pozorovat vysokou
pozitivní korelaci mezi páry proměnných cyl, disp a hp. naopak skoro žádná korelace není mezi proměnnou qsec a proměnnou drat. Korelační matice je zároveň symetrická, jelikož
korelace mezi $x$ a $y$ je stejná jako korelace mezi $y$ a $x$. Díky této vlastnosti je možné zobrazit pouze část korelační matice pod úhlopříčkou bez ztráty jakékoliv informace.
}

\newpage
\section{Logistická regrese}
Logistická regrese je způsob, jak popsat vztah mezi jedním či několika prediktory a jednou binární vysvětlovanou 
proměnnou. K tomu slouží spojovací funkce, která transformuje lineární kombinaci prediktorů na index $z$. V případě
logistické regrese se tato funkce nazývá logistická a je definovaná jako

\begin{equation}
    \label{eq:logisticka_funkce}
    f(z) = \frac{1}{1 + e^{-z}}.
\end{equation}

Obor hodnot funkce je interval $\langle 0, 1 \rangle$. Proměnná $z$ je lineární kombinace prediktorů  $X_1, X_2, ..., X_k$, 
jejich koeficientů $\beta_1, \beta_2, ..., \beta_k$ a parametru $\alpha$.

\begin{equation}
    \label{eq:linearni_kombinace_z}
    \begin{split}
        z &= \alpha + \beta_1 X_1 + ... + \beta_2 X_2 +\beta_k X_k \\
          &= \alpha + \sum_{i=1}^k \beta_i X_i
    \end{split}
\end{equation}

Mějme tedy binární vysvětlovanou proměnnou $Y$, u které hodnota $1$ značí výskyt jevu. Pravděpodobnost, že jev nastane
vzhledem k definovaným prediktorům lze zapsat jako

\begin{equation}
    \label{eq:pravdepodobnost_y}
    P(Y = 1 \mid X_1, X_2, ..., X_k) = \frac{1}{1 + e^{- \left( \alpha + \sum_{i=1}^k \beta_i X_i \right) }},
\end{equation}

kde $\alpha$ a $\beta_i$ jsou parametry odhadnuté z datového souboru. 

\subsection{Interpretace parametrů}
Parametry $\alpha$ a $\beta_i$ značí logaritmus šance. $\alpha$ je logaritmus šance v případě, že všechny prediktory
jsou teoreticky rovné $0$. Parametr $\beta_i$ značí logaritmus šance pro prediktor $X_i$
V případě, že všechny prediktory jsou konstantní a prediktor $X_i$ se změní o jednotku, přirozený logaritmus
šance se změní o $B_i$. Toto lze pozorovat například u binárních prediktorů, kdy typicky přítomnost
daného prediktoru, značená jedničkou, změní výslednou šanci právě o odhadnutý parametr $\beta$.
Pro přechod z přirozeného logaritmu šance na šanci lze využít vztahu

\begin{equation}
\textit{šance } = e^{\beta_i}.
\end{equation}

Šance je podíl dvou pravděpodobností. Pokud bychom měli šanci jevu A oproti jevu B $2 : 1$, značí to, že výskyt jevu A je dvakrát tak pravděpodobný
jako výskyt jevu B a jev A se vyskytuje ve $\frac{2}{3}$ případů. Šance $e^{\beta_i}$ tedy značí vztah mezi prediktorem $X_i$ a vysvětlovanou proměnnou $Y$. Pokud je
šance kladná, značí to, že s vyšší hodnotou prediktoru $X_i$ se zvyšuje šance že $P(Y = 1)$. Pokud je naopak nižší, pravděpodobnost se zmenšuje. Pokud je potřeba
interpretovat pravděpodobnost jako šanci, použije se logitová funkce

\begin{equation}
    \label{eq:logitova_funkce}
    \textit{šance jevu A} = \frac{p}{1 - p}
\end{equation}

kde $p$ je pravděpodobnost výskytu jevu A.

\subsection{Největší pravděpodobnost}
Parametry logistického modelu v rovnici \ref{eq:pravdepodobnost_y} jsou pouze teoretické a je třeba je odhadnout. Již vypočtené odhady
se proto neznačí pouze $\beta$, ale $\hat{\beta}$. Pro odhad parametrů se při logistické regresi používá metoda maximální věrohodnosti. Pro výpočet
maximální věrohodnosti se počítá věrohodnostní funkce $L(\theta)$ kde $\theta$ jsou parametry logistického modelu $\alpha, \beta_1, ..., \beta_k$.
Pro logistickou regresi má věrohodnostní funkce tvar

\begin{equation}
    \label{eq:pravdepodobnostni_fce}
    L(\theta) = \Pi_{l = 1}^{m_1} P(X_i) \Pi_{l = m_1 + 1}^{n} 1 - P(X_i),
\end{equation}

kde $n$ je počet pozorování a $m_1$ je počet příznivých ($Y = 1$) jevů. Funkce předpokládá, že datový soubor je seřazen tak, že prvních $m_1$ výskytů
jsou jevy příznivé. $P(X_i)$ poté značí logistickou funkci \ref{eq:logisticka_funkce}. Pro vypočtení optimálního parametru $\beta_i$ je nutné vypočítat
maximum funkce $L(\theta)$ vzhledem k parametru $\beta_i$. Parametr $\beta_i$ lze tedy získat derivací funkce $L(\theta)$ vzhledem k parametru $\beta_i$

\begin{equation}
    \frac{\partial L(\theta)}{\partial \beta_i} = 0
\end{equation}

\newpage
\subsection{Matice záměn}
Matice záměn je nástroj pro vyhodnocení predikcí modelu. Matice je o velikosti $n \times n$. Pro potřeby logistické regrese se matice skládá ze dvou řádku a dvou sloupců.
V řádcích se nachází původní hodnoty, tedy hodnoty, které chceme předpovídat. Ve sloupcích se pak nachází předpovědi.  

\begin{table}[H]
    \centering
    \begin{tabular}{|c|c|c|c|}
        \hline
                          &   & Pozitivní predikce & Negativní predikce \\ \hline
                          &   & 1                  & 0                  \\ \hline
        Původní pozitivní & 1 & Skutečně pozitivní & Falešně negativní  \\ \hline
        Původní negativní & 0 & Falešně pozitivní  & Skutečně negativní \\ \hline
    \end{tabular}
    \caption{\label{tab:matice_zamen}Matice záměn}
\end{table}

Pro sestavení matice je potřeba množina dat, u kterých známe vysvětlovanou proměnnou. Na datech pak provedeme predikci, díky čemuž získáme predikované hodnoty. Porovnáním
původních a predikovaných hodnot vznikne matice \ref{tab:matice_zamen}. Každá ze čtyř vnitřních buněk má vlastní označení a interpretaci:

\begin{itemize}
    \item \textbf{Skutečně pozitivní} - počet správných predikcí, které byly rovné jedné
    \item \textbf{Falešně pozitivní} - počet predikcí rovných jedné, kde byla původní hodnota rovná nule
    \item \textbf{Skutečně negativní} - počet správných predikcí, které byly rovné nule
    \item \textbf{Falešně negativní} - počet predikcí rovných nule, kde byla původní hodnota rovná jedné
\end{itemize}

Z matice lze následně vypočítat mnoho statistik. Pro vyhodnocení regresního modelu lze použít např. přesnost, která se vypočítá jako počet všech správných predikcí nad
počtem všech provedených predikcí $n$.

\begin{equation}
    \textit{Přesnost = } \frac{\text{Skutečně pozitivní} + \text{Skutečně negativní}}{n}
\end{equation}

{\color{red}
Přesnost říká, kolik predikcí bylo klasifikováno správně. Pokud je ovšem poměr původních pozitivních a negativních hodnot velmi nevyrovnaný, tato statistika
není vhodná. Toto se může stát například v lékařství při identifikaci nemocného pacienta. Zde je důležitější úspěšně predikovat výskyt nemoci, který se ovšem v datovém
souboru často objevuje v menším poměru. V tomto případě lze použít statistiku zvanou senzitivita. Ta se rovná poměru správných pozitivních predikcí a
úhrnu všech pozitivních predikcí

\begin{equation}
    \label{eq:senzitivita}
    \textit{Senzitivita = } \frac{\text{Skutečně pozitivní}}{\text{Skutečně pozitivní} + \text{Falešně pozitivní}}
\end{equation}

Senzitivita tedy určuje poměr správně klasifikovaných pozitivních případů a všech pozitivně klasifikovaných případů. Pokud by bylo vhodné preferovat
spíše negativní klasifikace, lze použit statistiku zvanou specificita.

\begin{equation}
    \label{eq:specificita}
    \textit{Specificita = } \frac{\text{Skutečně negativní}}{\text{Skutečně negativní} + \text{Falešně negativní}}
\end{equation}

Nanesením sensitivity \ref{eq:senzitivita} a specificity \ref{eq:specificita} lze vytvořit tzn. ROC křivku.
}

{\color{red}
\subsection{Testování hypotéz}
... Dopsat
}

\subsection{Waldův test}
Waldův test ověřuje, zda je parametr $\beta_i$ v populaci významný či nikoliv. Definice testu hypotézy je tedy

$H_0:$ Koeficient $\beta_i$ je rovný nule \\
$H_A:$ Koeficient $\beta_i$ je různý od nuly.

Pro vyhodnocení hypotézy se používá kritická hodnota $Z$, který se vypočítá jako poměr testovaného parametru $S_{\hat{\beta_i}}$
a směrodatné chyby koeficientu $\beta_i$

\begin{equation}
    Z = \frac{\hat{\beta_i}}{S_{\hat{\beta_i}}}.
\end{equation}

Kritická hodnota $Z$ má za platnosti nulové hypotézy normální $N(0, 1)$ a její mocnina, $Z^2$, má chi-kvadrát rozdělení s jedním stupněm volnosti.
