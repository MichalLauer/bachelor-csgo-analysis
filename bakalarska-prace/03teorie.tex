\chapter{Teoretická část}
V následující části jsou popsány jak teoretické metody pro vizualizaci dat, tak i tvar, forma a vyhodnocení logistického regresního modelu. 
Ke každé části, která se věnuje popisu dat pomocí nějakého grafu, je přidána praktická ukázka s popisem a praktickým vysvětlením. U regresního 
modelu je následně dán důraz na teorii, jelikož model je následně prakticky použit v celé třetí části. Důležité je si uvědomit, že nástrojů pro
vizualizaci dat je mnoho a grafů či souhrnných statistik je nespočet. V této práci se ovšem věnujeme vizualizacím, které jsou pro regresní model
vhodné.
\section{Vizualizace dat}
... Půjčit knihu https://katalog.vse.cz/Record/

\subsection{Bodový graf}
... Půjčit knihu https://katalog.vse.cz/Record/

\subsection{Histogram}
... Půjčit knihu https://katalog.vse.cz/Record/

\subsection{Boxplot}
... Půjčit knihu https://katalog.vse.cz/Record/

\section{Logistický model}
\subsection{Logistický regresní model}
Logistický regresní model patří do rodiny Obecných lineárních modelů. Je to tedy zobecněný lineární model, který uvažuje tzn. spojovací funkci, která transformuje
nezávislou proměnnou $y$. Spojovací funkce, značená $g(x)$ se v případě logistického regresního modelu nazývá funkce \textit{logitová}.
\begin{equation}
    g(x) = ln\left(\frac{x}{1 - x}\right)
\end{equation}
Logitová funkce se dá také nazvat \uv{log-odds ratio}, neboli zlogaritmizovaný poměr mezi dvěma šancemi. To je z toho důvodu, že zlomek v přirozením logaritmu, tedy
$\frac{x}{1-x}$ se používá na převod pravděpodobnosti na šanci. Jako příklad si můžeme uvést pravděpodobnost $p = 0.2$, která je po transformaci rovna šanci $1:4$.

\begin{equation}
    g(f(x)) = \beta_0 + \sum_{i=1}^k \beta_i*x_i
\end{equation} 
Logistická regrese se snaží předpovědět s jakou pravděpodobností se stane určitý jev. Zde ale vzniká, problém, jelikož klasický regresní model je zapsán ve tvaru
\begin{equation}
    f(y) = \beta_0 + \sum_{i=1}^k \beta_i * x_i
\end{equation} 
kde $\beta_i$ je parametr pro proměnnou $x_i$. Protože proměnné $x$ nejsou ničím omezené, obor hodnot pro $f(y)$ je interval $(-\inf, \inf)$. Toto však je v základním rozporu
s pravděpodobnostmi. Ty jsou totiž definované v intervalu $<0, 1>$. Tento problém vyřeší právě spojovací funkce, která  

\subsection{Jednoduchý tvar modelu s dichotomickou proměnnou}
\subsection{Jednoduchý tvar modelu s interakcí}
\subsection{Model s více proměnnými}
\subsection{Vyhodnocení logistického modelu}