\chapter{Teoretická část}
V následující části jsou popsány jak teoretické metody pro vizualizaci dat, tak i tvar, forma a vyhodnocení logistického regresního modelu. 
Ke každé části, která se věnuje popisu dat pomocí nějakého grafu, je přidána praktická ukázka s popisem a praktickým vysvětlením. U regresního 
modelu je následně dán důraz na teorii, jelikož model je následně prakticky použit v celé třetí části. Důležité je si uvědomit, že nástrojů pro
vizualizaci dat je mnoho a grafů či souhrnných statistik je nespočet. V této práci se ovšem věnujeme vizualizacím, které jsou pro regresní model
vhodné.
\section{Vizualizace dat}
... Půjčit knihu https://katalog.vse.cz/Record/

\subsection{Bodový graf}
... Půjčit knihu https://katalog.vse.cz/Record/

\subsection{Histogram}
... Půjčit knihu https://katalog.vse.cz/Record/

\subsection{Boxplot}
... Půjčit knihu https://katalog.vse.cz/Record/
\newpage
\section{Logistický model}
\subsection{Logistický regresní model}
Logistický regresní model patří do rodiny Obecných lineárních modelů. Je to tedy zobecněný lineární model, který uvažuje tzn. spojovací funkci, která transformuje
proměnnou $y$. Proměnná, kterou chceme predikovat, musí být dichotomická. Spojovací funkce, značená $g(x)$ se v případě logistického regresního modelu nazývá funkce
\textit{logitová} \cite{Hebak2015}
\begin{equation}
    \label{eq:spojovaci_fce}
    g(x) = ln\left(\frac{x}{1 - x}\right)
\end{equation}
Logitová funkce se dá také nazvat \uv{log-odds ratio}, neboli zlogaritmizovaný poměr mezi dvěma šancemi. To je z toho důvodu, že zlomek v přirozením logaritmu, tedy
$\frac{x}{1-x}$ se používá na převod pravděpodobnosti na šanci. Jako příklad si můžeme uvést pravděpodobnost $p = 0.2$, která se po transformaci rovna šanci $1:4$.
K logitové funkci existuje i její funkce inverzní, která se nazývá logistická. S jejím použitím můžeme provádět opačný převod, tedy šanci na pravděpodobnost.
\begin{equation}
    \label{eq:logisticka_fce}
    g^{-1}(x) = \frac{1}{1 + e^{-z}}
\end{equation}
Zde nám proměnná $z$ značí výslednou predikci, neboli šanci, získanou z regresního modelu
\begin{equation}
    z = \beta_0 + \sum_{i=1}^k \beta_i*x_i
\end{equation}
kde $\beta$ značí odhadnuté parametry a $x$ zvolené prediktory. $\beta_0$ nám zde značí průnik s osou $y$, což lze také pochopit jako počáteční šanci, že se událost stane.
Důvod, proč používáme pro predikce šanci, je kvůli použití regresní přímky. Při hledání vhodných parametrů $\beta$ se snažíme najít přímku, která v případě použití 
metody \textit{nejmenších čtverců} snaží minimalizovat vzdálenost od každého bodu k přímce. Jak již jsme si ale zmínili, pravděpodobnost je v jiném intervalu, než v kterém
se nachází výsledná regresní přímka. Z tohoto důvodu používáme inverzi spojovací funkce na transformaci výsledné predikce
\begin{equation}
    g^{-1}(f(x)) = \beta_0 + \sum_{i=1}^k \beta_i*x_i
\end{equation}
Výsledný regresní graf je dán tzn. \uv{Sigmoidní křivkou}. Ta má obor hodnot v požadovaném intervalu, tedy $<0, 1>$, a zároveň je definovaná pro
všechna $\mathbb{R}$.

VLOŽIT OBRÁZEK FCE

Regresní model jde tedy zapsat dvěma způsoby. První způsob, tzn. \uv{Logitový zápis}, používá spojovací funkci společně s pravděpodobností $P(X)$ která značí, se událost stala.
$P(X)$ lze tedy označit jako pravděpodobnost, že se událost stane, za předpokladu různých nezávislých proměnných $x$, neboli $P(X = 1 | x_1, ..., x_n)$
\begin{equation}
\begin{split}
    \beta_0 + \sum_{i=1}^k \beta_i*x_i &= g(P(X)) \\
                                      &= \frac{p(x)}{1 - p(x)}
\end{split}
\end{equation}
Druhý zápis definuje danou pravděpodobnost za použití logistické funkce
\begin{align}
\begin{split}
    P(X) &= g^{-1}(f(x)) \\
        &= \frac{1}{1+e^{-z}} \\
        &= \frac{1}{1+e^{\beta_0 + \sum_{i=1}^k \beta_i*x_i}}    
\end{split}
\end{align} 
\subsection{Jednoduchý tvar modelu s dichotomickou proměnnou}
\subsection{Jednoduchý tvar modelu s interakcí}
\subsection{Model s více proměnnými}
\subsection{Vyhodnocení logistického modelu}