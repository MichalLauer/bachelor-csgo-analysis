\chapter{Teoretická část}
V následující části jsou popsány jak teoretické metody pro vizualizaci dat, tak i tvar, forma a vyhodnocení logistického regresního modelu. 
Ke každé části, která se věnuje popisu dat pomocí nějakého grafu, je přidána praktická ukázka s popisem a praktickým vysvětlením. U regresního 
modelu je následně dán důraz na teorii, jelikož model je následně prakticky použit v celé třetí části. Důležité je si uvědomit, že nástrojů pro
vizualizaci dat je mnoho a grafů či souhrnných statistik je nespočet. V této práci se ovšem věnujeme vizualizacím, které jsou pro regresní model
vhodné.
Testovací citace: \cite{Hebak2015}, \cite{Kleinbaum2010}
\section{Vizualizace dat}
... Dopsat

\subsection{Bodový graf}
... Dopsat

\subsection{Histogram}
... Dopsat

\subsection{Boxplot}
... Dopsat

\newpage

\section{Logistický regresní model}
... Dopsat

\subsection{Jednoduchý logistický regresní model}
... Dopsat

\subsection{Vícerozměrný logistický regresní model}
... Dopsat

\subsection{Optimalizace parametrů}
... Dopsat

\subsubsection{Metoda nejmenších čtverců}
... Dopsat

\subsubsection{Metoda největší pravděpodobnosti}
... Dopsat

\subsection{Vyhodnocení modelu}
... Dopsat
