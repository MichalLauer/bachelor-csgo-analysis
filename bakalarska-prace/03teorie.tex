\chapter{Teoretická část}
V následující části jsou popsány jak teoretické metody pro vizualizaci dat, tak i tvar, forma a vyhodnocení logistického regresního modelu. 
Ke každé části, která se věnuje popisu dat pomocí nějakého grafu, je přidána praktická ukázka s popisem a praktickým vysvětlením. U regresního 
modelu je následně dán důraz na teorii, jelikož model je následně prakticky použit v celé třetí části. Důležité je si uvědomit, že nástrojů pro
vizualizaci dat je mnoho a grafů či souhrnných statistik je nespočet. V této práci se ovšem věnujeme vizualizacím, které jsou pro regresní model
vhodné.
Testovací citace: \cite{Hebak2015}, \cite{Kleinbaum2010}
\section{Vizualizace dat}
... Dopsat

\subsection{Bodový graf}
Bodový graf slouží pro zobrazení vztahu dvou numerických proměnných. Z pravidla se vysvětlovaná proměnná dává na osu Y,
zatímco proměnná vysvětlující se nachází na ose X.  
\newpage
\subsection{Histogram}
Histogram je speciální typ sloupcového grafu. Je specifický tím, že zobrazuje počet jednotlivých pozorování dané kategorie či proměnné. Jeho silná stránka spočívá v tom,
že při vhodném výběru počtu skupin se dá pozorovat rozdělení spojité proměnné.
Zároveň histogram oproti klasickému sloupcovému grafu, který zobrazuje kategorie, nemá mezi sloupci žádné mezery. Na grafu 
\ref{fig:sloupcovy_graf} je příklad sloupcového grafu. Na ose X se nachází kategorická vysvětlovací proměnná. Vysvětlovaná proměnná Y je průměrná hrubá
koňská síla.

\begin{figure}[H]
    \centering
    \includegraphics{../../analyza/plots/sloupcovy_graf.png}
    \caption{Sloupcový graf z data setu mtcars} 
    \label{fig:sloupcovy_graf}
\end{figure}

Je-li histogram sestaven ze spojitých dat, je třeba data sloučit do skupin \textit{(bins)} o určité šířce. Správný výběr
počtu skupin je kritický, jelikož může velmi ovlivnit interpretaci dat. Pokud se vybere moc malý počet skupin, data se moc seskupí a
může se ztratit důležitý vztah. Pokud se ovšem vybere moc velký počet skupin, data se prakticky nerozdělí a nic nám to o rozdělení proměnné neřekne.
Tento efekt je znázorněn na grafu \ref{fig:histogram_porovnani}.

\begin{figure}[H]
    \centering
    \includegraphics{../../analyza/plots/histogram_porovnani.png}
    \caption{Porovnání histogramů s různým počtem skupin} 
    \label{fig:histogram_porovnani}
\end{figure}

Pro vhodný počet skupin existuje mnoho způsobů. Nejznámější je takzvané Sturgesovo pravidlo (\ref{eq:sturgesovo_pravidlo}), které se spočítá následujícím vztahem:

\begin{equation}
    \label{eq:sturgesovo_pravidlo}
    k \text{ } \dot{\mathbf{=}} \text{ } 1 + 3,22 * log_{10}(n)
\end{equation}

kde $k$ je výsledný zaokrouhlený počet skupin a $n$ je počet pozorování. Druhý parametr, který je pro tvorbu histogramu potřeba, je šířka skupiny.
Ta by měla být ideálně stejná pro všechny skupiny. Pokud tomu tak není, histogram může být zavádějící a čtenář mu nemusí plně rozumět.
Po vypočtení počtu skupin má šířka skupiny následující tvar:

\begin{equation}
    \label{eq:sirka_histogramu}
    w = \ceil[\bigg]{ \frac{max(x) - min(x)}{k} }
\end{equation}

kde $x$ je zobrazovaná proměnná, $k$ je počet skupin a $w$ je výsledná šířka intervalu, která je zaokrouhlena nahoru. To je z toho důvodu, aby se
každé pozorování promítlo právě do jedné kategorie. Pokud na stejný dataset, jako na grafu \ref{fig:histogram_porovnani}, použije toto pravidlo,
graf vypadá následovně:

\begin{figure}[H]
    \centering
    \includegraphics{../../analyza/plots/histogram.png}
    \caption{Histogram s počtem skupin dle Sturgesova pravidla} 
    \label{fig:histogram_sturges}
\end{figure}

Histogram lze samozřejmě rozepsat. Lze vytvořit tzn. tabulku četností (\ref{tab:tabulka_cetnosti_sturges}),
která mimo jiné obsahuje spodní hranici intervalu, horní hranici intervalu a počet pozorování. Důležité je,
aby přechody mezi intervaly byli jasné.

\begin{table}[H]
    \centering
    \begin{tabular}[t]{c|c|c}
        \hline
        Spodní hranice intervalu & Horní hranice intervalu & Počet pozorování \\
        \hline
         7.05 & 11.75 & 2\\
        \hline
        11.76 & 16.45 & 9\\
        \hline
        16.46 & 21.15 & 9\\
        \hline
        21.16 & 25.85 & 6\\
        \hline
        25.86 & 30.55 & 4\\
        \hline
        30.56 & 35.25 & 2\\
        \hline
    \end{tabular}
    \caption{Tabulka četnostní}
    \label{tab:tabulka_cetnosti_sturges}
\end{table}

\subsection{Boxplot}
... Dopsat

\newpage

\section{Logistický regresní model}
... Dopsat

\subsection{Jednoduchý logistický regresní model}
... Dopsat

\subsection{Vícerozměrný logistický regresní model}
... Dopsat

\subsection{Optimalizace parametrů}
... Dopsat

\subsubsection{Metoda nejmenších čtverců}
... Dopsat

\subsubsection{Metoda největší pravděpodobnosti}
... Dopsat

\subsection{Vyhodnocení modelu}
... Dopsat
