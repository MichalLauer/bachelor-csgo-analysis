\chapter{Teoretická část}
V následující části jsou popsány jak teoretické metody pro vizualizaci dat, tak i tvar, forma a vyhodnocení logistického regresního modelu. 
Ke každé části, která se věnuje popisu dat pomocí nějakého grafu, je přidána praktická ukázka s popisem a praktickým vysvětlením.
vhodné.
Testovací citace: \cite{Hebak2015}, \cite{Kleinbaum2010}

\section{Vizualizace dat}
\subsection{Bodový graf}
{\color{red}
Bodový graf slouží pro zobrazení vztahu dvou numerických proměnných. Z pravidla se vysvětlovaná proměnná dává na osu Y,
zatímco proměnná vysvětlující se nachází na ose X. Vysvětlovaná (nezávislá) proměnná je ta proměnná, která má být určitým způsobem předvídaná.
Vysvětlující proměnná se naopak snaží vysvětlovanou proměnnou předpovědět či nějakým způsobem popsat. Propojením vysvětlované a vysvětlující proměnné
na bodovém grafu lze vidět např. sílu korelace nebo vztah mezi proměnnými (např. lineární, kvadratický, logaritmický). Graf \ref{fig:bodovy_graf_mtcars} zobrazuje
negativní korelaci mezi váhou vozidla a mílemi ujetými za galon.
}

\begin{figure}[H]
    \centering
    \includegraphics{../../analyza/plots/bodovy_graf_mtcars.png}
    \caption{Bodový graf váhy a míly za galon z data setu mtcars} 
    \label{fig:bodovy_graf_mtcars}
\end{figure}

{\color{red}
\subsection{Sloupcový graf}
Sloupcový graf slouží k porovnání kategoriální kvalitativní proměnné. Na jednu osu (z pravidla osu X) se položí možné kategorie. Na druhou osu
se pak položí sledovaná statistika. Sledovat můžeme např. počet výskytů, průměr, nebo relativní počet výskytu. Pokud je sledovaná proměnná ordinální,
je také možné odvodit vztah mezi kategoriemi. Příklad sloupcového grafu je zobrazen na obrázku \ref{fig:sloupcovy_graf_mtcars}, který
porovnává průměrnou hrubou koňskou sílu s počtem válců. Je na něm také vidět vztah, kdy s vyšším počtem válců stoupá průměrná koňská síla.

\begin{figure}[H]
    \centering
    \includegraphics{../../analyza/plots/sloupcovy_graf_mtcars.png}
    \caption{Sloupcový graf z data setu mtcars} 
    \label{fig:sloupcovy_graf_mtcars}
\end{figure}

\subsection{Histogram}

Histogram je speciální typ sloupcového grafu. Hlavní rozdíl je v tom, že popisuje bodové rozdělení spojité proměnné a mezi sloupci není žádná mezera.
}
Pro histogram je třeba data sloučit do skupin \textit{(bins)} o určité šířce. Správný výběr počtu skupin je kritický, jelikož může velmi
silně ovlivnit interpretaci dat. Pokud se vybere moc malý počet skupin, data se seskupí a může se ztratit důležitý vztah. Pokud se ovšem
vybere moc velký počet skupin, v datech bude obtížné najít nějaký obecný vztah či trend.
Tento efekt je znázorněn na obrázku \ref{fig:histogram_porovnani_mtcars}.

\begin{figure}[H]
    \centering
    \includegraphics{../../analyza/plots/histogram_porovnani_mtcars.png}
    \caption{Porovnání histogramů s různým počtem skupin} 
    \label{fig:histogram_porovnani_mtcars}
\end{figure}

Pro vhodný počet skupin existuje mnoho způsobů. Nejznámější je takzvané Sturgesovo pravidlo, které se spočítá následujícím vztahem:

\begin{equation}
    \label{eq:sturgesovo_pravidlo}
    k \text{ } \dot{\mathbf{=}} \text{ } 1 + 3,3 * log_{10}(n)
\end{equation}

kde $k$ je výsledný zaokrouhlený počet skupin 
{\color{red}
nahoru
}
a $n$ je počet pozorování. Druhý parametr, který je pro tvorbu histogramu potřeba, je šířka skupiny.
Ta by měla být ideálně stejná pro všechny skupiny. Pokud tomu tak není, histogram může být zavádějící a čtenář mu nemusí plně rozumět.
Pro vypočtení počtu skupin má šířka skupiny následující tvar:

\begin{equation}
    \label{eq:sirka_histogramu}
    w = \frac{max(x) - min(x)}{k}
\end{equation}

{\color{red}
kde $x$ je zobrazovaná proměnná, $k$ je počet skupin a $w$ je výsledná šířka intervalu. Pokud na stejný dataset, jako na obrázku 
\ref{fig:histogram_porovnani_mtcars}, použije Sturgesovo pravidlo \ref{eq:sturgesovo_pravidlo} a výpočet šířky \ref{eq:sirka_histogramu},
obrázek vypadá následovně:
}

\begin{figure}[H]
    \centering
    \includegraphics{../../analyza/plots/histogram_mtcars_sturges.png}
    \caption{Histogram s počtem skupin dle Sturgesova pravidla} 
    \label{fig:histogram_mtcars_sturges}
\end{figure}

Histogram lze samozřejmě rozepsat. Lze vytvořit tzn. tabulku četností ,
která mimo jiné obsahuje spodní hranici intervalu, horní hranici intervalu a počet pozorování. Důležité je,
aby přechody mezi intervaly byli jasné.

\begin{table}[H]
    \centering
    \begin{tabular}[t]{c|c|c}
        \hline
        Spodní hranice intervalu & Horní hranice intervalu & Počet pozorování\\
        \hline
        9.791667 & 13.70833 & 3\\
        \hline
        13.708334 & 17.62500 & 9\\
        \hline
        17.625001 & 21.54167 & 11\\
        \hline
        21.541668 & 25.45833 & 3\\
        \hline
        25.458334 & 29.37500 & 2\\
        \hline
        29.375001 & 33.29167 & 3\\
        \hline
        33.291668 & 37.20833 & 1\\
        \hline
    \end{tabular}
    \caption{\label{tab:tabulka_cetnosti_sturges}Tabulka četnostní}
\end{table}

\subsection{Boxplot}
\subsubsection{Five-number summary}
Five-number summary je číselná tabulka, která pomocí pěti různých čísel shrnuje seřazenou číselnou řadu. Základní statistický nástroj pro
vytvoření takové tabulky jsou kvantily. Hodnota $P$-tého percentilu označuje číslo, které rozděluje seřazenou číselnou řadu na dva intervaly. 
První interval obsahuje $P*100\%$ číselné řady a druhý analogicky $(1-P)*100\%$. Různé hodnoty percentilů mohou mít specifičtější pojmenování a značí se $Q_P$.
Percentil $P = 0.5$ se označuje jako medián a rozděluje seřazenou číselnou řadu na polovinu. Percentily, kde $P = 0.25$ nebo $P = 0.75$, se označují
jako kvartily a značí se $Q_{1}$ a $Q_{3}$. Oba tyto typy kvartilů jsou použité při tvorbě Five-number summary tabulky. Jako příklad je
uvedena tabulka \ref{tab:five-number_summary}

\begin{table}[H]

    \centering
    \begin{tabular}[t]{c|c|c|c|c}
        \hline
        $Q_{0} (Q_0)$ & $Q_{0.25} (Q_1) $ & $Q_{0.50}$ & $Q_{0.75} (Q_3)$ & $Q_{1.00}$\\
        \hline
        1.513 & 2.58125 & 3.325 & 3.61 & 5.424\\
        \hline
    \end{tabular}
    \caption{\label{tab:five-number_summary}Five-number summary tabulka hmotnosti vozidla (lb/1000)}
\end{table}

$Q_{0}$ a $Q_{1.00}$ označují minimum a maximum číselné řady. Kvartily $Q_{1}$ a $Q_{3}$ jsou čísla, která rozděluji časovou řadu na na čtvrtiny. V prvním
případě, tedy $Q_1 = Q_{0.25}$, je 25\% čísel menší než 1.513 a 75\% dat větší. Pro kvantil $Q_3 = Q_{0.75}$ je 75\% čísel menších než 3.61 a 25\% větších. $Q_{0.50}$ označuje medián.

\subsubsection{Boxplot}
Boxplot je grafické zobrazení a rozšíření Five-number summary tabulky. Krom grafické zobrazení 
{\color{red}
pěti kvantilů  ukazuje odlehlé a extrémní hodnoty.
}
V boxplotu se také nachází obdélník, který ukazuje mezikvartilové rozsah (IQR), tedy prostředních 50 \% dat. V obdélníku se také nachází černá čára, která značí medián.
Z prostředního obdélníku vedou oběma směry čáry, které značí teoretické minimum a maximum
{\color{red}
(což mohou a nemusí být hodnoty $Q_0$ a $Q_1$).
}
Obě dve hranice se počítají následovně

\begin{align}
    min &= Q_1 - 1.5 * IQR \\
    max &= Q_3 + 1.5 * IQR
\end{align}

{\color{red}
Hodnoty, které spadají do intervalu $\langle Q_1 - 1.5IQR; Q_1 - 3IQR\rangle$ a $\langle Q_3 + 1.5IQR, Q_3 + 3IQR \rangle$ se nazývají jako odlehlé.
Hodnoty které leží mimo tento vztah, tedy hodnoty menší než $Q_1 - 3IQR$ nebo větší než $Q_3 + 3IQR$ se nazývají jako hodnoty extrémní a
v boxplotu jsou z pravidla vyznačeny nějakým speciálním znakem, např. kolečkem.
Díky grafickému zobrazení lze lehce porovnávat rozdělení jedné vysvětlované proměnné tříděné přes několik kategorií.
}

\begin{figure}[H]
    \centering
    \includegraphics{../../analyza/plots/boxplot_mtcars.png}
    \caption{Boxplot váhy auta pro různé počty válců} 
    \label{fig:boxplot_mtcars}
\end{figure}

{\color{red}
Černé tečky v obrázku \ref{fig:boxplot_mtcars} v kategorii osmi válců značí odlehlé hodnoty, t. j. hodnoty
v intervalu $\langle Q_3 + 1.5IQR, Q_3 + 3IQR \rangle$.
}

\newpage
\section{Logistická regrese}
Logistická regrese je způsob, jak popsat vztah mezi jedním či několika prediktory a jednou binární predikovanou 
proměnnou. K tomu slouží spojovací funkce, která transformuje lineární kombinaci prediktorů na index $z$. V případě
logistické regrese se tato funkce nazývá logistická a je definovaná jako

\begin{equation}
    \label{eq:logisticka_funkce}
    f(z) = \frac{1}{1 + e^{-z}}
\end{equation}

Obor hodnot funkce je interval $\langle 0, 1 \rangle$. Proměnná $z$ je lineární kombinace prediktorů  $X_1, X_2, ..., X_k$,
jejich koeficienty $\beta_1, \beta_2, ..., \beta_k$ a parametru $\alpha$.

\begin{equation}
    \label{eq:linearni_kombinace_z}
    z = \alpha + \sum_{i=1}^k \beta_i X_i
\end{equation}

Mějme tedy binární predikovanou (vysvětlovanou) proměnnou $Y$, u které hodnota $1$ značí výskyt jevu. Pravděpodobnost, že jev nastane
vzhledem k definovaným prediktorům lze zapsat jako

\begin{equation}
    \label{eq:pravdepodobnost_y}
    P(Y = 1 \mid X_1, X_2, ..., X_k) = \frac{1}{1 + e^{-\alpha + \sum_{i=1}^k \beta_i X_i}}
\end{equation}

kde $\alpha$ a $\beta_i$ jsou parametry odhadnuté z datového souboru. 

\subsection{Interpretace parametrů}
Parametry $\alpha$ a $\beta_i$ značí logaritmus šance. $\alpha$ je  logaritmus šance, kde všechny prediktory
jsou teoreticky rovné $0$. Parametr $\beta_i$ značí logaritmus šance pro prediktor $X_1$ při změně o jednotku. 
Pokud je prediktor $Y$ skutečně binární, tedy značí $0$ pro nepřítomnost jevu a $1$ pro přítomnost jevu,
lze šance vypočítat jako

\begin{equation}
\text{šance } = e^{\beta_i}    
\end{equation}

Šance je podíl dvou pravděpodobností. Pokud bychom měli šanci jevu A oproti jevu B $2 : 1$, značí to, že výskyt jevu A je dvakrát tak pravděpodobný
jako výskyt jevu B a jev A se vyskytuje ve $2 \over 3$ případů. Šance $e^{\beta_i}$ tedy značí vztah mezi prediktorem $X_i$ a predikovanou proměnnou $Y$. Pokud je
šance kladná, značí to, že s vyšší hodnotou prediktoru $X_i$ se zvyšuje šance že $P(Y = 1)$. Pokud je naopak nižší, pravděpodobnost se zmenšuje. Pokud je potřeba
interpretovat pravděpodobnost jako šanci, použije se logitová funkce

\begin{equation}
    \label{eq:logitova_funkce}
    \text{šance jevu A} = \frac{p}{1 - p}
\end{equation}

kde $p$ je pravděpodobnost výskytu jevu A.

\subsection{Maximální pravděpodobnost}
Parametry logistického modelu v rovnici \ref{eq:pravdepodobnost_y} jsou pouze teoretické a je třeba je určitým způsobem odhadnout. Již vypočtené odhady
se proto neznačí pouze $\beta$, ale $\hat{\beta}$. Pro odhad parametrů se při logistické regresi používá metoda zvaná Největší pravděpodobnost. Pro výpočet
největší pravděpodobnosti se počítá pravděpodobnostní funkce $L(\theta)$ kde $\theta$ jsou parametry logistického modelu $\alpha, \beta_1, ..., \beta_k$.
Pro logistickou regresi má pravděpodobnostní funkce tvar

\begin{equation}
    \label{eq:pravdepodobnostni_fce}
    L(\theta) = \Pi_{l = 1}^{m_1} P(X_i) \Pi_{l = m_1 + 1}^{n} 1 - P(X_i) 
\end{equation}

kde $n$ je počet pozorování a $m_1$ je počet příznivých ($Y = 1$) jevů. Funkce předpokládá, že datový soubor je seřazen tak, že prvních $m_1$ výskytů
jsou jevy příznivé. $P(X_i)$ poté značí logistickou funkci \ref{eq:logisticka_funkce}. Pro vypočtení optimálního parametru $\beta_i$ je nutné vypočítat
maximum funkce $L(\theta)$ vzhledem k parametru $\beta_i$. Parametr $\beta_i$ lze tedy získat derivací funkce $L(\theta)$ vzhledem k parametru $\beta_i$

\begin{equation}
    \frac{\partial L(\theta)}{\partial \beta_i} = 0
\end{equation}

\subsection{Waldův test}
Waldův test je test hypotézy o tom, zda je parametr logistického regresního modelu $\beta_i$ významný či nikoliv. Definice
testu hypotézy je tedy

$H_0:$ Koeficient $\beta_i$ je rovný nule \\
$H_A:$ Koeficient $\beta_i$ je různý od nuly

Kritická hodnota $Z$ má normální rozdělení $Z ~ N(0, 1)$ a její mocnina, $Z^2$, má Chi-square rozdělení s jedním stupněm volnosti.
$Z$ a lze vypočítat jako

\begin{equation}
    Z = \frac{\hat{\beta_i}}{S_{\hat{\beta_i}}}
\end{equation}

kde $\hat{\beta_i}$ je odhadnutý testovaný parametr logistického regresního modelu a $S_{\hat{\beta_i}}$ je směrodatná chyba koeficientu $\beta_i$

\subsection{Test poměru věrohodností}
Test poměru věrohodnosti porovnává slouží k porovnání dvou modelů. Důležitá podmínka je, aby jeden model byl zmenšený a druhý model plný.
To v praxi znamená, že zmenšený model obsahuje stejné prediktory jako model plný, ale ne všechny. Prediktory, které v zmenšeném modelu
nejsou jsou následně obsahem testovací hypotézy. Pokud se model plný a zmenšený liší o prediktor $X_i$, jsou hypotézy definovány následovně

$H_0:$ Koeficient $\beta_i$ je rovný nule \\
$H_A:$ Koeficient $\beta_i$ je různý od nuly

Do výpočtu kritické hodnoty pak v poměru vstupují hodnoty pravděpodobnostní funkce \ref{eq:pravdepodobnostni_fce}, kde 
$\hat{L_1}$ značí hodnotu pravděpodobnostní funkce pro plný model a $\hat{L_1}$ hodnotu pro model zmenšený.

\begin{equation}
    Z = -2ln(\frac{\hat{L_1}}{\hat{L_2}})
\end{equation}

při velké hodnotě $n$ má $Z$ zhruba Chi-square rozdělení s jedním stupněm volnosti.

\subsection{Vyhodnocení modelu}
