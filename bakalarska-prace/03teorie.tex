\chapter{Teoretická část}
V následující části jsou popsány jak teoretické metody pro vizualizaci dat, tak i tvar, forma a vyhodnocení logistického regresního modelu. 
Ke každé části, která se věnuje popisu dat pomocí nějakého grafu, je přidána praktická ukázka s popisem a praktickým vysvětlením. U regresního 
modelu je následně dán důraz na teorii, jelikož model je následně prakticky použit v celé třetí části. Důležité je si uvědomit, že nástrojů pro
vizualizaci dat je mnoho a grafů či souhrnných statistik je nespočet. V této práci se ovšem věnujeme vizualizacím, které jsou pro regresní model
vhodné.
Testovací citace: \cite{Hebak2015}, \cite{Kleinbaum2010}

\section{Vizualizace dat}
\subsection{Bodový graf}
Bodový graf slouží pro zobrazení vztahu dvou numerických proměnných. Z pravidla se vysvětlovaná proměnná dává na osu Y,
zatímco proměnná vysvětlující se nachází na ose X. Graf \ref{fig:bodovy_graf_mtcars} zobrazuje negativní vztah mezi
váhou vozidla a mílemi ujetými za galon.

\begin{figure}[H]
    \centering
    \includegraphics{../../analyza/plots/bodovy_graf_mtcars.png}
    \caption{Bodový graf váhy a míly za galon z data setu mtcars} 
    \label{fig:bodovy_graf_mtcars}
\end{figure}

\subsection{Histogram}

Histogram je speciální typ sloupcového grafu. Je specifický tím, že zobrazuje počet jednotlivých pozorování ve skupině. Skupiny je nutné vytvořit
v případě, že je zobrazovaná proměnná spojitá. Zároveň histogram oproti klasickému sloupcovému grafu, který zobrazuje kategorie, nemá mezi sloupci žádné mezery.
Na grafu \ref{fig:sloupcovy_graf_mtcars} je příklad sloupcového grafu. Na ose X se nachází kategorická vysvětlovací proměnná. Vysvětlovaná proměnná Y je průměrná hrubá
koňská síla.

\begin{figure}[H]
    \centering
    \includegraphics{../../analyza/plots/sloupcovy_graf_mtcars.png}
    \caption{Sloupcový graf z data setu mtcars} 
    \label{fig:sloupcovy_graf_mtcars}
\end{figure}

Pro histogram je třeba data sloučit do skupin \textit{(bins)} o určité šířce. Správný výběr počtu skupin je kritický, jelikož může velmi
silně ovlivnit interpretaci dat. Pokud se vybere moc malý počet skupin, data se seskupí a může se ztratit důležitý vztah. Pokud se ovšem
vybere moc velký počet skupin, v datech bude obtížné najít nějaký obecný vztah či trend.
Tento efekt je znázorněn na grafu \ref{fig:histogram_porovnani_mtcars}.

\begin{figure}[H]
    \centering
    \includegraphics{../../analyza/plots/histogram_porovnani_mtcars.png}
    \caption{Porovnání histogramů s různým počtem skupin} 
    \label{fig:histogram_porovnani_mtcars}
\end{figure}

Pro vhodný počet skupin existuje mnoho způsobů. Nejznámější je takzvané Sturgesovo pravidlo (\ref{eq:sturgesovo_pravidlo}), které se spočítá následujícím vztahem:

\begin{equation}
    \label{eq:sturgesovo_pravidlo}
    k \text{ } \dot{\mathbf{=}} \text{ } 1 + 3,22 * log_{10}(n)
\end{equation}

kde $k$ je výsledný zaokrouhlený počet skupin a $n$ je počet pozorování. Druhý parametr, který je pro tvorbu histogramu potřeba, je šířka skupiny.
Ta by měla být ideálně stejná pro všechny skupiny. Pokud tomu tak není, histogram může být zavádějící a čtenář mu nemusí plně rozumět.
Pro vypočtení počtu skupin má šířka skupiny následující tvar:

\begin{equation}
    \label{eq:sirka_histogramu}
    w = \ceil[\bigg]{ \frac{max(x) - min(x)}{k} }
\end{equation}

kde $x$ je zobrazovaná proměnná, $k$ je počet skupin a $w$ je výsledná šířka intervalu, která je zaokrouhlena nahoru. To je z toho důvodu, aby se
každé pozorování promítlo právě do jedné kategorie. Pokud na stejný dataset, jako na grafu \ref{fig:histogram_mtcars_sturges}, použije sturgesovo
(rovnice \ref{eq:sturgesovo_pravidlo}) pravidlo, graf vypadá následovně:

\begin{figure}[H]
    \centering
    \includegraphics{../../analyza/plots/histogram_mtcars_sturges.png}
    \caption{Histogram s počtem skupin dle Sturgesova pravidla} 
    \label{fig:histogram_mtcars_sturges}
\end{figure}

Histogram lze samozřejmě rozepsat. Lze vytvořit tzn. tabulku četností (\ref{tab:tabulka_cetnosti_sturges}),
která mimo jiné obsahuje spodní hranici intervalu, horní hranici intervalu a počet pozorování. Důležité je,
aby přechody mezi intervaly byli jasné.

\begin{table}[H]

        \centering
    \begin{tabular}[t]{c|c|c}
        \hline
        Spodní hranice intervalu & Horní hranice intervalu & Počet pozorování\\
        \hline
        7.05 & 11.75 & 2\\
        \hline
        11.76 & 16.45 & 9\\
        \hline
        16.46 & 21.15 & 9\\
        \hline
        21.16 & 25.85 & 6\\
        \hline
        25.86 & 30.55 & 4\\
        \hline
        30.56 & 35.25 & 2\\
        \hline
    \end{tabular}
    \caption{\label{tab:tabulka_cetnosti_sturges}Tabulka četnostní}
\end{table}

\subsection{Boxplot}
\subsubsection{Five-number summary}
Five-number summary je číselná tabulka, která pomocí pěti různých čísel shrnuje seřazenou číselnou řadu. Základní statistický nástroj pro
vytvoření takové tabulky jsou percentily. Hodnota $P$-tého percentilu označuje číslo, které rozděluje seřazenou číselnou řadu na dva intervaly. 
První interval obsahuje $P*100\%$ číselné řady a druhý analogicky $(1-P)*100\%$. Různé hodnoty percentilů mohou mít specifičtější pojmenování a značí se $Q_P$.
Percentil $P = 0.5$ se označuje jako medián a rozděluje seřazenou číselnou řadu na polovinu. Percentily, kde $P = 0.25$ nebo $P = 0.75$, se označují
jako kvartily a značí se $Q_{1}$ a $Q_{3}$. Oba tyto typy kvartilů jsou použité při tvorbě Five-number summary tabulky. Jako příklad je
uvedena tabulka \ref{tab:five-number_summary}

\begin{table}[H]

    \centering
    \begin{tabular}[t]{c|c|c|c|c}
        \hline
        $Q_{0} (Q_0)$ & $Q_{0.25} (Q_1) $ & $Q_{0.50}$ & $Q_{0.75} (Q_3)$ & $Q_{1.00}$\\
        \hline
        1.513 & 2.58125 & 3.325 & 3.61 & 5.424\\
        \hline
    \end{tabular}
    \caption{\label{tab:five-number_summary}Five-number summary tabulka hmotnosti vozidla (lb/1000)}
\end{table}

$Q_{0}$ a $Q_{1.00}$ označují minimum a maximum číselné řady. Kvartily $Q_{1}$ a $Q_{3}$ jsou čísla, která rozděluji časovou řadu na na čtvrtiny. V prvním
případě, tedy $Q_1 = Q_{0.25}$, je 25\% čísel menší než 1.513 a 75\% dat větší. Pro kvantil $Q_3 = Q_{0.75}$ je 75\% čísel menších než 3.61 a 25\% větších. $Q_{0.50}$ označuje medián.

\subsubsection{Boxplot}
Boxplot je grafické zobrazení a rozšíření Five-number summary tabulky. Krom grafické zobrazení pěti percentilů obvykle ukazuje statistiky jako průměr a extrémní hodnoty.
V boxplotu se také nachází obdélník, který ukazuje interkvartilní rozsah \textit{(IQR)}, tedy prostředních 50\% dat. V obdélníku se také nachází černá čára, která značí medián.
Z prostředního obdélníku vedou oběma směry čáry, které značí teoretické minimum a maximum (ne hodnoty $Q_0$ a $Q_1$). Obě dve hranice se počítají následovně

\begin{align}
    min &= Q_1 - 1.5 * IQR \\
    max &= Q_3 + 1.5 * IQR
\end{align}

Hodnoty, které spadají do intervalu $<Q_1 - 1.5 * IQR ; Q_1>$ a $<Q_3, Q_3 + 1.5 * IQR>$ se nazývají jako odlehlé. Hodnoty které leží mimo tento vztah,
tedy hodnoty menší než $Q_1 - 1.5 * IQR$ či větší než $Q_3 + 1.5 * IQR$, mají speciální název a označují se jako hodnoty extrémní.
V boxplotu jsou z pravidla znázorněni nějakým speciálním znakem, např. kolečkem.
Díky grafickém zobrazení lze lehce porovnávat distribuci jedné vysvětlované proměnné vůči několika kategoriím.

\begin{figure}[H]
    \centering
    \includegraphics{../../analyza/plots/boxplot_mtcars.png}
    \caption{Boxplot váhy auta pro různé počty válců} 
    \label{fig:boxplot_mtcars}
\end{figure}

Trojúhelník v grafu \ref{fig:boxplot_mtcars} značí průměr pro danou kategorii. Černé tečky v kategorii osmi válců značí extrémní hodnoty, t. j. hodnoty
větší než $<Q_3, Q_3 + 1.5 * IQR>$ jsou extrémní.

\section{Logistický regresní model}
... Dopsat

\subsection{Jednoduchý logistický regresní model}
... Dopsat

\subsection{Vícerozměrný logistický regresní model}
... Dopsat

\subsection{Optimalizace parametrů}
... Dopsat

\subsubsection{Metoda nejmenších čtverců}
... Dopsat

\subsubsection{Metoda největší pravděpodobnosti}
... Dopsat

\subsection{Vyhodnocení modelu}
... Dopsat
