\chapter{Praktická část}
V následující části je popis dat, transformace dat, a tvorba logistického regresního modelu. Prve jsou představené datové soubory,
se kterými se pracuje. Následně jsou požité grafy, které jsou představené v sekci \ref{sec:vizualizace_dat}. Následně jsou vytvořené
logistické regresní modely, jejich výstup je interpretován a různé modelu jsou mezi sebou porovnány. V závěru se nachází každé podsekce
se nachází zamyšlení, jak by daný model mohl být vylepšen a jaké je jeho použití v reálném životě.

\section{Datové soubory}
Dataset\footnote{https://www.kaggle.com/datasets/mateusdmachado/csgo-professional-matches} obsahuje čtyři souboru, které popisují zápasy ve hře
\ac{CSGO}. K potřebám této bakalářské práce budou použity pouze soubory \textit{players.csv} a \textit{results.csv}. Soubor \textit{economy.csv} obsahuje
informace o vývoji ekonomiky v daném zápase. Tato informace by byla velmi užitečná v případě, že se snažíme předpovídat výsledek zápasu, který aktuálně probíhá.
Tato informace není dostupná před začátkem zápasu a tudíž je tento dataset pro tuto bakalářskou práci nepoužitelný. Druhý soubor, \textit{picks.csv}, obsahuje
postup výběr map mezi dvěma týmy v daném zápase. Tato práce je omezená na zápasy, které jsou typu Bo1 nebo Bo3. Bo1 znamená, že první tým co vyhraje mapu vyhrál i zápas.
Bo3 značí, že se hrají celkem tři mapy Kdo první vyhraje dvě mapy, vyhrál celý zápas. Tento postup by se dal využít v případě simulace zápasu mezi dvěma týmy. Dalo by 
se předpovědět, vzhledem k historickým postupům, jaká mapa má jakou procentní šanci být vybrána či zabanována\footnote{tým ji v zápasu zakáže a nemůže si ji vybrat druhý tým}.
Jelikož je práce zaměřená pouze na předpověď zápasu s již známou mapou, není příležitost dataset využít.

\subsection{soubor players.csv}
Soubory \textit{players.csv} obsahuje statistiky jednotlivých hráčů v daném zápase. Původní dataset obsahuje 101 sloupců a 379 680 záznamů.
Názvy všech sloupců je možné vidět v přiložené tabulce \ref{tab:players.csv_puvodni}. Pro potřeby logistické regrese je nutné datový soubor transformovat do stavu,
kdy se jeden řádek rovná statistikám jednoho hráče na jedné mapě. Transformovaný dataset má 10 sloupců a 643 620 řádků. Příklad záznamu v transformovaném
datasetu je v přiložené tabulce \ref{tab:players_csv_transformovano}.

\newpage
Transformovaný dataset má 10 sloupců, které unikátně identifikují statistiky každého hráče na určité mapě v jednom zápase. Interpretace je následující:
\begin{itemize}
    \item \textbf{match\_id} - identifikátor zápasu
    \item \textbf{player\_id} - identifikátor hráče
    \item \textbf{team} - jméno týmu
    \item \textbf{map} - název hrané mapy
    \item \textbf{kills} - počet zabití hráče v zápase na dané mapě
    \item \textbf{assists} - počet asistencí hráče v zápase na dané mapě
    \item \textbf{deaths} - počet smrtí hráče v zápase na dané mapě
    \item \textbf{hs} - procento zabití, které lze označit jako headshot\footnote{hráč zabil nepřítele střelou do hlavy}
    \item \textbf{fkdiff} - rozdíl, kolikrát hráč zabil jako první nepřítele versus kolikrát byl jako první zabit
    \item \textbf{rating} - shrnutí mnoha statistik za zápas, díky kterým lze hráč ohodnotit\footnote{\url{https://www.hltv.org/news/20695/introducing-rating-20}}
\end{itemize}

\subsection{soubor results.csv}
Druhý datový soubor, který je pro analýzu použit, obsahuje výsledky daných zápasů. Dataset se původně skládá z 45 773 řádků a 19 sloupců. Dataset obsahuje na rozdíl
od datového souboru \textit{players.csv} chybné záznamy, které značí, že tým hrál sám proti sobě. Také jsou zde uvedené názvy týmu jako \uv{?}, které nelze interpretovat.
Tyto záznamy jsou proto odstraněny. Po transformacích vznikne tabulka o 8 sloupcích a 91 436 řádcích. Každý záznam identifikuje výsledek jednoho týmu v jednom zápasu
na jedné mapě. Příklad je zobrazen v přiložené tabulce \ref{tab:results_csv_transformovano}. Jednotlivé sloupce lze interpretovat následovně:
\begin{itemize}
    \item \textbf{date} - datum, kdy se hrál zápas
    \item \textbf{match\_id} - identifikátor zápasu
    \item \textbf{team} - jméno týmu
    \item \textbf{map} - název hrané mapy
    \item \textbf{map\_winner} - binární značení, zda tým vyhrál (1) či prohrál (0)
    \item \textbf{starting\_ct} - binární značení, zda tým začal zápas na straně Counter-Teroristů (1) či Teroristů (0)
    \item \textbf{team\_rank} - rank týmu v okamžik, kdy se zápas hrál\footnote{\url{https://www.hltv.org/news/16061/introducing-csgo-team-ranking}}
    \item \textbf{run\_mean\_3\_months} - klouzavý průměr týmu za poslední tři měsíce
\end{itemize}

\subsection{Omezení datasetu}
Dataset obsahuje záznamy o zápasech a statistikách od konce roku 2015 do začátku roku 2020. Jelikož máme mnoho záznamu, není problém se takovýchto
záznamu zbavit a odstranit je. Dále jsou ze souboru smazány polo vyplněné statistiky, kde máme např. statistiku kills, ale ne statistiku deaths. Toto je z důvodu
vývoje stránky\footnote{\url{https://www.hltv.org/}}, která data sbírá, a vývoje hry samotné. Ne vždy je možné data získat ať už kvůli jiným verzím či retrospektivní
kompatibilitě.

Jak již bylo zmíněno, z datového souboru \textit{results.csv} jsou odstraněné záznamy, kde tým hrál sám proti sobě, nebo je název týmu \uv{?}. Z datového souboru  
\textit{players.csv} jsou pak odstraněné záznamy o zápasech, kde nemáme údaje o všech deseti hráčích. Občas se může stát, že tým hrál pouze ve čtyřech, např. z důvodu
nedochvilnosti. Při finálním spojení obou souborů se pak může stát, že jsou specifické záznamy pouze v jednom datovém souboru. Může se tedy stát, že existují
záznamy o statistikách pro zápas, pro který nemáme finální výsledek. Opačně také může nastat situace, kdy existuje záznam o výsledku zápasu, ale nejsou záznamy o statistikách,
nebo jsou statistiky nekompletní. V obou těchto případech se záznamů zbavíme, jelikož i po kombinaci existuje velké množství záznamů.

\newpage
\section{Explorační analýza dat}
Explorační analýza dat je část, kdy se datasety vizualizují, hledají se různé vztahy, závislosti a zajímavé interpretace. K exploraci datasetu
jsou použité nástroje představené v sekci \ref{sec:vizualizace_dat}. 

\subsection{Počet zápasů přes kategorie map}
První zajímavý údaj může být, kolikrát se daná mapa hrála. Z výsledného grafu lze zjistit, zda je nějaká mapa výrazně preferovanější než jiná, nebo se mapy
hrají v přibližně stejném poměru.

\begin{figure}[H]
    \centering
    \includegraphics{../obrazky/zapasy_pres_mapu.png}
    \caption{Počet zápasů na každé mapě} 
    \label{fig:zapasy_pres_mapu}
\end{figure}

Z obrázku \ref{fig:zapasy_pres_mapu} lze odvodit, že mapy se ve stejném poměru nehrají. Nejhranější mapa je mapa s názvem Mirage. Ta je v map
poolu\footnote{množina map, které se hrají} nejdéle bez žádné velké aktualizace vzhledu mapy, struktury mapy či rozložení. Naopak nejméně hraná mapa
je Vertigo. Ta byla do map poolu přidán relativně nedávno\footnote{\url{https://www.hltv.org/news/26418/vertigo-added-to-active-duty-map-pool-in-new-update}} a 
mezi stálice se teprv dostává. Za svou existenci prošla mnoha aktualizacemi, které se implementují ze zpětné vazby hráčů. Jelikož je mapa nová, hráči stále objevují 
nové způsoby a strategie, jak mapu hrát. Druhý důvod proč je mapa méně preferovaná je částečně spojený s tím faktem, že je mapa nová. Pro nezkušené týmy je na mapě
lehké porazit týmy, které jsou lepší. To je z toho důvodu, že na klasických mapách (jako např. mapa Mirage či Inferno), mají zkušené týmy velikou výhodu - hráli mapu již
mnohokrát. Na nové mapě tato výhoda mizí a díky tomu mají nezkušené týmy určitou výhodu překvapení.

\subsection{Histogram zabití}
Statistika zabití na úrovni hráčů značí, kolik zabili v daném kole nepřátel. Obecně se dá říct, že čím víc hráč zabije nepřátel, tím více pomohl svému týmu.
Samozřejmě že existuje mnoho faktorů, které malý počet zabitých nepřátel vysvětli. Může to být například role. Pokud je hráč velitel týmu nebo podpora týmu a má malý
počet zabitých nepřátel, neznamená to nutně, že zhoršují šanci týmu vyhrát.

\begin{figure}[H]
    \centering
    \includegraphics{../obrazky/histogram_zabiti.png}
    \caption{Počet zápasů na každé mapě} 
    \label{fig:histogram_zabiti}
\end{figure}

Obrázek histogramu počtu zabití naznačuje, že počet zabití v zápase má normální rozdělení s malým sklonem doprava. Průměrný počet zabití za zápas a mediánový
počet zabití za zápas se tedy pohybuje kolem hodnoty 17 zabití za zápas.

\subsection{Boxplot ratingu přes kategorie map}
Pro statistiku rating se lze podívat na její distribuci přes různé kategorie mapy. Pokud je medián ratingu na nějaké mapě výrazně vyšší, lze usoudit,
že je mapa strukturovaná pro individuální hráče. Takový typ mapy bude odměňovat spíše individuální výkon než souhru týmu a strategický plán. Obrázek níže
zobrazuje boxploty hráčů přes mapy \textbf{pouze v případě, že zápas vyhráli.}

\begin{figure}[H]
    \centering
    \includegraphics{../obrazky/boxplot_ratingu.png}
    \caption{Počet zápasů na každé mapě} 
    \label{fig:boxplot_ratingu_pres_mapu}
\end{figure}

Z obrázku boxplotů \ref{fig:boxplot_ratingu_pres_mapu} lze usoudit, že se mezikvartilové rozpětí a medián ratingu je na všech mapách podobný. Obrázek tedy naznačuje,
že na žádné mapě není výrazná výhoda pro individuální rating hráče. Lze udělat závěr, že na výhru v zápase mají jiné statistiky větší vliv.

\subsection{Korelace mezi statistikami}
Pro logistickou regresi je důležité, aby prediktory nebyli korelované. Toto lze zjistit z korelační matice. Ta vyobrazí korelaci mezi individuálními prediktory.

\begin{figure}[H]
    \centering
    \includegraphics{../obrazky/prediktory_korelace.png}
    \caption{Korelace} 
    \label{fig:korelacni_matice}
\end{figure}

Korelační matice \ref{fig:korelacni_matice} ukazuje, že korelace mezi prediktorem rating a prediktory kills, fkdiff a deaths existuje relativně vysoká korelace.
Jelikož je rating hodnocení hráče v daném zápase, dává smysl, že je pozitivně korelované s počtem zabití, statistikou fkdiff a záporně korelované s počtem úmrtí.
Z tohoto důvodu je statistika z modelů odebrána.

\section{Sestavení modelu}
... Slovní popis modelu a jeho proměnných, rovnice

\subsection{Interpretace}
... Matematická interpretace modelu a výstupu z Rka (koeficienty, odchylky, významnost...)

\section{Vyhodnocení modelu}
\subsection{Statistika 1}
... Různé statistiky, které ještě musím vybrat
