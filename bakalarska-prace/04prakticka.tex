\chapter{Praktická část}

{\color{red}
V této části se nachází popis dat, transformace dat, a tvorba logistického regresního modelu. Nejprve jsou představené datové soubory,
se kterými se pracuje. Následně jsou použité grafy, které jsou představené v sekci \ref{sec:vizualizace_dat}. Pomocí vizualizace dat lze představit proměnné, které
by mohli do logistického modelu vstupovat. Také lze díky grafům získat intuici, jak datový soubor vypadá a jaké výsledky je možné očekávat.
Poté jsou vytvořené logistické regresní modely, jejich výstup je interpretován a různé modelu jsou mezi sebou porovnány. V závěru každé podsekce
se nachází zamyšlení, jak by daný model mohl být vylepšen a jaké je jeho použití v reálném životě.
}

{\color{red}
\section{Datové soubory}
Dataset\footnote{https://www.kaggle.com/datasets/mateusdmachado/csgo-professional-matches} obsahuje čtyři souboru, které popisují zápasy ve hře
\ac{CSGO}. K potřebám této bakalářské práce budou použity pouze soubory \textit{players.csv} a \textit{results.csv}. Ostatní dva soubory obsahují
informace, které nezle k predikci v případě této bakalářské práce použít.
}

{\color{red}
\subsection{soubor players.csv}
Soubory \textit{players.csv} obsahuje statistiky jednotlivých hráčů v daném zápase. Původní dataset obsahuje 101 proměnných a 379 680 pozorování.
V původním datovém souboru se jeden řádek (záznam) rovná statistikám jednoho hráče za celý zápas. Pro potřeby bakalářské práce je však nutné získat statistiky hráčů
za jednotlivé mapy (jeden zápas může být hrán až na třech mapách). Proto je původní datový soubor transformován do podoby, kde se jeden záznam rovná statistikám
právě jednoho hráče na právě jedné mapě, a to bez ohledu, kolik map se v daném zápase hrálo. Jinak řečeno, transformovaný datový soubor nebere v potaz, zda
se daná mapa hrála jako první, druhá, či třetí.
Transformovaný dataset má 10 proměnných a 643 620 záznamů. Příklad jednotlivých pozorování v transformovaném
datasetu je v přiložené tabulce \ref{tab:players_csv_transformovano}.
}
\newpage
Transformovaný dataset má 10 proměnných, které unikátně identifikují statistiky každého hráče na určité mapě v jednom zápase. Interpretace je následující:
\begin{itemize}
    \item \textbf{match\_id} --- identifikátor zápasu
    \item \textbf{player\_id} --- identifikátor hráče
    \item \textbf{team} --- jméno týmu
    \item \textbf{map} --- název hrané mapy
    \item \textbf{kills} --- počet zabití hráče v zápase na dané mapě
    \item \textbf{assists} --- počet asistencí hráče v zápase na dané mapě
    \item \textbf{deaths} --- počet smrtí hráče v zápase na dané mapě
    \item \textbf{hs} --- procento zabití, které lze označit jako headshot\footnote{hráč zabil nepřítele střelou do hlavy}
    \item \textbf{fkdiff} --- rozdíl, kolikrát hráč zabil jako první nepřítele versus kolikrát byl jako první zabit
    \item \textbf{rating} --- shrnutí více statistik za jeden zápas do jednoho ukazatele výkonu\footnote{\url{https://www.hltv.org/news/20695/introducing-rating-20}}
\end{itemize}

{\color{red}
\subsection{soubor results.csv}
Druhý datový soubor, který je pro analýzu použit, obsahuje výsledky daných zápasů. Dataset se původně skládá z 45 773 řádků a 19 proměnných. Dataset obsahuje na rozdíl
od datového souboru \textit{players.csv} jeden chybný záznam. Dle něho hrál tým sám proti sobě, což nedává smysl. Jelikož je zápas na webovém portálu zadán
správně, nejspíše se jedná o neznámou chybu, která nastala při exportu dat z webového portálu.
}
Po transformacích vznikne tabulka o 8 proměnných a 91 436 řádcích. Každé pozorování identifikuje výsledek jednoho týmu v jednom zápasu
na jedné mapě. Příklad je zobrazen v přiložené tabulce \ref{tab:results_csv_transformovano}. Jednotlivé proměnné lze interpretovat následovně:
\begin{itemize}
    \item \textbf{date} --- datum, kdy se hrál zápas
    \item \textbf{match\_id} --- identifikátor zápasu
    \item \textbf{team} --- jméno týmu
    \item \textbf{map} --- název hrané mapy
    \item \textbf{map\_winner} --- binární značení, zda tým vyhrál (1) či prohrál (0)
    \item \textbf{starting\_ct} --- binární značení, zda tým začal zápas na straně Counter-Teroristů (1) či Teroristů (0)
    \item \textbf{team\_rank} --- rank týmu v okamžik, kdy se zápas hrál\footnote{\url{https://www.hltv.org/news/16061/introducing-csgo-team-ranking}}
    \item \textbf{run\_mean\_3\_months} --- klouzavý průměr týmu za poslední tři měsíce
\end{itemize}

{\color{red}
\subsection{Omezení datasetu}
Dataset obsahuje pozorování o zápasech a statistikách od konce roku 2015 do začátku roku 2020. Díky velkému počtu záznamů je možné chybné či neúplné záznamy
smazat. To může nastat např. když tým má méně než 5 hráčů nebo když zápas nemá v každém týmu právě 5 hráčů. Tým může mít méně než 5 hráčů z toho důvodu, že je např.
amatérsky\footnote{neprofesionální, tím pádem nemusí mít všichni hráči na webovém portálu založená profil}. Pokud má hráčů více, vyberou se hráči dle délky svého působení v týmu.
Tým může mít v zápasu více než pět hráčů, pokud použijí náhradníka. Také jsou smazané záznamy, které neobsahují všechny potřebné statistiky. To může nastat u velmi historických
zápasů, kde výpočet daných statistik ještě nebyl možný.
}

\newpage
\section{Cíl analýzy}
Cílem analýzy je vytvořit logistické regresní modely, které předpovídají pravděpodobnost výhry daného hráče či týmu. Pro předpověď se použijí statistiky hráčů
za každou mapu a výsledné týmové statistiky za každou mapu. Statistiky hráčů, které jsou označovaný za prediktory jsou: \textit{kills}, \textit{assists}, \textit{deaths},
\textit{hs}, \textit{fkdiff} a \textit{rating}. Za týmové statistiky jsou pak považovány proměnné \textit{starting\_ct} a \textit{run\_mean\_3\_months}. Lze očekávat, že
prediktory \textit{kills}, \textit{assists}, \textit{hs}, \textit{rating} a \textit{fkdiff} budou šanci na výhru zvyšovat. Naopak prediktor \textit{deaths} by měl
výslednou šanci na výhru snížit. 

{\color{green}}
Nehodila by se tato část spíše někde v úvodu bakalářské práce?
}
První typ logistického modelu bude predikovat výhru zápasu na určité mapě podle individuálních statistik hráče a týmových prediktorů. Nebere se v potaz
výkon spoluhráčů ani průměrný rank týmu za poslední tři měsíce. Cílem modelu je identifikovat, která statistika hráčů má na určité mapě největší vliv na výhru.

Druhý model se bude soustředit na predikci výhry týmu. K tom využije agregované statistiky hráčů, kteří jsou součástí týmu. Může se stát, že za tým historicky hrálo více hráčů.
V případě, že tým vymění jednoho hráče za druhého, bude za tým historicky hrát 6 hráčů. V takovém případě model počítá s hráči, kteří za tým hrají nejdelší dobu. V případě schody
se vezme hráč, který hrál v týmu jako poslední. Agregace statistik je spočítaná jako aritmetický průměr přes všechny zápasy, kde daná pětice za tým hraje. Dále jsou v modelu
prediktory \textit{starting\_ct} a \textit{run\_mean\_3\_months}. Cíl druhého modelu je zjistit, jaké prediktory jsou pro předpověď výhry týmu významné. Model je vytvořen
jak pro celý datový soubor, tak pro jeden vybraný tým.

Poslední třetí logistický model je vytvořen pro jeden specifický tým, který bude předpovídat pravděpodobnost výhry proti historicky nejlepším třiceti týmům. Cílem tohoto modelu
je zjistit, jak se liší významnost prediktorů, pokud tým hraje pouze proti těm nejlépe hodnocením týmům. 

\section{Průzkumová analýza dat}
Průzkumová analýza slouží k vizualizaci prediktorů, hledání různých vztahů a rozdělení proměnných. Díky průzkumu lze určit, které proměnné není vhodné použít pro tvorbu
logistického regresního modelu, např. kvůli problému multikolinearity.

\newpage
\subsection{Korelační matice}
Pro logistickou regresi je důležité, aby prediktory nebyli lineárně závislé. Kombinaci korelací mezi kvantitativními prediktory lze zjistit z korelační matice.

\begin{figure}[H]
    \centering
    \includegraphics{../obrazky/prediktory_corr_matice.png}
    \caption{Korelační matice} 
    \label{fig:korelacni_matice}
\end{figure}

Z korelační matice \ref{fig:korelacni_matice} lze vyčíst, korelace se blíží nule mezi průměrem týmu za poslední tři měsíce a agregovanými statistikami hráčů. Zároveň je
vidět středně silná korelace mezi prediktorem \textit{rating} a statistikami \textit{fkdiff}, \textit{deaths} a \textit{kills}. Z tohoto důvodu je
prediktor \textit{rating} z tvorby modelů vyloučen.

\newpage
\subsection{Histogram kvantitativních prediktorů}
Histogram kvantitativních prediktorů umožní zobrazit rozdělení prediktorů.

\begin{figure}[H]
    \centering
    \includegraphics{../obrazky/histogram_prediktoru.png}
    \caption{Histogram prediktorů} 
    \label{fig:histogram_prediktoru}
\end{figure}

Histogram prediktorů \textit{rating}, \textit{hs}, \textit{kills} a \textit{deaths} má normální rozdělení a v proměnné se nenachází mnoho extrémních hodnot.
Prediktor \textit{fkdiff} má bimodální\footnote{rozdělení má dva vrcholy} rozdělení. Prediktor \textit{assists} je sklonění doprava, což značí velké množství 
odlehlých či extrémních hodnot.

\newpage
\subsection{Sloupcový graf výher přes počáteční stranu}
Strana, na které tým začíná, může ovlivnit výsledek zápasu. Ovlivnění navíc může být rozdílné podle toho, na jaké mapě se zápas odehrává.

\begin{figure}[H]
    \centering
    \includegraphics{../obrazky/sloupce_podle_strany.png}
    \caption{Procento vyhraných zápasů na dané mapě za stranu Counter-Terroristů} 
    \label{fig:sloupcovy_graf_strany}
\end{figure}

Počáteční strana má největší vliv na mapě Cobblestone, kde necelých 52 procent týmů, co začalo na straně Counter-Terroristů, mapu vyhrálo. Sloupcový graf naopak naznačuje, že 
je pro týmy nevýhodné začínat na mapě Overpass za stranu Counter-Terroristů. Historicky necelých čtyřicet-devět procent týmu, co na mapě začalo za Counter-Terroristy, mapu prohrálo.

