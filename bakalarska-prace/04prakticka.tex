\chapter{Praktická část}
{\color{red}
Cílem praktické části je prozkoumat dva typy modelů. První typ modelu bude predikovat výhru zápasu pro jednotlivé hráče přes kategorie map. Hlavním cílem 
modelu bude identifikovat významné charakteristiky hráčů na vybraných mapách a následně rozdíly interpretovat. V druhém modelu bude zkoumaná výhra týmu v zápase.
V tomto typu modelu se jako prediktory použijí agregované charakteristiky hráčů jednotlivých týmu. Pro agregaci bude použit buď aritmetický nebo geometrický průměr. 

V této části se dále nachází popis dat a transformace dat. Nejprve jsou představené datové soubory,
se kterými se pracuje. Následně jsou použité grafy, které jsou představené v sekci \ref{sec:vizualizace_dat}. Pomocí vizualizace dat lze představit proměnné, které
do logistického modelu budou vstupovat. Také lze díky grafům získat povědomí, jak datový soubor vypadá a jaké mají proměnné rozdělení.
Poté jsou vytvořené logistické regresní modely, jejich výstup je interpretován a různé modely jsou mezi sebou porovnány.

Každý model je trénován na 80\% dat. Pro interpretaci významnosti prediktorů je použit Waldův test, který je součástí každé tabulky
ve sloupci $Pr(>|z|)$. Alfa pro určení hladiny významnosti je stanovena jako $\alpha = 0.05$. Pro matici záměn je model otestován na validačních datech, 
které činí 20\% původního datového souboru.
}

{\color{red}
\section{Cíle analýzy}
Cílem analýzy je vytvořit logistické modely pro předpověď výhry jak pro individuální hráče, tak pro tým. Jelikož se charakteristiky hráčů i zápasů mohou měnit dle mapy,
na které se zápas odehrává, modely pro individuální hráče jsou tříděné přes kategorie map. Modely pro předpověď výhry týmu používají interakci mezi mapou a začínající
stranou.

Pro hráče budou sestaveny logistické modely z celého datového souboru přes kategorie map. Modely budou následně porovnány pomocí matice záměn
a bude porovnána významnost jejich parametrů. Modely budou porovnané na vybraných dvou mapách, a to Mirage a Vertigo. Mirage je klasický mapa, která je
ve hře od jeho vydání. Vertigo je naopak nejnovější přírůstek do profesionální scény a hráči, v době extrakce datového souboru, mapu ještě plně strategicky 
neznali. Modely budou porovnánu podle přesnosti predikce a podle různé významnosti prediktorů.

Modely pro týmy budou sestavené prve z celého datového souboru a následně pro dva vybrané referenční týmy. První tým, pro který se vytvoří model, bude tým Astralis.
Ten je dlouhodobě považován za jeden z nejlepších týmu na světě. Druhým týmem bude německý celek Sprout. Tým Sprout se řadí v době
extrakce dat k profesionálnímu týmu s průměrným třicátým místem na světovém žebříčku. Modely jsou vytvořené pouze pro dva týmy z toho důvodu, že je celkový
 počet týmů velmi vysoký a bylo by komplikované 
porovnávat všechny týmy naráz. Z toho důvodu se budou hledat významné rozdíly mezi dvěma referenčními modely a jedním celkovým. Mezi modely bude porovnaná jak přesnost
predikce, tak významnost parametrů. Jelikož je původní spojený datový soubor na úrovni hráčů, charakteristiky hráčů se musí agregovat na úroveň týmů a zápasů. Pro tuto
agregaci bude použit aritmetický nebo geometrický průměr.
}

\section{Datové soubory}
Dataset\footnote{https://www.kaggle.com/datasets/mateusdmachado/csgo-professional-matches} obsahuje čtyři 
{\color{red}
datové soubory,
}
které popisují zápasy ve hře
\ac{CSGO}. K potřebám této bakalářské práce budou použity pouze soubory \textit{players.csv} a \textit{results.csv}. Datové soubory jsou následně spojeny do jedné tabulky,
která obsahuje charakteristiky všech hráčů v právě jednom zápase, potřebné informace o zápase a výsledek (zda hráč zápas vyhrál či nikoliv). Data jsou následně dle potřeby
sjednocena pro týmy v zápase. Charakteristiky se agregují pomocí aritmetického nebo geometrického průměru. Zbylé dva soubory obsahují
informace, které jsou již z probíhajících zápasů a z volby map. Tyto informace pro predikci výhry ještě před začátkem zápasu nelze využít. Žádný z těchto
zbylých dvou souborů (\textit{picks.csv}, \textit{economy.csv}) proto v bakalářské práci není použit.

\subsection{Soubor players.csv}
{\color{red}
Datový soubor
}
\textit{players.csv} obsahuje 
{\color{red}
charakteristiky
}
jednotlivých hráčů v daném zápase. Původní datový soubor obsahuje 101 proměnných a 383 317 pozorování.
V původním datovém souboru se jeden řádek (pozorování) rovná charakteristikám jednoho hráče za celý zápas
{\color{red}
, který se může odehrávat až na třech mapách.
}
Pro potřeby bakalářské práce je tak nutné získat charakteristiky hráčů
za jednotlivé mapy. Proto je původní datový soubor transformován do podoby, kde se jedno pozorování rovná charakteristikám
právě jednoho hráče na právě jedné mapě, a to bez ohledu na to, kolik map se v daném zápase hrálo. Jinak řečeno, transformovaný datový soubor nebere v potaz, zda
se daná mapa hrála jako první, druhá, či třetí.
Transformovaný datový soubor má 10 proměnných a 634 650 pozorování. Příklad jednotlivých pozorování v transformovaném
datovém souboru je v přiložené tabulce \ref{tab:players_csv_transformovano}.

\newpage
Transformovaný 
{\color{red}
datový soubor
}
má 10 proměnných. Interpretace je následující:
\begin{itemize}
    \item \textbf{match\_id} --- identifikátor zápasu
    \item \textbf{player\_id} --- identifikátor hráče
    \item \textbf{team} --- jméno týmu
    \item \textbf{map} --- název hrané mapy
    \item \textbf{kills} --- počet zabití hráče v zápase na dané mapě
    \item \textbf{assists} --- počet asistencí hráče v zápase na dané mapě
    \item \textbf{deaths} --- počet smrtí hráče v zápase na dané mapě
    \item \textbf{hs} --- procento zabití, které lze označit jako headshot\footnote{hráč zabil nepřítele střelou do hlavy}
    \item \textbf{fkdiff} --- rozdíl, kolikrát hráč zabil jako první nepřítele versus kolikrát byl zabit jako první
    \item \textbf{rating} --- shrnutí více charakteristik za jeden zápas do jednoho ukazatele výkonu\footnote{\url{https://www.hltv.org/news/20695/introducing-rating-20}}
\end{itemize}

\subsection{Soubor results.csv}
Druhý datový soubor, který je pro analýzu použit, obsahuje výsledky daných zápasů. Soubor se původně skládá z 45 772 záznamů a 19 proměnných. Datový soubor 
\textit{results.csv} obsahuje na rozdíl od datového souboru \textit{players.csv} jedno chybné pozorování. Dle něho hrál tým sám proti sobě, což nedává smysl.
Jelikož je zápas na webovém portálu zadán správně, nejspíše se jedná o neznámou chybu, která nastala při exportu dat z webového portálu.

Po transformacích vznikne tabulka o 7 proměnných a 91 502 záznamech. Každé pozorování identifikuje výsledek jednoho týmu v jednom zápase
na jedné mapě. Příklad je zobrazen v přiložené tabulce \ref{tab:results_csv_transformovano}. Jednotlivé proměnné lze interpretovat následovně:
\begin{itemize}
    \item \textbf{date} --- datum, kdy se hrál zápas
    \item \textbf{match\_id} --- identifikátor zápasu
    \item \textbf{team} --- jméno týmu
    \item \textbf{map} --- název hrané mapy
    \item \textbf{map\_winner} --- binární značení, zda tým vyhrál (1) či prohrál (0)
    \item \textbf{starting\_ct} --- binární značení, zda tým začal zápas na straně Counter-Teroristů (1) či Teroristů (0)
    \item \textbf{team\_rank} --- rank týmu v okamžik, kdy se zápas hrál\footnote{\url{https://www.hltv.org/news/16061/introducing-csgo-team-ranking}}
\end{itemize}

\subsection{Omezení datového souboru}
{\color{red}
Všechny datové soubory obsahují pozorování o zápasech a charakteristikách hráčů od konce roku 2015 do začátku roku 2020. Jelikož pro finální modely je nutné datové soubory
\textit{players.csv} a \textit{results.csv} sjednotit, může se stát, že se vytvoří zápas bez hráčů. Může také nastat situace, kdy zápas nebude mít přiřazených právě 
deset různých hráčů. Tým může mít méně než 5 hráčů z toho důvodu, že je např.
amatérsky\footnote{neprofesionální, tím pádem nemusí mít všichni hráči na webovém portálu založená profil}. Více hráčů může hrát za tým v případě, že tým použil náhradníka.
Zároveň je možné, že kvůli historickým změnám ve hře a na webovém portálu nebude možné získat všechny potřebné charakteristiky hráčů. V případě všech zmíněných chyb jsou
chybné záznamy odstraněny.
}

\section{Průzkumová analýza dat}
Průzkumová analýza vizualizuje prediktory, hledá různé vztahy a rozdělení proměnných. Díky průzkumu lze určit, které proměnné není vhodné použít pro tvorbu
logistického regresního modelu, např. kvůli problému multikolinearity.

\newpage
\subsection{Korelační matice}
Pro logistickou regresi je důležité, aby prediktory nebyly lineárně závislé. Přehled korelací mezi kvantitativními prediktory lze zjistit z korelační matice.

\begin{figure}[H]
    \centering
    \includegraphics{../obrazky/prediktory_corr_matice.png}
    \caption{Korelační matice} 
    \label{fig:korelacni_matice}
\end{figure}

Z korelační matice \ref{fig:korelacni_matice} lze vyčíst, 
{\color{red}
že korelace mezi rankem týmu a charakteristikami hráčů se blíží nule. Z toho plyne, že neexistuje lineární závislost mezi výkonem hráče a 
rankem týmu.
}
Zároveň je vidět silná korelace mezi prediktorem \textit{rating} a prediktory \textit{fkdiff}, \textit{deaths} a \textit{kills}.
{\color{red}
Jelikož by díky vysoké korelaci prediktorů vznikl problém multikolinearity, prediktor \textit{rating} ve finálních modelech není použit.
}

\newpage
\subsection{Histogramy kvantitativních prediktorů}
Histogramy kvantitativních prediktorů umožní zobrazit jejich rozdělení.

\begin{figure}[H]
    \centering
    \includegraphics{../obrazky/histogram_prediktoru.png}
    \caption{Histogram prediktorů} 
    \label{fig:histogram_prediktoru}
\end{figure}

{\color{red}
Histogram prediktorů z obrázku \ref{fig:histogram_prediktoru} ukazuje, že přediktory \textit{rating}, \textit{hs}, \textit{kills} a \textit{deaths} mají normální rozdělení
a v prediktorech se nenachází mnoho extrémních hodnot.
}
Prediktor \textit{fkdiff} má bimodální rozdělení. Prediktor \textit{assists} je 
{\color{red}
zešikmení
}
doprava, což značí velké množství odlehlých či extrémních hodnot. 
{\color{red}
Pro logistickou regresi není předpoklad normálního rozdělení prediktorů Analýza proto slouží k získání povědomí o tom, 
jakých hodnot každý prediktor nabývá a jaké je jejich rozdělení.
}

\newpage
{\color{red}
\subsection{Sloupcový graf výher přes počáteční stranu}
Každá mapa funguje a vypadá jinak, což ovlivňuje mimo jiné i možné strategie. Proto jiné mapy mohou více vyhovovat jiným týmům. Toto způsobuje, že
tým může mít vyšší procento výhry na mapě, pokud začíná na straně Counter-Terroristů.

\begin{figure}[H]
    \centering
    \includegraphics{../obrazky/sloupce_podle_strany.png}
    \caption{Procento vyhraných zápasů na dané mapě za stranu Counter-Terroristů} 
    \label{fig:sloupcovy_graf_strany}
\end{figure}

Z obrázku \ref{fig:sloupcovy_graf_strany} je patrné, že týmu Astralis nevyhovuje začínat mapu Cobblestone na straně Counter-Terroristů. Tým Sprout má největší procento
vyhraných zápasů pri počáteční straně jako Counter-Terroristé na mapě Vertigo. Nejmenší pak na mapě Mirage. Při pohledu na sloupcový graf pro všechny týmy lze vyčíst,
že celkově počáteční strana nemá vliv. Díky obrázku \ref{fig:sloupcovy_graf_strany} je vidět, že šanci na výhru ovlivňuje interakce mezi mapou a začínající stranou. 
}

\section{Predikce výhry hráče}
{\color{red}
Cílem modelu je predikovat výhru zápasu pro jednotlivé hráče a identifikovat významné prediktory na odlišných mapách.
}
Prediktory se týkají pouze výkonu jednotlivých hráčů, model tedy pro předpověď výhry hráče nepoužívá charakteristiky spoluhráčů. Pro porovnání jsou vybrané mapy Mirage a Vertigo.
Mapa Mirage je jednou z nejvíce tradičních map a mapa Vertigo je naopak nejnovější přídavek do hry. Díky rozdílným modelům bude možné zkoumat, na čem pravděpodobnost výhry na 
mapách záleží.

\subsection{Model pro mapu Mirage}

\input{kod/modely/player_model_Mirage.tex}

{\color{red}
Z tabulky \ref{tab:player_model_Mirage} je vidět, že pro model není významný prediktor \textit{hs}. Prediktor je proto odebrán a model je znovu natrénován na stejných datech.
}

\input{kod/modely/player_model_Mirage_opt.tex}

{\color{red}
Tabulka \ref{tab:player_model_Mirage_opt} představuje model se všemi významnými prediktory. Prediktory \textit{kills}, \textit{assists} a \textit{fkdiff} šanci na výhru
hráče zvyšují. S každým zabitím hráč zvyšuje šanci na výhru zhruba 1,2 krát. Naopak prediktory \textit{deaths} a \textit{starting\_ct} šanci snižují. S každou hráčovou smrtí
se šance na výhru snižuje zhruba 0,69 krát. Pokud hráči začnou mapu na straně Counter-Terroristů, jejich šance na výhru se sníží zhruba 0,82 krát. 
Model lze zapsat také pomocí logistické funkce.
}

\begin{align}
    \begin{split}
        &P(1 | X_{kills}, X_{assists}, X_{deaths}, X_{fkdiff}, X_{starting_{ct}}) = \frac{1}{1 + e^{-z}} \\
        &z = 2,399 + 0,180*X_{kills} + 0,305*X_{assists} - 0,371*X_{deaths} + \\
        &+ 0,017*X_{fkdiff} - 0,203*X_{starting_{ct}}
    \end{split}
    \label{eq:player_funkce_Mirage}
\end{align}

\subsubsection{Matice záměn pro mapu Mirage}
Predikce jsou provedené na validačním podmnožině a na optimalizovaném modelu z tabulky \ref{tab:player_model_Mirage_opt}.

\input{kod/matice/player_matice_Mirage_opt.tex}

Model predikoval správně 7 007 výher 
{\color{red}
($\sim 81,4\%$)
}
a 6 689 proher
{\color{red}
($\sim 77,2\%$). Celkově model určil správně 13 696 objektů ($\sim 79,3\%$).
}

\input{kod/matice_out/player_stats_Mirage_opt.tex}

{\color{red}
Díky vyšší specificitě modelu o zhruba $4\%$ je model vhodnější na predikci prohry hráče.
}

\subsection{Model pro mapu Vertigo}

\input{kod/modely/player_model_Vertigo.tex}

Pro hráče jsou na mapě Vertigo významné pouze prediktory \textit{kills}, \textit{assists} a \textit{deaths}. Ostatní prediktory \textit{hs}, \textit{fkdiff} a \textit{starting\_ct}
jsou pro model nevýznamné. Nevýznamnost lze interpretovat tak, že pro hráče není důležité, na jaké straně mapu začnou hrát (prediktor  \textit{starting\_ct}), jak přesně
střílí (prediktor \textit{hs})
{\color{red}
, ani jak týmově hraje na začátku kola (prediktor \textit{fkdiff}).
}
Po vyřazení nevýznamných prediktorů má model následující významné parametry:

\input{kod/modely/player_model_Vertigo_opt.tex}

{\color{red}
Jak lze očekávat, prediktory \textit{kills} a \textit{assists} šanci na výhru zvyšují, zatím co prediktor \textit{deaths} šanci na výhru sníží. S každým zabitím hráče
na mapě Vertigo se jeho šance na výhru zvýší zhruba 1,2 krát a s každou smrtí se jeho šance na výhru sníží zhruba 0,70 krát.
}
Přepis funkce je pak následující:

\begin{align}
    \begin{split}
        &P(1 | X_{kills}, X_{assists}, X_{deaths}) = \frac{1}{1 + e^{-z}} \\
        &z = 1,971 + 0,183*X_{kills} + 0,308*X_{assists} - 0,357*X_{deaths}.
    \end{split}
\end{align}

\subsubsection{Matice záměn pro mapu Vertigo}
Matice záměn pro 
{\color{red}
optimální
}
logistický model sestavený pro mapu Vertigo
{\color{red}
vypadá následovně:
}

\input{kod/matice/player_matice_Vertigo_opt.tex}

Model úspěšné predikoval 430 výher
{\color{red}
($\sim 76,6\%$)
}
a 441 proher 
{\color{red}
($\sim 77,8\%$). Celkově model úspěšně predikoval 851 objektů ($\sim 77,2\%$)
}

\input{kod/matice_out/player_stats_Vertigo_opt.tex}

Z tabulky statistik \ref{tab:player_stats_Vertigo_opt} je vidět nepatrně lepší výkon při predikci výher (statistika Senzitivita) o necelé jedno procento.
Model se proto hodí spíše na predikci výher.

\subsection{Interpretace výsledků}
Model pro mapu Mirage je spolehlivější pro předpověď prohry hráče. Model pro mapu Vertigo má menší přesnost o zhruba 
2 procentní body a hodí se spíše k predikci výher hráčů.
{\color{red}
Přehled výsledku přes ostatní mapy lze nalézt v příloze \ref{chap:ap02}
}

Pro logistický model mapy Mirage není významný prediktor \textit{hs}. Největší vliv na výhru hráče na mapě Mirage má počet jeho smrtí, kdy s každou další smrtí se šance
na výhru zmenšuje zhruba 0,69 krát. Pro mapu Vertigo jsou významné pouze prediktory \textit{kills}, \textit{assists} a \textit{deaths} a největší vliv na výhru hráče
{\color{red}
má také počet smrtí, kde s každou hráčovou smrtí šance klesne zhruba 0,7 krát.
}

Rozdíl
{\color{red}
ve významnosti prediktorů
}
by mohl být vysvětlen právě stářím jednotlivých map. Jelikož je mapa Mirage už tradiční mapou, mnoho hráčů na ní zná různé triky a strategie. To vede k tomu, že
každý malý detail hraje velikou roli a většina vybraných prediktorů je pro model významná. Naopak mapa Vertigo je relativně nová a pro hráče je nejdůležitější individuální
výkon. Jelikož hráči mapu tak dobře strategicky neznají, jsou významné pouze charakteristiky hráčů.

\newpage
\section{Predikce výhry týmu}
Cílem modelů je predikovat výhru na základě agregovaných charakteristik hráčů za tým na mapě. Charakteristiky jsou agregované buď pomocí aritmetického průměru,
nebo pomocí průměru geometrického. Ostatní charakteristiky zápasu jako počáteční strana či výherní tým nejsou nijak změněny. V modelu pro predikci týmu vystupuje
navíc prediktor \textit{team\_rank}, který ukazuje rank daného týmu na mapě. Dále je v modelu zakomponovaná interakce
mezi počáteční stranou (\textit{starting\_ct}) a hranou mapou (\textit{map}). Příklad tabulky s agregovanými charakteristikami je v příloze \ref{tab:data_agregovana}.
Pro trénink a validaci je rozdělený původní datový soubor v poměru 8:2.

\subsection{Celkový model}

\input{kod/modely/team_model_All.tex}

První vytvořený model je sestavený na celém trénovacím datovém souboru. Pro model jsou významné všechny prediktory bez interakce, jmenovitě \textit{mean\_kills},
\textit{mean\_assists}, \textit{mean\_deaths}, \textit{mean\_hs},\textit{mean\_fkdiff} a \textit{team\_rank}. Interakce mezi prediktory \textit{map} a
\textit{starting\_ct} není významná u map Vertigo, Cache a Nuke.

Agregované charakteristiky hráče \textit{mean\_kills} a \textit{mean\_assists} šanci na výhru týmu zvyšují. Pokud se průměr zabitých nepřátel za tým zvýší o jednotku,
šance na výhru týmu se zvýší zhruba 3,96 krát. Pokud se průměr smrtí hráčů za tým zvýší o jednotku, šance na výhru se sníží zhruba 0,25 krát. Všechny statisticky
významné interakce mezi prediktory \textit{map} a \textit{starting\_ct} naznačují, že je pro tým nevýhodné začínat na straně Counter-Terroristů. Jejich šance
na výhru se vždy sníží, a to nejvíce na mapě Overpass, kde se šance sníží zhruba 0,62 krát. Zajímavý je koeficient u prediktoru \textit{team\_rank}, který říká,
že s růstem ranku týmu se šance na výhru sníží zhruba 0,99 krát. To lze vysvětlit tím, že lepší týmy hrají proti lepším týmům a jejich 
šance na výhru je nižší. Model lze zapsat jako přepis funkce následovně:

\begin{align}
    \begin{split}
        P(1 | &X_{mean\_kills}, X_{mean\_assists}, X_{mean\_deaths}, X_{mean\_hs}, X_{mean\_fkdiff}, X_{team\_rank}, \\
              &X_{mapCache*starting\_ct}, X_{mapCobblestone*starting\_ct}, X_{mapDust2*starting\_ct}, X_{mapInferno*starting\_ct}, \\
              &X_{mapMirage*starting\_ct}, X_{mapNuke*starting\_ct}, X_{mapOverpass*starting\_ct}, X_{mapTrain*starting\_ct}, \\
              &X_{mapVertigo*starting\_ct}) = \frac{1}{1 + e^{-z}} \\
        z = &0,322 + 1,376*X_{mean\_kills} + 0,143*X_{mean\_assists} - 1,402*X_{mean\_deaths} - \\
            &- 0,615*X_{mean\_hs} - 0,064*X_{mean\_fkdiff} - 0,001*X_{team\_rank} - \\
            &- 0,040*X_{mapCache*starting\_ct} - 0,282*X_{mapCobblestone*starting\_ct} - \\
            &- 0,318*X_{mapDust2*starting\_ct} - 0,272*X_{mapInferno*starting\_ct} - \\
            &- 0,239*X_{mapMirage*starting\_ct} - 0,113*X_{mapNuke*starting\_ct} - \\
            &- 0,418*X_{mapOverpass*starting\_ct} - 0,177*X_{mapTrain*starting\_ct} + \\
            &+ 0,259*X_{mapVertigo*starting\_ct}
    \end{split}
\end{align}

\subsubsection{Matice záměn pro obecný model}
Model je vyhodnocen na validační podmnožině, která činní 20\% původního datového souboru.

\input{kod/matice/team_matice_All.tex}

Model úspěšně predikoval 8 274 výher ($\sim 94,1\%$) a 8 203 proher ($\sim 94,0\%$). Celkem model predikoval správně 16 477 objektů ($\sim 94,1\%$).

\input{kod/matice_out/team_stats_All.tex}

Všechny výkonnostní statistiky z matice záměn jsou identické v řádu setin procent. Z tabulky \ref{tab:team_stats_All} nelze jednoznačně určit, zda je tým
vhodnější na predikci výher či proher.

\subsection{Model pro tým Astralis}
Tým Astralis je v době extrakce dat jedním z nejlepších týmu na světě. Vyhrál několik prestižních majorů a i v roce 2022 se tým Astralis považuje za nejlepší tým
ve hře \ac{CSGO} všech dob.

\input{kod/modely/team_model_Astralis.tex}

Významné prediktory pro tým Astralis jsou pouze \textit{mean\_kills} a \textit{mean\_deaths}.  Pro tým není statistický významné, jaký je průměrný počet asistencí
(\textit{mean\_assists}), jaké je průměrné procento zabití do hlav (\textit{mean\_hs}), jaký je průměrný výkon hráčů na začátku mapy (\textit{mean\_fkdiff}) ani 
rank týmu v daném zápase (\textit{team\_rank}). Tým Astralis navíc neovlivňuje počáteční strana, mají tedy stejnou šanci na výhru bez ohledu na 
začínající stranu.

\input{kod/modely/team_model_Astralis_opt.tex}

Tabulka \ref{tab:team_model_Astralis_opt} popisuje již optimální parametry pro model týmu Astralis. S každým dalším průměrným zabitím se zvýší šance
týmu Astralis na výhru zhruba 3,85 krát. S každou další průměrnou smrtí se šance na výhru týmu sníží zhruba 0,23 krát. 
Před začátkem zápasu, tedy při nulových prediktorech \textit{mean\_kills} a \textit{mean\_deaths}, je šance na výhru týmu Astralis zhruba 9,75 větší, než jeho
prohra. Optimální model lze zapsat jako přepis rovnice.

\begin{align}
    \begin{split}
        P(1 | &X_{mean\_kills},  X_{mean\_deaths}) = \frac{1}{1 + e^{-z}} \\
        z = &2,277 + 1,349*X_{mean\_kills} - 1,457*X_{mean\_deaths}
    \end{split}
\end{align}

\subsubsection{Matice záměn pro tým Astralis}

\input{kod/matice/team_matice_Astralis_opt.tex}

Optimální model správně predikoval 54 výher ($\sim 93,1\%$) a 108 proher ($\sim 93,9\%$). Celkem optimální model predikoval správně 162 objektů ($\sim 93,6\%$).

\input{kod/matice_out/team_stats_Astralis_opt.tex}

\subsection{Model pro tým Sprout}
Tým Sprout byl v době extrakce dat čistě Německý tým a patřil k průměrným profesionálnímu týmům. Na žebříčků týmu se obvykle řadil kolem třicátého místa.

\input{kod/modely/team_model_Sprout.tex}

Z tabulky modelu \ref{tab:team_model_Sprout} je vidět, že významné prediktory pro tým jsou pouze \textit{mean\_kills} a \textit{mean\_deaths}. Pro tým není významné,
na jaké straně začíná mapu (interakce \textit{map*starting\_ct}), jaký je jeho rank (\textit{team\_rank}), jak přesně průměrné střílí hráči týmu (\textit{mean\_hs}),
jak jsou průměrné hráči dobří na začátku kola (\textit{mean\_fkdiff}) ani kolik mají průměrně asistencí (\textit{mean\_assists}).

\input{kod/modely/team_model_Sprout_opt.tex}

S každým dalším průměrným zabitím (\textit{mean\_kills}) se šance na výhru týmu zvýší zhruba 4,95 krát. S každou další průměrnou smrtí (\textit{mean\_deaths})
se šance na výhru sníží zhruba 0,23 krát. Ještě před začátkem zápasu je šance na prohru týmu zhruba 0.12 krát větší, než na jeho výhru.

\begin{align}
    \begin{split}
        P(1 | &X_{mean\_kills},  X_{mean\_deaths}) = \frac{1}{1 + e^{-z}} \\
        z = &-2,098 + 1,600*X_{mean\_kills} - 1,485*X_{mean\_deaths}
    \end{split}
\end{align}

\subsubsection{Matice záměn pro tým Sprout}

\input{kod/matice/team_matice_Sprout_opt.tex}

Optimální model korektně předpověděl 54 výher ($\sim 93,1\%$) a 62 proher ($\sim 96,9\%$). Model úspěšně předpověděl celkem 116 objektů ($\sim 95,1\%$)

\input{kod/matice_out/team_stats_Sprout_opt.tex}

Jelikož je senzitivita modelu o zhruba 4 procentní body větší, model se více hodí na předpověď výhry týmu Sprout.

\subsection{Interpretace výsledků}
Pro celkový model jsou kromě interakcí $mapCache:starting\_ct$, $mapNuke:starting\_ct$ a $mapVertigo:starting\_ct$ významné všechny prediktory. Největší vliv na výhru má
prediktor \textit{mean\_deaths}, která šanci snižuje zhruba 0,25 krát. U všech významných kombinací mezi prediktory \textit{map} a \textit{starting\_ct} je koeficient
záporný. To naznačuje, že tým má menší šanci na výhru, pokud mapu začne na straně Counter-Terroristů. Největší negativní vliv je u mapy Overpass, kde se šance
sníží až 0,65 krát.

Tým Astralis má před začátkem zápasu, tedy při nulových prediktorech, šanci na výhru zhruba 9,75 větší, než na prohru. Vzhledem k tomu, že tým byl dlouhodobě považován
za jeden z nejlepších týmu na světě a v roce 2022 je mnoha hráči považován za nejlepší tým všech dob, je výsledek očekávaný. Tým Sprout má před začátkem zápasu 
šanci na prohru zhruba 0,25 krát větší, než na výhru. Tým je průměrně umístěn na třicátém místě, díky čemuž hraje převážně proti nejlepším padesáti týmům na světě.

Oba dva referenční modely mají stejné významné prediktory, a to \textit{mean\_kills} a \textit{mean\_deaths}. Pro žádný tým není významné, na jaké mapě začíná a jeho šanci
na výhru to nijak neovlivňuje. Z modelů lze usoudit, že nejlepší tým na světě a průměrný profesionální tým má stejné významné prediktory.Rozdíl mezi referenčními modely a 
celkovým modelem by bylo možné vysvětlit např. vlivem neprofesionálních týmů. Porovnávání neprofesionálních či polo-profesionálních týmu je složitější, jelikož týmy 
hrají méně zápasů, a není zaručeno, že mají dostatek zápasu na každé mapě. Datový soubor by bylo nutné tím pádem velmi omezit.