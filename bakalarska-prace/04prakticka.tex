\chapter{Praktická část}
Cílem praktické části je prozkoumat dva typy modelů. První typ modelu bude predikovat výhru zápasu pro jednotlivé hráče a druhý typ pro jednotlivé týmy. Hlavním cílem 
modelů pro hráče bude identifikovat významné charakteristiky hráčů na vybraných mapách a následně interpretovat rozdíly. V druhém modelu bude zkoumána výhra týmu v zápase.
V tomto typu modelu se jako prediktory použijí agregované charakteristiky hráčů jednotlivých týmu. Pro agregaci bude použit buď aritmetický, nebo geometrický průměr. Všechny
modely jsou vyhodnoceny pomocí testovacího datového souboru a matice záměn. Pro tvorbu modelů a veškerou práci s daty je použit programovací jazyk R \cite{team_r_2022}
a vývojové prostředí RStudio \cite{rstudio_rstudio_2022}. Zdrojový kód celé analýzy se nachází v příloze \ref{apx:03}.

\section{Cíle analýzy}
Cílem analýzy je vytvoření logistických modelů předpovídajících výhry jak pro individuální hráče, tak pro dva referenční týmy a pro  všechny týmy. Jelikož se charakteristiky hráčů i 
týmů mohou měnit dle mapy, na které se zápas odehrává, modely pro individuální hráče jsou tříděné podle kategorie map. Modely pro předpověď výhry týmu používají interakci
mezi mapou a začínající
stranou.

Pro hráče budou sestaveny logistické modely z celého datového souboru, a to zvlášť pro každou mapu. Modely budou následně vyhodnoceny pomocí statistik vycházejících
z matice záměn, dále bude porovnána významnost jejich parametrů. Modely budou porovnány na dvou vybraných mapách, a to na mapě Mirage a Vertigo. Mirage je klasickou mapou, která
se ve hře objevuje už od jejího vydání. Vertigo je naopak nejnovějším přírůstkem do profesionální scény, a tak hráči v době extrakce datového souboru mapu ještě plně strategicky 
neznali. Modely budou porovnány podle přesnosti predikce a podle toho, jaké prediktory jsou pro model významné.

Modely pro týmy budou sestavené pro dva vybrané referenční týmy a pro všechny týmy. První tým, pro který se vytvoří model, bude tým Astralis.
Ten je dlouhodobě považován za jeden z nejlepších týmu na světě. Druhým týmem bude německý celek Sprout. Profesionální tým Sprout se v době extrakce dat řadí na průměrné
třicáté místo na světovém žebříčku. Modely jsou vytvořené pouze pro dva týmy z toho důvodu, že je celkový počet týmů velmi vysoký a bylo by příliš komplikované porovnávat
všechny týmy najednou. Proto se budou hledat významné rozdíly mezi dvěma referenčními modely a jedním celkovým modelem, který je sestaven na celém datovém souboru. Mezi modely
bude porovnaná jak přesnost predikce, tak významnost prediktorů. 

\section{Příprava dat}
Dataset\footnote{https://www.kaggle.com/datasets/mateusdmachado/csgo-professional-matches} obsahuje čtyři datové soubory, které popisují zápasy ve hře
\ac{CSGO}. K potřebám této bakalářské práce budou použity pouze soubory \textit{players.csv} a \textit{results.csv}. Datové soubory jsou následně spojeny do jednoho datového
souboru, který obsahuje charakteristiky všech hráčů v právě jednom zápase, potřebné informace o zápase a výsledek (zda hráč zápas vyhrál, či nikoliv). Zbylé dva soubory obsahují
informace, které pocházejí z již probíhajících zápasů a z volby map. Tyto informace pro predikci výhry ještě před začátkem zápasu nelze využít. Žádný z těchto
zbylých dvou souborů (\textit{picks.csv}, \textit{economy.csv}) proto v bakalářské práci použit není.

\subsection{Soubor players.csv}
Datový soubor \textit{players.csv} obsahuje charakteristiky jednotlivých hráčů v daném zápase. Původní datový soubor obsahuje 101 proměnných a 383 317 pozorování.
V původním datovém souboru se jeden řádek (pozorování) rovná charakteristikám jednoho hráče za celý zápas, který se může odehrávat až na třech mapách.
Pro potřeby bakalářské práce je tak nutné získat charakteristiky hráčů za jednotlivé mapy. Proto je původní datový soubor transformován do podoby, kde se
jedno pozorování rovná charakteristikám právě jednoho hráče na právě jedné mapě, a to bez ohledu na to, kolik map se v daném zápase hrálo. Jinak řečeno, transformovaný
datový soubor nebere v potaz, zda se daná mapa hrála jako první, druhá či třetí.

Z transformovaného datového souboru jsou odstraněny záznamy o mapě Default. Ta značí automatickou výhru pro jeden tým, například díky formátu turnaje. V souboru je také
možné narazit na tým, za který za zápas na jedné mapě nehrálo právě 5 hráčů. Více než pět hráčů mohlo na mapě hrát v případě použitého náhradníka. 
Hráčů může hrát i méně, pokud se některý například nestihne dostavit. V obou případech jsou data pro danou mapu v zápase odstraněná.

Transformovaný a očištěný datový soubor obsahuje 10 proměnných a 640 225 pozorování. Příklad jednotlivých pozorování v transformovaném datovém souboru se nachází v
přiložené tabulce \ref{tab:players_csv_transformovano}. Interpretace charakteristik je následující:

\begin{itemize}
    \item \textbf{match\_id} --- identifikátor zápasu,
    \item \textbf{player\_id} --- identifikátor hráče,
    \item \textbf{team} --- jméno týmu,
    \item \textbf{map} --- název hrané mapy,
    \item \textbf{kills} --- počet zabití hráče v zápase na dané mapě,
    \item \textbf{assists} --- počet asistencí hráče v zápase na dané mapě,
    \item \textbf{deaths} --- počet smrtí hráče v zápase na dané mapě,
    \item \textbf{hs} --- procento zabití, které lze označit jako headshot\footnote{hráč zabil nepřítele střelou do hlavy},
    \item \textbf{fkdiff} --- rozdíl mezi tím, kolikrát hráč zabil nepřítele jako první a kolikrát byl zabit jako první,
    \item \textbf{rating} --- shrnutí více charakteristik za jeden zápas do jednoho ukazatele, výkonu.\footnote{\url{https://www.hltv.org/news/20695/introducing-rating-20}}
\end{itemize}

\subsection{Soubor results.csv}
Druhý datový soubor, který je k analýze použit, obsahuje výsledky daných zápasů. Soubor se původně skládá z 19 proměnných a 45 773 pozorování. Ten je transformován do podoby,
kde se každý řádek rovná právě jednomu výsledku právě jednoho týmu na právě jedné mapě. Nově vytvořený datový soubor obsahuje 7 proměnných a 91 502 pozorování.  
Transformovaný datový soubor obsahuje jedno chybné pozorování. Dle něho hrál tým sám proti sobě, což nedává logický smysl. Jelikož je zápas na webovém portálu zadán správně,
nejspíše se jedná o neznámou chybu, která nastala při exportu dat z webového portálu. Dále je v původním souboru rozdělen tým Astralis do dvou týmů: týmu Astralis a tým
\uv{?}. Důvod rozdělení je ten, že historicky hrál tým bez organizace a označoval se jako \uv{?}. Po vytvoření organizace Astralis jsou hráči řazeni pod tuto organizaci. Jelikož
se tímto týmem zabývá specializovaný logistický model, je označení týmu v transformovaném souboru sjednoceno pod název Astralis. Příklad transformovaného datového souboru
je zobrazen v přiložené tabulce \ref{tab:results_csv_transformovano}. Jednotlivé proměnné lze interpretovat následovně:

\begin{itemize}
    \item \textbf{date} --- datum, kdy proběhl zápas,
    \item \textbf{match\_id} --- identifikátor zápasu,
    \item \textbf{team} --- jméno týmu,
    \item \textbf{map} --- název hrané mapy,
    \item \textbf{map\_winner} --- binární značení, zda tým vyhrál (1), či prohrál (0),
    \item \textbf{starting\_ct} --- binární značení, zda tým začal zápas na straně counter-teroristů (1), nebo teroristů (0),
    \item \textbf{team\_rank} --- rank týmu v okamžik, kdy se zápas hrál.\footnote{\url{https://www.hltv.org/news/16061/introducing-csgo-team-ranking}}
\end{itemize}

\subsection{Datový soubor pro modelování} \label{sec:spojeny_datovy_soubor}
Pro modelování je nutné vytvořit jeden datový soubor, na kterém bude model sestaven. Datový model je získán spojením představených a transformovaných souborů
\textit{players.csv} a \textit{results.csv}. Datové soubory jsou spojené pomocí proměnných \textit{match\_id}, \textit{team} a \textit{map}. Při propojování
souborů byly smazány záznamy, které nebyly reprezentovány v obou souborech. To znamená, že charakteristiky hráčů v zápase, jehož výsledek soubor  \textit{results.csv} neobsahoval,
byly smazány. Dále byly smazány zápasy, pro které naopak nebyly zadané charakteristiky hráčů. Spojený datový soubor je pak na úrovni jednotlivých hráčů a ukazuje jejich
charakteristiky v právě jednom zápase na právě jedné mapě. Příklad spojeného datového souboru je v přiložené tabulce \ref{tab:data_spojena}.

\subsection{Agregovaný datový soubor} \label{sec:agregovany_datovy_soubor}
Spojený datový soubor obsahuje charakteristiky hráčů na úrovni jednotlivých zápasů. To je vhodné pro první logistický model, který se zabývá predikcí a identifikací
významných prediktorů pro individuální hráče. Pro predikci výhry týmu a identifikaci významných prediktorů pro tým je nutné data agregovat. K agregaci charakteristik, u kterých
to dává smysl, je použit aritmetický průměr. Pro charakteristiky, u kterých dává smysl násobení, je použit průměr geometrický. Jako příklad lze uvést charakteristiku 
počtu zabití (prediktor \textit{kills}), jež je agregovaná pro daný tým jako průměrný počet zabití (prediktor \textit{mean\_kills}) hráčů týmu na dané mapě. Pro výpočet
charakteristiky procent zabití do hlavy (prediktor \textit{hs}) je použit geometrický průměr. V modelu je použitý průměr procenta zabití do hlavy (prediktor \textit{mean\_hs}).
Příklad agregovaných dat se nachází v přiložené tabulce \ref{tab:data_agregovana}. Popis agregací je uveden v následující tabulce.

\begin{table}[H]
    \centering
    \caption{Přehled agregací charakteristik pro daný tým na dané mapě v daném zápase}
    \begin{tabular}{|l|l|l|}
    \hline
    \multicolumn{1}{|c|}{prediktor} & \multicolumn{1}{c|}{agregace} & \multicolumn{1}{c|}{popis}                        \\ \hline
    mean\_kills                     & aritmetický průměr            & průměrný počet zabití                             \\ \hline
    mean\_assists                   & aritmetický průměr            & průměrný počet asistencí                          \\ \hline
    mean\_deaths                    & aritmetický průměr            & průměrný počet smrtí                              \\ \hline
    mean\_hs                        & geometrický průměr            & průměrné procento zabití do hlavy                 \\ \hline
    mean\_fkdiff                    & aritmetický průměr            & průměrný rozdíl mezi prvním zabitím a první smrtí \\ \hline
    \end{tabular}
\end{table}

\subsection{Trénování a validace modelů}
Pro křížovou validaci modelů se vytvoří náhodné rozdělení dat v poměru 8:2. K trénování modelu bude použito 80 \% z datového souboru, na validaci je použito zbylých 20 \%.
Natrénované modely jsou vyhodnoceny pomocí Waldova testu, který se nachází v každé tabulce v posledním sloupci $Pr(>|z|)$. Pro určení významnosti je použitá
\textit{hladina významnosti} $\alpha = 0,05$. Případné nevýznamné prediktory jsou pak z modelu odstraněny a model je natrénován znovu na stejné trénovací množině. K
validaci modelu je vytvořena matice záměn, která popisuje přesnost predikcí modelu na validačních datech. K následnému vyhodnocení modelů jsou použity statistiky
\textit{přesnost}, \textit{senzitivita} a \textit{specificita}. 

\newpage
\section{Průzkumová analýza dat}
Průzkumová analýza vizualizuje prediktory, jejich rozdělení a hledá mezi nimi různé vztahy. Díky průzkumu lze určit, které proměnné není vhodné použít pro tvorbu
logistického regresního modelu, např. kvůli problému multikolinearity.

\subsection{Korelační matice}
Pro logistickou regresi je důležité, aby prediktory nebyly lineárně závislé. Přehled korelací mezi kvantitativními prediktory lze zjistit z korelační matice.

\begin{figure}[H]
    \centering
    \includegraphics{../obrazky/prediktory_corr_matice.png}
    \caption{Korelační matice} 
    \label{fig:korelacni_matice}
\end{figure}

Z korelační matice uvedené na obrázku \ref{fig:korelacni_matice} lze vyčíst, že korelace mezi rankem týmu a charakteristikami hráčů se blíží nule. Z toho plyne, že neexistuje
lineární závislost mezi výkonem hráče a rankem týmu. Zároveň je vidět silná korelace mezi prediktorem \textit{rating} a prediktory \textit{fkdiff}, \textit{deaths} a
\textit{kills}. Jelikož by kvůli vysoké korelaci prediktorů vznikl problém multikolinearity, prediktor \textit{rating} ve finálních modelech není použit.

\subsection{Histogramy kvantitativních prediktorů}

\begin{figure}[H]
    \centering
    \includegraphics{../obrazky/histogram_prediktoru.png}
    \caption{Histogram prediktorů} 
    \label{fig:histogram_prediktoru}
\end{figure}

Histogramy prediktorů z obrázku \ref{fig:histogram_prediktoru} ukazují, že prediktory \textit{kills}, \textit{deaths} a \textit{hs} mají normální rozdělení
a a nevykazují mnoho extrémních hodnot. Prediktor \textit{fkdiff} má bimodální rozdělení. Prediktor \textit{assists} je 
zešikmený doprava, což značí velké množství odlehlých či extrémních hodnot. Logistická regrese nemá předpoklad normálního rozdělení prediktorů, a proto analýza
slouží čistě k získání povědomí o tom, jakých hodnot každý prediktor nabývá a jaké je jejich rozdělení.

\subsection{Míra výhry pro nejlepší týmy}
Do logistického modelu je nutné vybrat referenční tým, který je považován za jeden z nejlepších. 

\begin{figure}[H]
    \centering
    \includegraphics{../obrazky/mira_vyhry_tymu.png}
    \caption{Procento vyhraných zápasů na dané mapě za stranu counter-terroristů} 
    \label{fig:mira_vyhry_tymu}
\end{figure}

Obrázek \ref{fig:mira_vyhry_tymu} ukazuje, že tým Astralis má z týmů v datovém souboru nejlepší míru výhry. Rozdíl mezi mírou výhry Astralis a týmu SK je necelých 7 procent.
Z tohoto důvodu je tým Astralis použit při analýze dat jako jeden z referenčních týmů.

\newpage
\section{Predikce výhry hráčů na různých mapách}
Cílem modelů je predikovat výhru zápasu pro jednotlivé hráče na různých mapách a identifikovat významné prediktory s nimi spojené. Model lze využít na profesionální
úrovni k identifikaci toho, jaká charakteristika nejvíce přispívá k výhře či prohře hráče. Prediktory se týkají pouze výkonu jednotlivých
hráčů, model tedy pro předpověď výhry hráče nepoužívá charakteristiky spoluhráčů. K modelování je použit spojený datový soubor, který je detailně popsaný v
sekci \ref{sec:spojeny_datovy_soubor}. Obecně lze model zapsat následovně:

\begin{align}
    \begin{split}
        &P(1 | X_{kills}, X_{assists}, X_{deaths}, X_{fkdiff}, X_{starting\_ct}) = \frac{1}{1 + e^{-z}} \\
        &z = \beta_0 + \beta_1*X_{kills} + \beta_2*X_{assists} + \beta_3*X_{deaths} + \\
        &+ \beta_4*X_{fkdiff} + \beta_5*X_{starting\_ct}.
    \end{split}
\end{align}

Porovnány jsou mapy Mirage a Vertigo. Mapa Mirage je jednou z nejtradičnějších map, mapa Vertigo je naopak nejnovějším přídavkem do hry.
Díky rozdílným modelům bude možné zkoumat, jaký prediktor hraje významnou roli při predikování výhry hráče na různých mapách.

\subsection{Model pro mapu Mirage}

\input{modely/Mirage_model.tex}

Tabulka \ref{tab:Mirage_model} představuje model pro mapu Mirage se všemi významnými prediktory. Prediktory \textit{kills}, \textit{assists} a \textit{fkdiff} šanci na výhru
hráče zvyšují. Naopak prediktory \textit{deaths}, \textit{hs} a \textit{starting\_ct} šanci snižují. Model lze také zapsat jako přepis funkce:

\begin{align}
    \begin{split}
        &P(1 | X_{kills}, X_{assists}, X_{deaths}, X_{fkdiff}, X_{starting\_ct}) = \frac{1}{1 + e^{-z}} \\
        &z = 2,3954 + 0,1807*X_{kills} + 0,3031*X_{assists} - 0,3661*X_{deaths} - \\
        &- 0,1957*X_{hs} + 0,0171*X_{fkdiff} - 0,1995*X_{starting\_ct}.
    \end{split}
\end{align}

Ze získaných koeficientů lze určit změnu šance na výhru s každým dalším zabitím, a to umocněním parametru $\beta_1 = 0,1807$ na konstantu $e$.

\begin{align}
    \begin{split}
        \textit{změna šance} &\sim e^{\beta_1} \\
                             &\sim e^{0,1807} \\
                             &\sim 1,1980
    \end{split}
\end{align}

S každým dalším zabitým hráčem se zvyšuje šance na výhru hráče zhruba 1,2krát. U prediktoru \textit{deaths} je parametr $\beta_3$ záporný. To znamená, že se šance
na výhru snižuje.

%%% https://www.statology.org/interpret-odds-ratio-less-than-1/
\begin{align}
    \begin{split}
        \textit{změna šance} &\sim e^{\beta_3} \\
                             &\sim e^{-0,3661} \\                     
                             &\sim 0,6934
    \end{split}
\end{align}

Parametr lze interpretovat tak, že s každou další hráčovou smrtí se jeho šance na výhru sníží o zhruba 31 \%, k čemuž lze dojít výpočtem $(1 - 0,6934)*100$. 

\subsubsection{Matice záměn pro mapu Mirage}
K predikci je použitá validační podmnožina a model z tabulky \ref{tab:Mirage_model}.

\input{matice/Mirage_matice.tex}

\input{statistiky/Mirage_stats.tex}

Z tabulky \ref{tab:Mirage_matice} lze vyčíst, že model predikoval správně 7 037 výher ($\sim 81,31 \,\% $) a 6 804 proher ($\sim 77,83 \,\% $).
Celkově model určil správně 13 841 objektů ($\sim 79,56 \,\% $). Statistiky z matice záměn jsou zobrazené v tabulce \ref{tab:Mirage_stats}.

\newpage
\subsection{Model pro mapu Vertigo}
Model pro mapu Vertigo má identické vstupní prediktory, podle kterých se predikuje proměnná \textit{map\_winner}, jako model pro mapu Mirage.

\input{modely/Vertigo_model.tex}

Pro mapu Vertigo se v modelu nacházejí nevýznamné prediktory, proto z něj byly prediktory \textit{hs}, \textit{fkdiff} a \textit{starting\_ct} odebrány.
Následně byl model znovu natrénován.

\input{modely/Vertigo_model_opt.tex}

Pro hráče jsou na mapě Vertigo významné pouze prediktory \textit{kills}, \textit{assists} a \textit{deaths}.
Optimalizovaný model lze zapsat do následující rovnice:

\begin{align}
    \begin{split}
        &P(1 | X_{kills}, X_{assists}, X_{deaths}) = \frac{1}{1 + e^{-z}} \\
        &z = 1,9090 + 0,1841*X_{kills} + 0,2981*X_{assists} - 0,3521*X_{deaths}.
    \end{split}
\end{align}

Každé zabití protihráče vede ke zvýšení šance hráče na výhru na mapě Vertigo o zhruba 20 \%. Každá další hráčova smrt vede ke snížení šance na výhru o zhruba 30 \%. 
Matice záměn i statistiky modelu jsou potom vyhodnocené na optimalizovaném modelu pomocí validačních dat.

\input{matice/Vertigo_matice_opt.tex}

\input{statistiky/Vertigo_stats_opt.tex}

Z tabulek \ref{tab:Vertigo_matice_opt} a \ref{tab:Vertigo_stats_opt} je patrné, že model predikoval správně 455 výher ($\sim 77,65 \,\% $) a 411 proher ($\sim 78,44 \,\% $).
Celkově model určil správně 866 objektů ($\sim 78,02 \,\% $).

\subsection{Interpretace a porovnání modelů}
Na mapě Mirage jsou pro hráče významné všechny prediktory. Nejvíc pozitivní vliv má na výhru hráče počet asistencí a největší negativní vliv
na výhru hráče má pak počet smrtí. Významnost prediktorů \textit{fkdiff} a \textit{assists} naznačuje, že je důležitá souhra hráčů a že zkušenější týmy
mají na mapě výhodu. Pokud hráči hrají spolu, mohou se při zabíjení nepřátel doplňovat (prediktor \textit{assists}). Zároveň je důležité, aby se v zápase
hráči navzájem podporovali a mohli získat první zabití v daném kole (prediktor \textit{fkdiff}). Pokud hráči začínají hrát mapu Mirage na straně counter-terroristů,
je jejich šance na výhru nižší.

Pro mapu Vertigo jsou  významné pouze prediktory \textit{kills}, \textit{assists} a \textit{deaths}. Pro hráče tedy není důležité, jak přesně
střílí (prediktor \textit{hs}), jak dobře se tým podporuje na začátku kola (prediktor \textit{fkdiff}) ani na jaké straně hráč začíná (prediktor \textit{starting\_ct}).
Největší vliv na výhru zde má prediktor \textit{assists}, největší vliv na prohru má pak prediktor \textit{deaths}. Porovnání obou statistik obou modelů je pak
v následující tabulce.

\input{tabulky/Mirage_Vertigo_porovnani.text}

Mapa Mirage se  díky své vysoké specificitě hodí nejvíce na predikci proher. Rozdíly mezi statistikami pro mapu Vertigo jsou pak nepatrné a pohybují se okolo 1 \%.
Největší rozdíl mezi modely je ve specificitě, která je pro model na mapě Mirage větší o zhruba 3,5 procentních bodů. Z modelů vyplývá, že je pro hráče na obou
mapách důležitější méně umírat, než více zabíjet.

Modely, matice záměn i vybrané statistiky z matic záměn pro ostatní mapy se nacházejí v příloze \ref{apx:02}. Zajímavé je, že pro všechny ostatní modely
jsou významné všechny prediktory. To z mapy Vertigo, která má významné pouze 3 prediktory, dělá poněkud unikátní mapu. Z výsledku lze usoudit, že
pro všechny mapy, které se hrají už nějakou dobu, je důležitý jak výkon hráče, tak sehranost týmu. Mapa Vertigo je nová, je na začátku svého
vývoje a pro hráče je stále relativně neznámá. Proto je na mapě Vertigo méně významná sehranost, zkušenosti a spolupráce v týmu. Měly by ji tak preferovat
nové či semi-profesionální týmy. Mapu také mohou využít profesionální týmy, které spoléhají spíše na individuální výkony hráčů než na týmové strategie.

\newpage
\section{Predikce výhry týmu}
Cílem modelů je predikovat výhru týmu na základě agregovaných charakteristik hráčů týmu. Tento typ modelu je užitečný zejména v sázkových kancelářích.
Celkový model je také vhodný při predikci výhry týmu, pro který není dostatek dat. V takovém případě by totiž nebylo možné vytvořit separátní model, a predikci výhry tak
lze stanovit pomocí jednoho celkového modelu. Díky referenčním modelům lze určit šanci, že vyhraje určitý tým. Modely umožní sázkovým kancelářím stanovit výdělečný kurz pro zápas.
Matematický přepis modelu je následující:

\begin{align}
    \begin{split}
        P(1 | &X_{mean\_kills}, X_{mean\_assists}, X_{mean\_deaths}, X_{mean\_hs}, X_{mean\_fkdiff}, X_{team\_rank}, \\
              &X_{mapCache*starting\_ct}, X_{mapCobblestone*starting\_ct}, X_{mapDust2*starting\_ct}, X_{mapInferno*starting\_ct}, \\
              &X_{mapMirage*starting\_ct}, X_{mapNuke*starting\_ct}, X_{mapOverpass*starting\_ct}, X_{mapTrain*starting\_ct}, \\
              &X_{mapVertigo*starting\_ct}) = \frac{1}{1 + e^{-z}} \\
        z = &\beta_0 + \beta_1*X_{mean\_kills} + \beta_2*X_{mean\_assists} + \beta_3*X_{mean\_deaths} + \\
            &+ \beta_4*X_{mean\_hs} + \beta_5*X_{mean\_fkdiff} + \beta_6*X_{team\_rank} + \\
            &+ \beta_7*X_{mapCache*starting\_ct} + \beta_8*X_{mapCobblestone*starting\_ct} + \\
            &+ \beta_9*X_{mapDust2*starting\_ct} +\beta_{10}*X_{mapInferno*starting\_ct} + \\
            &+ \beta_{11}*X_{mapMirage*starting\_ct} + \beta_{12}*X_{mapNuke*starting\_ct} + \\
            &+ \beta_{13}*X_{mapOverpass*starting\_ct} + \beta_{14}*X_{mapTrain*starting\_ct} + \\
            &+ \beta_{15}*X_{mapVertigo*starting\_ct}.
    \end{split}
\end{align}

Prediktory \textit{mean\_kills}, \textit{mean\_assists}, \textit{mean\_deaths}, \textit{mean\_hs} a  \textit{mean\_fkdiff} jsou průměrné charakteristiky hráčů v týmu na dané mapě
v daném zápase a lze očekávat, že šanci na výhru ovlivňují. Dále do modelu vstupuje prediktor \textit{team\_rank}, který značí rank týmu v daném zápase. Model následně
obsahuje interakci mezi proměnnou \textit{map} a \textit{starting\_ct}. Interakce se objevuje zejména z toho důvodu, že vliv počáteční strany může být na každé mapě jiný.
Přesněji jsou prediktory
definované v sekci \ref{sec:agregovany_datovy_soubor}

\subsection{Celkový model}

\input{modely/Global_model.tex}

Parametry celkového modelu se nachází v tabulce \ref{tab:Global_model}. První vytvořený model je sestavený na celém trénovacím datovém souboru. Pro model jsou
významné všechny prediktory bez interakce, jmenovitě \textit{mean\_kills}, \textit{mean\_assists}, \textit{mean\_deaths}, \textit{mean\_hs},\textit{mean\_fkdiff} a
\textit{team\_rank}. Interakce mezi prediktory \textit{map} a \textit{starting\_ct} není významná u map Vertigo, Cache ani Nuke.

\begin{align}
    \begin{split}
        P(1 | &X_{mean\_kills}, X_{mean\_assists}, X_{mean\_deaths}, X_{mean\_hs}, X_{mean\_fkdiff}, X_{team\_rank}, \\
              &X_{mapCache*starting\_ct}, X_{mapCobblestone*starting\_ct}, X_{mapDust2*starting\_ct}, X_{mapInferno*starting\_ct}, \\
              &X_{mapMirage*starting\_ct}, X_{mapNuke*starting\_ct}, X_{mapOverpass*starting\_ct}, X_{mapTrain*starting\_ct}, \\
              &X_{mapVertigo*starting\_ct}) = \frac{1}{1 + e^{-z}} \\
        z = &  0,3423 + 1,3688*X_{mean\_kills} + 0,1321*X_{mean\_assists} - 1,3958*X_{mean\_deaths} - \\
            &- 0,4997*X_{mean\_hs} - 0,0741*X_{mean\_fkdiff} - 0,0009*X_{team\_rank} - \\
            &- 0,0240*X_{mapCache*starting\_ct} - 0,3211*X_{mapCobblestone*starting\_ct} - \\
            &- 0,2051*X_{mapDust2*starting\_ct} - 0,2993*X_{mapInferno*starting\_ct} - \\
            &- 0,2362*X_{mapMirage*starting\_ct} - 0,1404*X_{mapNuke*starting\_ct} - \\
            &- 0,4307*X_{mapOverpass*starting\_ct} - 0,2341*X_{mapTrain*starting\_ct} + \\
            &+ 0,2211*X_{mapVertigo*starting\_ct}
    \end{split}
\end{align}

Agregované charakteristiky hráče \textit{mean\_kills} a \textit{mean\_assists} šanci na výhru týmu zvyšují. Naopak prediktory \textit{mean\_deaths}, \textit{mean\_hs}, 
\textit{mean\_fkdiff} a \textit{team\_rank} šanci na výhru zmenšují. Pokud se průměr zabitých nepřátel za tým zvýší o jednotku,
šance na výhru týmu se zvýší zhruba 3,93krát. Pokud se průměr smrtí hráčů za tým zvýší o jednotku, šance na výhru se sníží o zhruba 75 \%. Všechny statisticky
významné interakce mezi prediktory \textit{map} a \textit{starting\_ct} naznačují, že je pro tým nevýhodné začínat na straně counter-terroristů. Jejich šance
na výhru se vždy sníží, a to nejvíce na mapě Overpass, kde dojde zhruba k 35\% zhoršení.

\subsubsection{Matice záměn pro obecný model}

\input{matice/Global_matice.tex}

\input{statistiky/Global_stats.tex}

Dle matice \ref{tab:Global_matice} model úspěšně predikoval 8 321 výher ($\sim 94,77 \,\% $) a 8 337 proher ($\sim 94,16 \,\% $). Celkem
model predikoval správně 16 658 objektů ($\sim 94,47 \,\% $). Všechny výkonnostní statistiky z matice záměn se liší v řádu tisícin procent.
Z tabulky \ref{tab:Global_stats} nelze jednoznačně určit, zda je tým vhodnější k predikci výher, či proher.

\subsection{Model pro tým Astralis}

\input{modely/Astralis_model_opt.tex}

Po odstranění nevýznamných prediktorů lze získat logistický model pro tým Astralis, který je popsán v tabulce \ref{tab:Astralis_model_opt}. Pro tým Astralis jsou významné pouze prediktory
\textit{mean\_kills} a \textit{mean\_deaths}. 

\begin{align}
    \begin{split}
        P(1 | &X_{mean\_kills}, X_{mean\_deaths}, X_{team\_rank}) = \frac{1}{1 + e^{-z}} \\
        z =   &1,5840 + 1,3331*X_{mean\_kills} - 1,4069*X_{mean\_deaths}
    \end{split}
\end{align}

S každým dalším průměrným zabitím (prediktor \textit{mean\_kills}) se zvýší šance týmu Astralis na výhru zhruba 3,84krát. S každou další průměrnou smrtí
(prediktor \textit{mean\_deaths}) se šance na výhru týmu sníží o zhruba 76 \%.

\subsubsection{Matice záměn pro tým Astralis}

\input{matice/Astralis_matice_opt.tex}

\input{statistiky/Astralis_stats_opt.tex}

Z matice záměn v tabulce \ref{tab:Astralis_matice_opt} lze vyčíst, že model predikoval správně 44 výher ($\sim 88,00 \,\%$) a 118 proher ($\sim 94,40 \,\%$).
Celkem optimální model predikoval správně 162 objektů ($\sim 92,57 \,\%$). Statistiky jsou zobrazené v tabulce \ref{tab:Astralis_stats_opt}.

\subsection{Model pro tým Sprout}
Tým Sprout byl v době extrakce dat čistě německým týmem. Patřil k průměrným profesionálním týmům, v žebříčků týmů se obvykle řadil okolo třicátého místa.

\input{modely/Sprout_model_opt.tex}

Tabulka \ref{tab:Sprout_model_opt} obsahuje významné prediktory pro model týmu Sprout, kterými jsou pouze \textit{mean\_kills} a \textit{mean\_deaths}.
Matematický zápis modelu obsahuje následující přepis.

\begin{align}
    \begin{split}
        P(1 | &X_{mean\_kills}, X_{mean\_deaths}, X_{team\_rank}) = \frac{1}{1 + e^{-z}} \\
        z =   &-2,7474 + 1,5741*X_{mean\_kills} - 1,4307*X_{mean\_deaths}
    \end{split}
\end{align}

S každým dalším průměrným zabitím (prediktor \textit{mean\_kills}) se šance na výhru týmu zvýší zhruba 4,83krát. S každou další průměrnou smrtí (prediktor \textit{mean\_deaths})
se šance na výhru sníží zhruba o 76 \%.

\input{matice/Sprout_matice_opt.tex}

\input{statistiky/Sprout_stats_opt.tex}

Z matice záměn v tabulce \ref{tab:Sprout_matice_opt} vyplývá, že model predikoval správně 47 výher ($\sim 92,16 \,\%$) a 71 proher ($\sim 98,61 \,\%$).
Celkem optimální model predikoval správně 118 objektů ($\sim 95,93 \,\%$). Statistiky jsou zobrazené v tabulce \ref{tab:Sprout_stats_opt}.

\subsection{Interpretace a porovnání modelů}
Pro celkový model, tedy model pro predikci výhry jakéhokoliv týmu, jsou významné všechny průměrné charakteristiky hráčů. Šanci na výhru nejvíce zvyšuje každé
další průměrné zabití. Šanci na prohru pak nejvíce zvyšuje každá další průměrná smrt. Z modelu lze usoudit, že při určování šance na výhru na určitých mapách je významné, na které 
straně tým začíná. Pokud začíná na straně counter-terroristů, jeho šance na výhru je na mapách Cobblestone, Dust 2,
Inferno, Mirage, Overpass a Train nižší. Na mapách Cache, Nuke a Vertigo je pak strana, na které tým začíná, nevýznamná. Statistiky z matice záměn jsou velmi podobné, a proto
lze model využít jak k predikci výhry, tak prohry. Na rozdíl od modelů pro tým Astralis a Sprout je model velmi komplexní a obsahuje 15 prediktorů.

Model pro tým Astralis naznačuje, že je pro tým významný pouze průměrný počet zabití a průměrný počet úmrtí. Výhru týmu neovlivňuje průměrný počet asistencí, průměrné
procento zabití do hlavy, průměrný rozdíl prvního zabití a úmrtí, rank týmu v okamžik zápasu ani počáteční strana na jakékoliv mapě. Vytvořený model pro tým
Astralis je vhodný spíše k predikci výhry než prohry. Na šanci na výhru týmu Astralis má větší negativní vliv průměrný počet smrtí, proto by se měl tým soustředit více
na bezpečnou a taktickou hru.

Významnými prediktory pro tým Sprout jsou pouze průměrný počet zabití a průměrný počet smrtí. Výhru týmu tedy neovlivňuje průměrný počet asistencí, průměrné
procento zabití do hlavy, průměrný rozdíl prvního zabití a úmrtí, rank týmu v okamžik zápasu ani počáteční strana na jakékoliv mapě. Vliv průměrného počtu zabití je větší než 
vliv průměrného počtu smrtí, a proto by měl tým více riskovat a hrát spíše na vysoký počet průměrného zabití. Tento model je nejvhodnější k predikci výhry týmu, méně je 
hodný k predikci prohry.

\input{tabulky/Tymy_porovnani.text}

Z modelů lze usoudit, že nejlepší tým na světě i průměrný
profesionální tým má stejné významné prediktory. Tabulka \ref{tab:Tymy_porovnani} ukazuje porovnání mezi jednotlivými statistikami přes vybrané modely. Model pro tým
Sprout má největší přesnost  i senzitivitu, zatímco celkový model má největší specificitu. Ze statistik vychází, že nejméně spolehlivě předpovídá model pro
referenční tým Astralis. 

Rozdíl mezi referenčními modely a celkovým modelem by bylo možné vysvětlit např. vlivem neprofesionálních týmů. Amatérské týmy jsou nevyrovnané a musí využít každé možné výhody,
kterou mohou získat, aby vyhráli. Profesionální týmu jsou pak velmi vyrovnané. Z tohoto důvodu lze usoudit, že je pro amatérské týmy je mnoho prediktoru významných. Když proti sobě hrají profesionální týmy, jejich rozdíl je
méně patrný, a je proto nejvíce signifikantní výkon jednotlivých hráčů.