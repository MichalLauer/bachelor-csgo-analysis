% Verze pro jednostranný tisk:
\documentclass[11pt,a4paper]{report}
\usepackage[top=25mm,bottom=25mm,right=25mm,left=30mm,head=12.5mm,foot=12.5mm]{geometry}
\let\openright=\clearpage

%% Definice různých užitečných maker (viz popis uvnitř souboru)
\input{makra}

%%% Údaje o práci
% Název práce v jazyce práce (přesně podle zadání)
\def\NazevPrace{Modely logistické regrese v oblasti esportových dat}

% Typ práce
\def\TypPrace{BAKALÁŘSKÁ PRÁCE}


% Jméno autora
\def\AutorPrace{Michal Lauer}

% Rok odevzdání. měsíc (slovně)
\def\DatumOdevzdani{Duben 2022}

% Vedoucí práce: Jméno a příjmení s~tituly
\def\Vedouci{Ing. Zdeněk Šulc, Ph.D.}

% Studijní program a obor
\def\StudijniProgram{Aplikovaná informatika}
\def\StudijniObor{Aplikovaná informatika}

% Text čestného prohlášení pro MUŽE pro bakalářskou práci
\def\Prohlaseni{Prohlašuji, že jsem bakalářskou práci \textit{\NazevPrace} vypracoval samostatně za použití v práci uvedených pramenů a literatury.}

% Nepovinné poděkování (vedoucímu práce, konzultantovi, tomu, kdo
% zapůjčil software, literaturu apod.)
\def\Podekovani{%
Rád bych poděkoval panu doktorovi Zdenku Šulcovi, který mou bakalářskou práci podpořil, i přes odlišný studijní obor.
Dále děkuji autorům knih, jmenovitě ... , za poskytnutou příležitost se ve logistických modelech zlepšit. Bez nich by se práce psala velmi složitě.
}

% Abstrakt (doporučený rozsah cca 150-250 slov; nejedná se o zadání práce)
\def\Abstrakt{%
Práce se zabývá predikcí výsledku esportových zápasů dle několika proměnných. V prvním rámci bakalářské práce je v krátkosti představen esport a problematika, kterou se práce zabývá. Dále jsou zde
představeny důležité termíny a pojmy, které jsou klíčkové ke správné interpretaci výsledků. V druhé části, která se zabává teorií, jsou představeny metody k popisu dat, modely, a hodnocení kvality modelů. V poslední části
jsou metody  použity v praxi. Nejprve je představen dataset a jeho proměnné. Z proměnných je vybráno pouze několik hlavních prediktorů, které jsou následně použity k výsledné predikci. Dále se zde nachází
popis prediktorů a pomocí grafů a slovní interpretace. V závěru je zde sestaven multivariabilní logistický regresní model, který předpovídá výsledek zápasů. Ten je vyhodnocen vyhodnocen pomocí již zmínených statistik.    
}
\def\AbstraktEN{%
-- Bude přeložen po odsouhlasení abstraktu v češtině
}

% 3 až 5 klíčových slov (doporučeno)
\def\KlicovaSlova{klíčové slovo, další pojem, jiný důležitý termín, a ještě jeden}
\def\KlicovaSlovaEN{keyword, important term, another topic, and another one}

%% Titulní strana a různé povinné informační strany
\begin{document}
\include{zacatek}

%%% Strana s automaticky generovaným obsahem bakalářské práce
\setcounter{tocdepth}{2}
\tableofcontents

%%% Obrázky v bakalářské práci
\openright
\listoffigures

%%% Tabulky v bakalářské práci (opět nemusí být nutné uvádět)
\clearpage
\listoftables

%%% Použité zkratky v bakalářské práci (opět nemusí být nutné uvádět)
\chapter*{Seznam použitých zkratek}
\addcontentsline{toc}{chapter}{Seznam použitých zkratek}

\begin{acronym}[CS:GO]
    %% esport tituly
    \acro{CS:GO}{Coutner-Strike: Global Offensive}
    %% esport žánry
    \acro{BR}{Battle Royale \textit{(hra o přežití)}}
    \acro{MOBA}{Multiplayer online battle arena}
    \acro{FPS}{First-person shooter \textit{(střílečka z pohledu první osoby)}}
    %% ostatní
    \acro{TGNS}{Twin Galaxies National Scoreboard}
\end{acronym}


\pagestyle{fancy}
%%% Jednotlivé kapitoly práce jsou pro přehlednost uloženy v samostatných souborech
\chapter*{Úvod}
\addcontentsline{toc}{chapter}{Úvod}
Esport je jedno z nejrychleji rostoucích odvětví v dnešní době. V roce 2021 se jeho tržní hodnota pohybovala kolem jedné miliardy dolarů - skoro
50\% nárůst oproti roku 2020. Dle portálu statista.com lze předpovídat, že v roce 2024 esport překročí hodnotu 1,5 miliardy dolarů \cite{Gough2021}.
Dalo by se spekulovat, že za takový velký nárůst je zodpovědná aktuální pandemie. Většina populace, hlavně ta mladší, je nucena zůstat doma. Toto otevřelo dveře
se s esportem přirozeně seznámit a nějakým způsobem se ho účastnit \textit{(online divák, soutěžící, organizátor, fanoušek...)}. Esport je vlastně sport,
akorát s počítačovými hrami. Hrají se různé kategorie her - střílečky \textit{(\ac{CS:GO}, Valorant)}, arény \textit{(\ac{LoL})}, či karetní hry
\textit{(\ac{HS})}.

Toto téma jsem si zvolil hlavně kvůli tomu, že se o oblast zajímám od mého mládí. Když jsem si vybíral téma na bakalářskou práci,
chtěl jsem propojit statistiku s něčím, co mě baví a naplňuje - toto je ideální kombinace. Zároveň bylo mím cílem vytvořit práci, která bude v dnešní době relevantní.
Zvolené téma je dle mého názoru velmi aktuální, avšak ne pro širokou veřejnost, nýbrž pouze pro lidi, co se o zajímají o esport či sázení.
Podobné logistické modely, avšak velmi složitější, mohou totiž sloužit například k vyhodnocení sázkových kurzů. Esport, tak jak klasický sport, je se sázením propojen. 

Finální cíl práce je vytvořit logistický model, který předpovídá výsledek zápasů. Tento model je vyhodnocen různými klasifikacemi pro vyhodnocení kvality.
Práce také popisuje grafické metody vizualizace dat a teorii k tvorbě a vyhodnocení logistických modelů. Také zde najdeme popis datového souboru a postup,
jakým byli vybráni nejvýznamnější prediktory - ať už statisticky, či čistě ze znalosti esportu. 

\include{kap01}
\include{kap02}
\include{kap03}
\include{kap04}
% \include{...}
% \include{...}
\include{zaver}

%%% Seznam použité literatury
%% Toto platí v případě použití samostatné bibliografické databáze
\printbibliography[title={Seznam použitého softwaru}, heading={bibintoc}, keyword={sw}]

\printbibliography[title={Seznam použité literatury}, heading={bibintoc}, notkeyword={ezdroj,sw}]

\printbibliography[title={Seznam elektronických zdrojů}, heading={bibintoc}, keyword={ezdroj}]


%%% Přílohy k bakalářské práci, existují-li. Každá příloha musí být alespoň jednou
%%% odkazována z vlastního textu práce. Přílohy se číslují.
\part*{Přílohy}
\appendix
\chapter{Datové soubory}
\section{Původní soubor players.csv}




\include{app02}
% \include{...}
% \include{...}

\end{document}
