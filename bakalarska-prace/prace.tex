% Verze pro jednostranný tisk:
\documentclass[11pt,a4paper]{report}
\usepackage[top=25mm,bottom=25mm,right=25mm,left=30mm,head=12.5mm,foot=12.5mm]{geometry}
\let\openright=\clearpage

%% Definice různých užitečných maker (viz popis uvnitř souboru)
\input{makra}

%%% Údaje o práci
% Název práce v jazyce práce (přesně podle zadání)
\def\NazevPrace{Modely logistické regrese v oblasti esportových dat}

% Typ práce
\def\TypPrace{BAKALÁŘSKÁ PRÁCE}

% Jméno autora
\def\AutorPrace{Michal Lauer}

% Rok odevzdání. měsíc (slovně)
\def\DatumOdevzdani{Duben 2022}

% Vedoucí práce: Jméno a příjmení s~tituly
\def\Vedouci{Ing. Zdeněk Šulc, Ph.D.}

% Studijní program a obor
\def\StudijniProgram{Aplikovaná informatika}
\def\StudijniObor{Aplikovaná informatika}

% Text čestného prohlášení pro MUŽE pro bakalářskou práci
\def\Prohlaseni{Prohlašuji, že jsem bakalářskou práci \textit{\NazevPrace} vypracoval samostatně za použití v práci uvedených pramenů a literatury.}

% Nepovinné poděkování (vedoucímu práce, konzultantovi, tomu, kdo
% zapůjčil software, literaturu apod.)
\def\Podekovani{%
Rád bych poděkoval panu doktoru Zdeňku Šulcovi, který mou bakalářskou práci podpořil a vedl, i přes odlišný studijní obor.
Dále děkuji autorům citovaných knihm, článků a softwaru za poskytnutou příležitost se v logistických modelech zlepšit a napsat 
na toto téma bakalářskou práci.
}

% Abstrakt (doporučený rozsah cca 150-250 slov; nejedná se o zadání práce)
\def\Abstrakt{%
Esport je sport ve virtuálním světě, který se od počátku dvacátého prvního století rozrůstá mezi novou generací, která
vyrůstala ve světě počítačů, mobilů a technologií. Esport nejsou jen klasické sporty jako fotbal, hokej či rugby, ale soutěžit se může
i v různých počítačtích, mobilních či konzolovích hrách. Tato práce je zaměřená na jednu z počítačovích her, a to na hru \ac{CSGO}.
Téma je aktuální zejména díky tomu, že je esport ve světě relativně nová záležitost a neustále se vyvíjí. Na to se musí přizpůsobit například
sázkové kanceláře, které využívají analýzu esportových dat pro nabízení mnoha různých kurzů. 

Cílem bakalářské práce je kvantitativně zanalyzovat esportové zápasy ze hry \ac{CSGO}, predikovat výhry
hráčů a týmů a zjistit, jaké prediktory jsou pro výhru zápasu statisticky významné. Použitý datový soubor je z internetového portálu
kaggle.com a obsahuje data od roku 2015 až do roku 2020.

Teoretická část práce se zabývá představením a historií esportu a esportové hry \ac{CSGO}. Teoretická část je pak zaměřená na analýzu dat v 
programovacím jazyku R. Predikce výher je založena na logistickém vícerozměrném modelu a k jeho vyhodnocení je použita matice záměn a statistiky z 
ní vypočítané. Pro určení významnosti prediktorů je použit Waldův test.

Výsledek práce jsou logistické modely, které jsou schopné predikovat výhru hráče či týmu podle různých charakteristik. Zároveň jsou
identifikované významné prediktory, které výhru zápasu ovlivňují. Toto zjištění by bylo možné použit např. pro stanovení kurzu sázkovou 
kanceláří na výhru či prohru daného hráče či týmu.
}

\def\AbstraktEN{%
{\color{red}
Přeložit před odevzdáním
}
}


% 3 až 5 klíčových slov (doporučeno)
\def\KlicovaSlova{model, logistická regrese, predikce, esport}
\def\KlicovaSlovaEN{model, logistic regression, prediction, esport}

%% Titulní strana a různé povinné informační strany
\begin{document}
\pagenumbering{gobble}
\include{zacatek}

%%% Strana s automaticky generovaným obsahem bakalářské práce
\setcounter{tocdepth}{2}
\tableofcontents

\pagestyle{fancy}
\pagenumbering{arabic}
%%% Jednotlivé kapitoly práce jsou pro přehlednost uloženy v samostatných souborech
\chapter{Úvod}
Esport je označení pro elektronický sport. Obsahuje všechny důležité oblasti jako klasický sport (např. turnaje, trénování, investice, stadiony, či sázení)
s tím rozdílem, že se hraje na nějakém zařízení (počítač, konzole, mobil). Je to jedno z nejrychleji rostoucích odvětví v dnešní době. V roce 2021 se tržní hodnota
esportu pohybovala kolem jedné miliardy dolarů --- skoro 50\% nárůst oproti roku 2020. Lze předpovídat, že v roce 2024 esport překročí hodnotu 1,5 miliardy dolarů.
Dalo by se spekulovat, že za takový velký nárůst je zodpovědná pandemie koronaviru v letech 2019 --- 2022. Kombinace rozvoje počítačových her a generace, která je na
práci s počítačem zvyklá od mala, vzniklo výborné prostředí pro organický růst esportu. Většina populace byla nucena zůstat doma a to otevřelo dveře
se s esportem přirozeně seznámit a nějakým způsobem se ho účastnit (online divák, soutěžící, organizátor, fanoušek...). 

Práce se zaměřuje primárně na esportový titul \acf{CSGO}. Je to jeden z nejdéle hraných esportových titulů, boří mnohé divácké rekordy a je aktuálně nejhranějším
\ac{FPS} esport titulem. \ac{CSGO} vyniká nejen detailní herní mechanikou, ale i bohatou a zajímavou historií. Hra je unikátní i tím, že obsahuje mnoho různých
módů\footnote{rozšíření, jak hru hrát. Každý mód má svá vlastní pravidla, mapy, či herní fanoušky} a hráč může strávit mnoho hodin pouze objevováním komunitních
serverů, hraním klasických zápasů či trénováním na offline mapách.

Finálním cílem práce je vytvořit vícerozměrný logistický regresní model, který předpovídá výsledky zápasů. Pro tvorbu kvalitního modelu bude kritické zvolit vhodné
charakteristiky hráčů a jednotlivých týmů. Pro predikci jsou použity charakteristiky, které se nacházejí ve dvou samostatných datových
souborech, které podávají informace jak už o zápase (např. datum, výsledek zápasu, výsledek jednotlivých map, typ zápasu) a hráčích (např. charakteristiky za zápas). 

Predikovat výhru hráče a týmu je důležité zejména v oblasti sázení, která je s esportem úzce spojená. Přesné predikce a kvalitní modely sázkovým kancelářím umožňují
stanovovat výhodné a profitabilní kurzy. Kurzy pak nemusí být pouze na výhru či prohru, ale např. na charakteristiky hráčů či hranou mapu. Logistický model je 
pro predikci výhry či prohry preferován kvůli své snadné interpretaci.

Práce je rozdělená do tří částí. V první části je kladen důraz na esport, jeho vývoj, a na esportový titul \ac{CSGO}. Jsou zde také představená pravidla, podle kterých se
hra hraje. V druhé části jsou popsány popisné a statistické metody. Jsou zde definované grafické nástroje pro popis datového souboru, logistického
regresního modelu, a evaluační nástroje pro model. Třetí část se zaměřuje na praktickou tvorbu modelů, jejich interpretaci, a vzájemné porovnání.

\chapter{Představení esportu}
\section{Historie esportu}
I přes fakt, že esport není obecně známý pojem mezi širokou veřejností, má přes 70 let bohaté historie. Za jeho počátky by se
daly považovat arkádové automaty, kde hráči z počátku soutěžili sami proti sobě. Největší rozvoj arkádových automatů se děl kolem 70 let minulého 
století. Nejen za tímto účelem byla 9. 2. 1982 založena \textit{\ac{TGNS}}. \ac{TGNS} měla na starosti nejen udržování výsledkové tabulky \textit{(ang. scoreboard)},
ale i tvorbu prvotních pravidel pro férovou hru. Za tímto účelem byla vydána kniha \textit{Twin Galaxies' Official Video Game \& Pinball Book of World Records}.

Na přelomu osmdesátých let minulého století se začal esport vyvíjet již více profesionálním směrem. V roce 1972 pořádala Stanfordská Universita historicky první
esportový turnaj v arkádové hře \textit{Spacewar!}. Výherce si mohl odnést předplatné magazínu Rolling Stones. Dále v roce 1983 byl založen první esportový profesionální team,
který se nacházel ve Spojených státech. Všechno toto se stalo díky podnikateli Walteru Day, který je jak zakladatel společnosti \ac{TGNS} a založil již zmíněný
prvních esportový team. Ač se Walter považuje za jednoho z hlavních pionýrů esportu, v roce 2010 \ac{TGNS} opustil kvůli své vášni pro hudbu.

Další důležitou kapitolou ve vývoji esportu je příchod internetu a výkonných počítačů. Hráči měli rychlejší sestavy, stolní počítače byli cenově dostupnější a díky tomu
se dostali k více lidem. Klesala cena hardwaru, vývoj nové technologie a her se zrychloval. Díky rozvoji počítačových sítí se mohli hrát LAN party\footnote{Hráči hrají v jedné
místnosti na lokální počítačové síti.} či organizovat BYOC turnaje\footnote{z ang. Bring Your Own Computer, kde si hráči si na akci donesou vlastní počítač}. Dále už esport 
potřeboval jen čas a dnes má tržní hodnotu přes jednu miliardu amerických dolarů \cite{Gough2021}, \cite{Larch2019}.

\section{Zasazení do dnešní doby}
Jak již bylo zmíněno, esport je v dnešní době téměř miliardová záležitost. Díky pandemii, která trvá již třetím rokem, si esport ještě přilepšil. Dle průzkumu \cite{Gough2021a}
z října roku 2020 si 73 \% dotázaných myslí, že se úroveň zájmu \textit{(ang. level of investment)} a obchodní činnost \textit{(ang. deal activity)} v Q4 2020 a Q1 2021
zvětší. Respondenti, kteří se průzkumu zúčastnili, jsou považování za \uv{industry professionals}. Tento průzkum potvrzuje fakt, že tržní hodnota esportu mezi lety 2019 a 2020
vzrostla o téměř 50 \% \cite{Gough2021}.

K takto prudkému růstu tržní hodnoty esportu z velké části přispěla právě pandemie. Mladá generace byla nucena zůstat doma, což dovolilo i esportem nedotčeným jedincům do
tohoto světa proniknout. Větší zájem o esport přinesl i větší tržby herním studiím, které začali do esportových turnajů investovat více peněz\cite{Professeur2021}\cite{liquipedia2021}.
S větším počtem diváku roste i marketingový potenciál, investiční příležitost a kariérní růst.

V dnešní době má esport mnoho titulů, proto představím jen ty nejvýznamnější. Největší esport rivalita je mezi herním titule \ac{LoL} a Dota 2. Oba tituly jsou žánru \ac{MOBA}, díky 
čemuž mají podobnou, avšak velmi rozdílnou fanouškovskou základnu. Historie mezi tituly je velmi složitá, avšak mimo rozsah této práce. Pro rozšíření znalosti mohu doporučit videa 
z youtubového kanálu theScore esport o tomto tématu - \href{https://www.youtube.com/watch?v=h9Zv_TiVzmg}{The~Story of Dota 2} a \href{https://www.youtube.com/watch?v=tHtfD-MnQK8}{The Story of League of Legends}.

Druhý dominantní žánr je \ac{FPS}. V této kategorii dominují hry \ac{CS:GO} a Valorant. Zde proti sobě hrají dva teamu, většinou složené z pěti hráčů. Každý hráč pak má v teamu různou roli, jako např. velitel
či odstřelovač. Jeden team má obvykle za úkol něco zničit \textit{(položit bombu, unést rukojmí)} a druhý team jim v tom musí zabránit \textit{(ochránit oblast proti bombě, záchrana rukojmí)}.

Poslední žánr, který zmíním, je \ac{BR}. V těchto hrách hraje buď každý hráč za sebe, ve dvojicích, nebo skupinách po čtyřech. Zde hráči padají na začátku kola na velkou mapu. Jejich úkolem je
získat tzn. \uv{loot} \textit{(vybavení)}, aby mohl porazit ostatní hráče a kolo sám, nebo s týmem vyhrát. Nacházejí se zde různé role, avšak trošku rozdílné oproti žánru \ac{FPS}. Hlavním titulem
této kategorie je hra Fortnite, která žánru dominuje. Stal se z ní jak esport titul, tak perfektní marketingové místo pro teenagery. Hráči si zde mohou koupit oblečky různých filmových či komiksových postav.
Pokud vychází nový film, ve hře se může objevit \uv{event} \textit{(událost)}, který daný film propaguje. Toto lze vidět například na propagaci \href{https://www.youtube.com/watch?v=TanGK9o_d24}{Avengers: Endgame}.

\section{Představení titulu Counter-Strike: Global Offensive}
\aclu{CS:GO} jak ho známe dnes, má bohatou a dlouho historii. Ne vždy se to ovšem jmenovalo stejně. Úplně první iterace hry se jmenovala čistě Counter-Strike a byl to pouze mód\footnote{upravení či rozšíření hry} do
hry Half-Life. Half-Life byl vyvinutí společností Valve, tehdy primárně společností vyvíjející hry. Mód byl vytvořen studenty vysoké školy, panem \textit{Minh Le} a \textit{Jess Cliffe}. Toto rozšíření začali programovat
v roce 1999. Jelikož mód byl neoficiálním rozšířením, Valve se o něj moc nezajímalo. Až po pěti betaverzích hry Counter-Strike si společnost Valve všimla rozšíření, její komunity, ale především jejich autorů. Minh a Jess
se v roce 2000 stali oficiálními zaměstnanci Valve, prodali \uv{duševní vlastnictví} módu Valve. Autoři, nově jako zaměstnanci Valve, roku 2000 vydávají první oficiální verzi hry Counter-Strike. I přes toto \uv{oficiální}
datum vydání je většina komunity přesvědčena, že výročí má \ac{CS:GO} v den svého úplně první vydání, a to 18. června 1999.

Hra je z žánru \aclu{FPS} a hraje se hlavně online proti skutečným hráčům. Counter-Strike se v herní komunitě rychle rozrostl jeho unikátností. Hra se dá velmi dobře popsat pořekadlem
\textit{\uv{Lehké hrát, těžké vypilovat} (ang. Easy to play, hard to master.)}. Hra má mechaniky\footnote{herní prvky či unikátní vlastnosti}, které jsou lehké na pochopení, ale velmi těžké na vypilování k dokonalosti.
Spolu s touto vlastností je hra vlastně velmi jednoduchá a hráč hraje buď za policisty, nebo za teroristy. Hráči tak mohli, a stále můžou, hru velmi lehce a rychle začít hrát - tento formát se totiž za posledních 20 let nezměnil. 

Hra tedy rostla zejména díky své komunitě. Hráči hru různě upravovali, přidávali další módy, typy her, zbraně, mapy či audiovizuální obsah. Tento trend se přenášel přes mnoho různých verzí hry. První velký \uv{průlom} udělala
verze 1.6, tedy Counter-Strike 1.6. Ta kvetla jak esportem, tak komunitním obsahem. Jen v České a Slovenské republice bylo několik herních serverů, na kterých se mohlo sejít sta tisíce hráčů. Např. na česko-slovenském 
herním portálu \textit{kotelna} hrálo celkem přes 1,5 milionu unikátních hráčů \cite{csko2021}. Hra byla populární nejen mezi \uv{casual} hráči, ale i profesionály.

Counter-Strike 1.6 je pionýrem esportu pro \ac{FPS} žánr. Za podpory Valve se hráli první major\footnote{turnaj pořádaný přímo Valve, který má největší prestiž} turnaje, kde hráči mohli ukázat svůj um za tehdy relativně velkou
sumu peněz. Hra se časem vyvíjela, hráči nalézali nové strategie či triky a Valve vydalo novou verzi - Counter-Strike: Source. Tato nová verze získala nepříliš pozitivní ohlas, jelikož velmi rozdělila herní komunitu. Představila 
nové mechaniky, staré mechaniky změnila a hráčům, zejména v esportu, se nechtělo učit něco úplně nového. Valve se rozhodlo sjednotit herní komunitu, a proto vydalo novou verzi hry - \aclu{CS:GO}

\ac{CS:GO} se snažilo sjednotit oba tábory - Counter-Strike 1.6 a Counter-Strike: Source. Hra vyšla 21. srpna 2012 a z počátku nebyla tolik úspěšná, ale díky přidání různých skinů\cite{Valve2013} na zbraně hra přilákala
úplně nové publikum. Díky novému a velkému publiku se začali hrát menší esportové turnaje právě ve hře \ac{CS:GO}, ke kterým se později přidali i profesionále z předchozích dvou verzí. Díky tomuto organickému růstu má
Counter-Strike velmi silnou komunitu, která se o hru i nadále stará. I přes netradiční interakci mezi Valve a herní komunitou hra stále roste. \ac{CS:GO} se díky své dlouhé historii, bohaté komunitě a různým možnostem,
jak hru hrát, dostala na špičku esportu. I přes několik titulů, které se s hrou snaží soutěžit, je hra stále největším a nejsledovanějším esport titulem v rámci \ac{FPS} žánru\cite{Henningson2020}.

\section{Propojení práce a titulu Counter-Strike: Global Offensive}
\subsection{Vysvětlení hry}
Jak již bylo zmíněno, \ac{CS:GO} hraje pět hráčů proti pěti \textit{(dále jen 5v5)}. Hra se většinou hraje online, avšak velké esportové turnaje se hrají offline, tedy v nějaké např. aréně. Hra má v základu 30 kol a po 
prvních patnácti se mění strany. Jedna strana jsou policisté \textit{(Counter-Terrorists či CT)}, kteří mají za úkol chránit \uv{bomboviště} - část mapy, která má vybouchnout. Naopak cíl Teroristů \textit{(T)} je právě
bombu položit a \uv{bomboviště} nechat vybouchnout. Vyhrává team, který první dosáhne 16 kol. Pokud ovšem po prvních 30 kolech je stav nerozhodný, tedy 15:15, hraje se prodloužení. Tento formát není standardizovaný pro
všechny turnaje, proto se budu zaměřovat čistě na turnaje, které pořádá Valve \textit{(již zmíněné a nejvíc prestižní Majory)}. Zde se hraje prodloužení ve formát Bo6, tedy kdo první získá 4 body, vyhraje zápas. Takto 
může jít zápas teoreticky do nekonečna. Nejdelší semi-profesionální zápas, který se ovšem neodehrál na Majoru, se stal mezi týmem exceL a XENEX\cite{hltv.org2015}. Zápas pokračoval do úctyhodných 88 kol.

V každém kole má tým určitý počet peněz. Každá hráč začíná polovinu \textit{(ted v první a šestnácté kolo)} s \$800. Finance každého hráče pak záleží na mnoha faktorech - kolik vyhrál jeho team kol v řadě, kolik nakoupil
zbraní, kolik zabil nepřítelů, kolik peněz dostane hráč za zabití či jak kolo skončí. V profesionálním teamu je velmi obtížné pracovat s financemi, jelikož hráči musí být s financemi na jedné stránce. V tuto chvíli přichází
na řadu tzn. In-Game Leader \textit{(velitel teamu)}. Tuto roli má většinou jeden hráč v každém teamu. Je to ta nejdůležitější role ze všech - má na starosti finance, rozhoduje kdy se koupí a kdy půjde tzn. eco 
\textit{(hráči nekoupí nic, aby ušetřili peníze)}, jaké se budou hrát mapy či jaká se půjde v daném kole strategie.V dnešní době k tomu In-Game Leader má i pomocníka - trenéra. Ten hru nehraje, ale pozoruje hráče a dává jim
různé typy a triky. Role trenéra není nijak silně definovaná a každý esportový team má trošku jiného trenéra. V jednom případě může být trenér čistě jako podpora - pomáhá hráčům když se nedaří a řeší interní problémy. V jiném
teamu může ovšem mít velký zásah do hry, pomáhat In-Game Leaderovi se strategiemi, obelstění soupeře či sledováním předchozích zápasů pro kontinuální zlepšování teamu.
Další role v teamu jsou například Entry Fragger \textit{(má za úkol získat první zabití pro team)}, support \textit{(podporuje svůj team s pomocí různých granátů nebo se často pro svůj team obětuje)}, AWP hráč
\textit{(hráč je specifický tím, že hraje primárně s jednou zbraní - odstřelovací puškou AWP)} a Lurker \textit{(chodí po mapě sám a snaží se nepřítele odchytnout ze stran, které by nečekali)}

Zápasy se pak hrají ve formátech \uv{Best of}. \textit{Best of 3} například znamená, že se hrají tři mapy - kdo vyhraje dvě, vyhrál zápas. Turnaje se pak odehrávají v tradičních formátech, jako je pavouk.
Ten se charakterizuje tím, že vypadá jak pavučina, jde z leva a každý team může prohrát pouze jednou. Následně tu máme Upper/Lower bracket formát, který je v podstatě \uv{pavoučí formát}, akorát jsou zde dvě
sítě a každý team může prohrát maximálně jednou, jinak je vyřazen. Specifičtější formát pro \ac{CS:GO} je například swiss, který je složitější a mimo rozsah této práce.

Práce se zabývá právě predikcí daného zápasu. K tomuto lze použít např. hodnocení hráčů či umístění teamu na světovém žebříčku. Dále také záleží, na jaké mapě se zápas hraje. Když se budeme dívat na historické 
zápasy, musíme se dívat pouze na to, kdo zápas vyhrál - ne na rozdíl kol. Ten může být velmi zavádějící a zápas s výsledkem 16:7 mohl být více vyrovnaný než zápas s výsledkem 16:13. Toto jde vidět na zkušenostech teamu.
Zápas mezi velmi dobrými esportovými teamy může dopadnout 16:7, ale může být velmi těsný - oba dva týmy hráli dobře a o výsledku rozhodovali maličkosti. Pokud hrají dva méně zkušené týmy, zápas může skončit např. 16:13,
ale nemusí být vůbec blízko - jeden team může dělat zbytečné chyby \textit{(které by lepší team neudělal)}, které se normálně nedějí, což může vést k více těsnému výsledku i přes rozdílné hodnocení teamů.  
\chapter{Teoretická část}
V následující části jsou popsány jak teoretické metody pro vizualizaci dat, tak i tvar, forma a vyhodnocení logistického regresního modelu. 
Ke každé části, která se věnuje popisu dat pomocí nějakého grafu, je přidána praktická ukázka s popisem a praktickým vysvětlením.
vhodné.
Testovací citace: \cite{Hebak2015}, \cite{Kleinbaum2010}

\section{Vizualizace dat}
\subsection{Bodový graf}
{\color{red}
Bodový graf slouží pro zobrazení vztahu dvou numerických proměnných. Z pravidla se vysvětlovaná proměnná dává na osu Y,
zatímco proměnná vysvětlující se nachází na ose X. Vysvětlovaná (nezávislá) proměnná je ta proměnná, která má být určitým způsobem předvídaná.
Vysvětlující proměnná se naopak snaží vysvětlovanou proměnnou předpovědět či nějakým způsobem popsat. Propojením vysvětlované a vysvětlující proměnné
na bodovém grafu lze vidět např. sílu korelace nebo vztah mezi proměnnými (např. lineární, kvadratický, logaritmický). Graf \ref{fig:bodovy_graf_mtcars} zobrazuje
negativní korelaci mezi váhou vozidla a mílemi ujetými za galon.
}

\begin{figure}[H]
    \centering
    \includegraphics{../../analyza/plots/bodovy_graf_mtcars.png}
    \caption{Bodový graf váhy a míly za galon z data setu mtcars} 
    \label{fig:bodovy_graf_mtcars}
\end{figure}

{\color{red}
\subsection{Sloupcový graf}
Sloupcový graf slouží k porovnání kategoriální kvalitativní proměnné. Na jednu osu (z pravidla osu X) se položí možné kategorie. Na druhou osu
se pak položí sledovaná statistika. Sledovat můžeme např. počet výskytů, průměr, nebo relativní počet výskytu. Pokud je sledovaná proměnná ordinální,
je také možné odvodit vztah mezi kategoriemi. Příklad sloupcového grafu je zobrazen na obrázku \ref{fig:sloupcovy_graf_mtcars}, který
porovnává průměrnou hrubou koňskou sílu s počtem válců. Je na něm také vidět vztah, kdy s vyšším počtem válců stoupá průměrná koňská síla.

\begin{figure}[H]
    \centering
    \includegraphics{../../analyza/plots/sloupcovy_graf_mtcars.png}
    \caption{Sloupcový graf z data setu mtcars} 
    \label{fig:sloupcovy_graf_mtcars}
\end{figure}

\subsection{Histogram}

Histogram je speciální typ sloupcového grafu. Hlavní rozdíl je v tom, že popisuje bodové rozdělení spojité proměnné a mezi sloupci není žádná mezera.
}
Pro histogram je třeba data sloučit do skupin \textit{(bins)} o určité šířce. Správný výběr počtu skupin je kritický, jelikož může velmi
silně ovlivnit interpretaci dat. Pokud se vybere moc malý počet skupin, data se seskupí a může se ztratit důležitý vztah. Pokud se ovšem
vybere moc velký počet skupin, v datech bude obtížné najít nějaký obecný vztah či trend.
Tento efekt je znázorněn na obrázku \ref{fig:histogram_porovnani_mtcars}.

\begin{figure}[H]
    \centering
    \includegraphics{../../analyza/plots/histogram_porovnani_mtcars.png}
    \caption{Porovnání histogramů s různým počtem skupin} 
    \label{fig:histogram_porovnani_mtcars}
\end{figure}

Pro vhodný počet skupin existuje mnoho způsobů. Nejznámější je takzvané Sturgesovo pravidlo, které se spočítá následujícím vztahem:

\begin{equation}
    \label{eq:sturgesovo_pravidlo}
    k \text{ } \dot{\mathbf{=}} \text{ } 1 + 3,3 * log_{10}(n)
\end{equation}

kde $k$ je výsledný zaokrouhlený počet skupin 
{\color{red}
nahoru
}
a $n$ je počet pozorování. Druhý parametr, který je pro tvorbu histogramu potřeba, je šířka skupiny.
Ta by měla být ideálně stejná pro všechny skupiny. Pokud tomu tak není, histogram může být zavádějící a čtenář mu nemusí plně rozumět.
Pro vypočtení počtu skupin má šířka skupiny následující tvar:

\begin{equation}
    \label{eq:sirka_histogramu}
    w = \frac{max(x) - min(x)}{k}
\end{equation}

{\color{red}
kde $x$ je zobrazovaná proměnná, $k$ je počet skupin a $w$ je výsledná šířka intervalu. Pokud na stejný dataset, jako na obrázku 
\ref{fig:histogram_porovnani_mtcars}, použije Sturgesovo pravidlo \ref{eq:sturgesovo_pravidlo} a výpočet šířky \ref{eq:sirka_histogramu},
obrázek vypadá následovně:
}

\begin{figure}[H]
    \centering
    \includegraphics{../../analyza/plots/histogram_mtcars_sturges.png}
    \caption{Histogram s počtem skupin dle Sturgesova pravidla} 
    \label{fig:histogram_mtcars_sturges}
\end{figure}

Histogram lze samozřejmě rozepsat. Lze vytvořit tzn. tabulku četností ,
která mimo jiné obsahuje spodní hranici intervalu, horní hranici intervalu a počet pozorování. Důležité je,
aby přechody mezi intervaly byli jasné.

\begin{table}[H]
    \centering
    \begin{tabular}[t]{c|c|c}
        \hline
        Spodní hranice intervalu & Horní hranice intervalu & Počet pozorování\\
        \hline
        9.791667 & 13.70833 & 3\\
        \hline
        13.708334 & 17.62500 & 9\\
        \hline
        17.625001 & 21.54167 & 11\\
        \hline
        21.541668 & 25.45833 & 3\\
        \hline
        25.458334 & 29.37500 & 2\\
        \hline
        29.375001 & 33.29167 & 3\\
        \hline
        33.291668 & 37.20833 & 1\\
        \hline
    \end{tabular}
    \caption{\label{tab:tabulka_cetnosti_sturges}Tabulka četnostní}
\end{table}

\subsection{Boxplot}
\subsubsection{Five-number summary}
Five-number summary je číselná tabulka, která pomocí pěti různých čísel shrnuje seřazenou číselnou řadu. Základní statistický nástroj pro
vytvoření takové tabulky jsou kvantily. Hodnota $P$-tého percentilu označuje číslo, které rozděluje seřazenou číselnou řadu na dva intervaly. 
První interval obsahuje $P*100\%$ číselné řady a druhý analogicky $(1-P)*100\%$. Různé hodnoty percentilů mohou mít specifičtější pojmenování a značí se $Q_P$.
Percentil $P = 0.5$ se označuje jako medián a rozděluje seřazenou číselnou řadu na polovinu. Percentily, kde $P = 0.25$ nebo $P = 0.75$, se označují
jako kvartily a značí se $Q_{1}$ a $Q_{3}$. Oba tyto typy kvartilů jsou použité při tvorbě Five-number summary tabulky. Jako příklad je
uvedena tabulka \ref{tab:five-number_summary}

\begin{table}[H]

    \centering
    \begin{tabular}[t]{c|c|c|c|c}
        \hline
        $Q_{0} (Q_0)$ & $Q_{0.25} (Q_1) $ & $Q_{0.50}$ & $Q_{0.75} (Q_3)$ & $Q_{1.00}$\\
        \hline
        1.513 & 2.58125 & 3.325 & 3.61 & 5.424\\
        \hline
    \end{tabular}
    \caption{\label{tab:five-number_summary}Five-number summary tabulka hmotnosti vozidla (lb/1000)}
\end{table}

$Q_{0}$ a $Q_{1.00}$ označují minimum a maximum číselné řady. Kvartily $Q_{1}$ a $Q_{3}$ jsou čísla, která rozděluji časovou řadu na na čtvrtiny. V prvním
případě, tedy $Q_1 = Q_{0.25}$, je 25\% čísel menší než 1.513 a 75\% dat větší. Pro kvantil $Q_3 = Q_{0.75}$ je 75\% čísel menších než 3.61 a 25\% větších. $Q_{0.50}$ označuje medián.

\subsubsection{Boxplot}
Boxplot je grafické zobrazení a rozšíření Five-number summary tabulky. Krom grafické zobrazení 
{\color{red}
pěti kvantilů  ukazuje odlehlé a extrémní hodnoty.
}
V boxplotu se také nachází obdélník, který ukazuje mezikvartilové rozsah (IQR), tedy prostředních 50 \% dat. V obdélníku se také nachází černá čára, která značí medián.
Z prostředního obdélníku vedou oběma směry čáry, které značí teoretické minimum a maximum
{\color{red}
(což mohou a nemusí být hodnoty $Q_0$ a $Q_1$).
}
Obě dve hranice se počítají následovně

\begin{align}
    min &= Q_1 - 1.5 * IQR \\
    max &= Q_3 + 1.5 * IQR
\end{align}

{\color{red}
Hodnoty, které spadají do intervalu $\langle Q_1 - 1.5IQR; Q_1 - 3IQR\rangle$ a $\langle Q_3 + 1.5IQR, Q_3 + 3IQR \rangle$ se nazývají jako odlehlé.
Hodnoty které leží mimo tento vztah, tedy hodnoty menší než $Q_1 - 3IQR$ nebo větší než $Q_3 + 3IQR$ se nazývají jako hodnoty extrémní a
v boxplotu jsou z pravidla vyznačeny nějakým speciálním znakem, např. kolečkem.
Díky grafickému zobrazení lze lehce porovnávat rozdělení jedné vysvětlované proměnné tříděné přes několik kategorií.
}

\begin{figure}[H]
    \centering
    \includegraphics{../../analyza/plots/boxplot_mtcars.png}
    \caption{Boxplot váhy auta pro různé počty válců} 
    \label{fig:boxplot_mtcars}
\end{figure}

{\color{red}
Černé tečky v obrázku \ref{fig:boxplot_mtcars} v kategorii osmi válců značí odlehlé hodnoty, t. j. hodnoty
v intervalu $\langle Q_3 + 1.5IQR, Q_3 + 3IQR \rangle$.
}

\newpage
\section{Logistická regrese}
Logistická regrese je způsob, jak popsat vztah mezi jedním či několika prediktory a jednou binární predikovanou 
proměnnou. K tomu slouží spojovací funkce, která transformuje lineární kombinaci prediktorů na index $z$. V případě
logistické regrese se tato funkce nazývá logistická a je definovaná jako

\begin{equation}
    \label{eq:logisticka_funkce}
    f(z) = \frac{1}{1 + e^{-z}}
\end{equation}

Obor hodnot funkce je interval $\langle 0, 1 \rangle$. Proměnná $z$ je lineární kombinace prediktorů  $X_1, X_2, ..., X_k$,
jejich koeficienty $\beta_1, \beta_2, ..., \beta_k$ a parametru $\alpha$.

\begin{equation}
    \label{eq:linearni_kombinace_z}
    z = \alpha + \sum_{i=1}^k \beta_i X_i
\end{equation}

Mějme tedy binární predikovanou (vysvětlovanou) proměnnou $Y$, u které hodnota $1$ značí výskyt jevu. Pravděpodobnost, že jev nastane
vzhledem k definovaným prediktorům lze zapsat jako

\begin{equation}
    \label{eq:pravdepodobnost_y}
    P(Y = 1 \mid X_1, X_2, ..., X_k) = \frac{1}{1 + e^{-\alpha + \sum_{i=1}^k \beta_i X_i}}
\end{equation}

kde $\alpha$ a $\beta_i$ jsou parametry odhadnuté z datového souboru. 

\subsection{Interpretace parametrů}
Parametry $\alpha$ a $\beta_i$ značí logaritmus šance. $\alpha$ je  logaritmus šance, kde všechny prediktory
jsou teoreticky rovné $0$. Parametr $\beta_i$ značí logaritmus šance pro prediktor $X_1$ při změně o jednotku. 
Pokud je prediktor $Y$ skutečně binární, tedy značí $0$ pro nepřítomnost jevu a $1$ pro přítomnost jevu,
lze šance vypočítat jako

\begin{equation}
\text{šance } = e^{\beta_i}    
\end{equation}

Šance je podíl dvou pravděpodobností. Pokud bychom měli šanci jevu A oproti jevu B $2 : 1$, značí to, že výskyt jevu A je dvakrát tak pravděpodobný
jako výskyt jevu B a jev A se vyskytuje ve $2 \over 3$ případů. Šance $e^{\beta_i}$ tedy značí vztah mezi prediktorem $X_i$ a predikovanou proměnnou $Y$. Pokud je
šance kladná, značí to, že s vyšší hodnotou prediktoru $X_i$ se zvyšuje šance že $P(Y = 1)$. Pokud je naopak nižší, pravděpodobnost se zmenšuje. Pokud je potřeba
interpretovat pravděpodobnost jako šanci, použije se logitová funkce

\begin{equation}
    \label{eq:logitova_funkce}
    \text{šance jevu A} = \frac{p}{1 - p}
\end{equation}

kde $p$ je pravděpodobnost výskytu jevu A.

\subsection{Maximální pravděpodobnost}
Parametry logistického modelu v rovnici \ref{eq:pravdepodobnost_y} jsou pouze teoretické a je třeba je určitým způsobem odhadnout. Již vypočtené odhady
se proto neznačí pouze $\beta$, ale $\hat{\beta}$. Pro odhad parametrů se při logistické regresi používá metoda zvaná Největší pravděpodobnost. Pro výpočet
největší pravděpodobnosti se počítá pravděpodobnostní funkce $L(\theta)$ kde $\theta$ jsou parametry logistického modelu $\alpha, \beta_1, ..., \beta_k$.
Pro logistickou regresi má pravděpodobnostní funkce tvar

\begin{equation}
    L(\theta) = \Pi_{l = 1}^{m_1} P(X_i) \Pi_{l = m_1 + 1}^{n} 1 - P(X_i) 
\end{equation}

kde $n$ je počet pozorování a $m_1$ je počet příznivých ($Y = 1$) jevů. Funkce předpokládá, že datový soubor je seřazen tak, že prvních $m_1$ výskytů
jsou jevy příznivé. $P(X_i)$ poté značí logistickou funkci \ref{eq:logisticka_funkce}. Pro vypočtení optimálního parametru $\beta_i$ je nutné vypočítat
maximum funkce $L(\theta)$ vzhledem k parametru $\beta_i$. Parametr $\beta_i$ lze tedy získat derivací funkce $L(\theta)$ vzhledem k parametru $\beta_i$

\begin{equation}
    \frac{\partial L(\theta)}{\partial \beta_i} = 0
\end{equation}

\subsection{Vyhodnocení modelu}
... Dopsat

\chapter{Praktická část}
{\color{red}
Cílem praktické části je prozkoumat dva typy modelů. První typ modelu bude predikovat výhru zápasu pro jednotlivé hráče přes kategorie map. Hlavním cílem 
modelu bude identifikovat významné charakteristiky hráčů na vybraných mapách a následně rozdíly interpretovat. V druhém modelu bude zkoumaná výhra týmu v zápase.
V tomto typu modelu se jako prediktory použijí agregované charakteristiky hráčů jednotlivých týmu. Pro agregaci bude použit buď aritmetický nebo geometrický průměr. 

V této části se dále nachází popis dat a transformace dat. Nejprve jsou představené datové soubory,
se kterými se pracuje. Následně jsou použité grafy, které jsou představené v sekci \ref{sec:vizualizace_dat}. Pomocí vizualizace dat lze představit proměnné, které
do logistického modelu budou vstupovat. Také lze díky grafům získat povědomí, jak datový soubor vypadá a jaké mají proměnné rozdělení.
Poté jsou vytvořené logistické regresní modely, jejich výstup je interpretován a různé modely jsou mezi sebou porovnány.
}

{\color{red}
\section{Cíle analýzy}
Cílem analýzy je vytvořit logistické modely pro předpověď výhry jak pro individuální hráče, tak pro tým. Jelikož se charakteristiky hráčů i zápasů mohou měnit dle mapy,
na které se zápas odehrává, modely pro individuální hráče jsou tříděné přes kategorie map. Modely pro předpověď výhry týmu používají interakci mezi mapou a začínající
stranou.

Pro hráče budou sestaveny logistické modely z celého datového souboru přes kategorie map. Modely budou následně porovnány pomocí matice záměn
a bude porovnána významnost jejich parametrů. Modely budou porovnané na vybraných dvou mapách, a to Mirage a Vertigo. Mirage je klasický mapa, která je
ve hře od jeho vydání. Vertigo je naopak nejnovější přírůstek do profesionální scény a hráči, v době extrakce datového souboru, mapu ještě plně strategicky 
neznali. Modely budou porovnánu podle přesnosti predikce a podle různé významnosti prediktorů.

Modely pro týmy budou sestavené prve z celého datového souboru a následně pro dva vybrané referenční týmy. První tým, pro který se vytvoří model, bude tým Astralis.
Ten je dlouhodobě považován za jeden z nejlepších týmu na světě. Druhým týmem bude německý celek Sprout. Tým Sprout se řadí v době
extrakce dat k profesionálnímu týmu s průměrným třicátým místem na světovém žebříčku. Modely jsou vytvořené pouze pro dva týmy z toho důvodu, že je celkový
 počet týmů velmi vysoký a bylo by komplikované 
porovnávat všechny týmy naráz. Z toho důvodu se budou hledat významné rozdíly mezi dvěma referenčními modely a jedním celkovým. Mezi modely bude porovnaná jak přesnost
predikce, tak významnost parametrů. Jelikož je původní spojený datový soubor na úrovni hráčů, charakteristiky hráčů se musí agregovat na úroveň týmů a zápasů. Pro tuto
agregaci bude použit aritmetický nebo geometrický průměr.
}

\section{Datové soubory}
Dataset\footnote{https://www.kaggle.com/datasets/mateusdmachado/csgo-professional-matches} obsahuje čtyři 
{\color{red}
datové soubory,
}
které popisují zápasy ve hře
\ac{CSGO}. K potřebám této bakalářské práce budou použity pouze soubory \textit{players.csv} a \textit{results.csv}. Datové soubory jsou následně spojeny do jedné tabulky,
která obsahuje charakteristiky všech hráčů v právě jednom zápase, potřebné informace o zápase a výsledek (zda hráč zápas vyhrál či nikoliv). Data jsou následně dle potřeby
sjednocena pro týmy v zápase. Charakteristiky se agregují pomocí aritmetického nebo geometrického průměru. Zbylé dva soubory obsahují
informace, které jsou již z probíhajících zápasů a z volby map. Tyto informace pro predikci výhry ještě před začátkem zápasu nelze využít. Žádný z těchto
zbylých dvou souborů (\textit{picks.csv}, \textit{economy.csv}) proto v bakalářské práci není použit.

\subsection{soubor players.csv}
{\color{red}
Datový soubor
}
\textit{players.csv} obsahuje 
{\color{red}
charakteristiky
}
jednotlivých hráčů v daném zápase. Původní datový soubor obsahuje 101 proměnných a 383 317 pozorování.
V původním datovém souboru se jeden řádek (pozorování) rovná charakteristikám jednoho hráče za celý zápas
{\color{red}
, který se může odehrávat až na třech mapách.
}
Pro potřeby bakalářské práce je tak nutné získat charakteristiky hráčů
za jednotlivé mapy. Proto je původní datový soubor transformován do podoby, kde se jedno pozorování rovná charakteristikám
právě jednoho hráče na právě jedné mapě, a to bez ohledu na to, kolik map se v daném zápase hrálo. Jinak řečeno, transformovaný datový soubor nebere v potaz, zda
se daná mapa hrála jako první, druhá, či třetí.
Transformovaný datový soubor má 10 proměnných a 634 650 pozorování. Příklad jednotlivých pozorování v transformovaném
datovém souboru je v přiložené tabulce \ref{tab:players_csv_transformovano}.

\newpage
Transformovaný 
{\color{red}
datový soubor
}
má 10 proměnných. Interpretace je následující:
\begin{itemize}
    \item \textbf{match\_id} --- identifikátor zápasu
    \item \textbf{player\_id} --- identifikátor hráče
    \item \textbf{team} --- jméno týmu
    \item \textbf{map} --- název hrané mapy
    \item \textbf{kills} --- počet zabití hráče v zápase na dané mapě
    \item \textbf{assists} --- počet asistencí hráče v zápase na dané mapě
    \item \textbf{deaths} --- počet smrtí hráče v zápase na dané mapě
    \item \textbf{hs} --- procento zabití, které lze označit jako headshot\footnote{hráč zabil nepřítele střelou do hlavy}
    \item \textbf{fkdiff} --- rozdíl, kolikrát hráč zabil jako první nepřítele versus kolikrát byl zabit jako první
    \item \textbf{rating} --- shrnutí více charakteristik za jeden zápas do jednoho ukazatele výkonu\footnote{\url{https://www.hltv.org/news/20695/introducing-rating-20}}
\end{itemize}

\subsection{soubor results.csv}
Druhý datový soubor, který je pro analýzu použit, obsahuje výsledky daných zápasů. Soubor se původně skládá z 45 772 záznamů a 19 proměnných. Datový soubor 
\textit{results.csv} obsahuje na rozdíl od datového souboru \textit{players.csv} jedno chybné pozorování. Dle něho hrál tým sám proti sobě, což nedává smysl.
Jelikož je zápas na webovém portálu zadán správně, nejspíše se jedná o neznámou chybu, která nastala při exportu dat z webového portálu.

Po transformacích vznikne tabulka o 7 proměnných a 91 502 záznamech. Každé pozorování identifikuje výsledek jednoho týmu v jednom zápase
na jedné mapě. Příklad je zobrazen v přiložené tabulce \ref{tab:results_csv_transformovano}. Jednotlivé proměnné lze interpretovat následovně:
\begin{itemize}
    \item \textbf{date} --- datum, kdy se hrál zápas
    \item \textbf{match\_id} --- identifikátor zápasu
    \item \textbf{team} --- jméno týmu
    \item \textbf{map} --- název hrané mapy
    \item \textbf{map\_winner} --- binární značení, zda tým vyhrál (1) či prohrál (0)
    \item \textbf{starting\_ct} --- binární značení, zda tým začal zápas na straně Counter-Teroristů (1) či Teroristů (0)
    \item \textbf{team\_rank} --- rank týmu v okamžik, kdy se zápas hrál\footnote{\url{https://www.hltv.org/news/16061/introducing-csgo-team-ranking}}
\end{itemize}

\subsection{Omezení datového souboru}
{\color{red}
Všechny datové soubory obsahují pozorování o zápasech a charakteristikách hráčů od konce roku 2015 do začátku roku 2020. Jelikož pro finální modely je nutné datové soubory
\textit{players.csv} a \textit{results.csv} sjednotit, může se stát, že se vytvoří zápas bez hráčů. Může také nastat situace, kdy zápas nebude mít přiřazených právě 
deset různých hráčů. Tým může mít méně než 5 hráčů z toho důvodu, že je např.
amatérsky\footnote{neprofesionální, tím pádem nemusí mít všichni hráči na webovém portálu založená profil}. Více hráčů může hrát za tým v případě, že tým použil náhradníka.
Zároveň je možné, že kvůli historickým změnám ve hře a na webovém portálu nebude možné získat všechny potřebné charakteristiky hráčů. V případě všech zmíněných chyb jsou
chybné záznamy odstraněny.
}

\section{Průzkumová analýza dat}
Průzkumová analýza vizualizuje prediktory, hledá různé vztahy a rozdělení proměnných. Díky průzkumu lze určit, které proměnné není vhodné použít pro tvorbu
logistického regresního modelu, např. kvůli problému multikolinearity.

\newpage
\subsection{Korelační matice}
Pro logistickou regresi je důležité, aby prediktory nebyly lineárně závislé. Přehled korelací mezi kvantitativními prediktory lze zjistit z korelační matice.

\begin{figure}[H]
    \centering
    \includegraphics{../obrazky/prediktory_corr_matice.png}
    \caption{Korelační matice} 
    \label{fig:korelacni_matice}
\end{figure}

Z korelační matice \ref{fig:korelacni_matice} lze vyčíst, 
{\color{red}
že korelace mezi rankem týmu a charakteristikami hráčů se blíží nule. Z toho plyne, že neexistuje lineární závislost mezi výkonem hráče a 
rankem týmu.
}
Zároveň je vidět silná korelace mezi prediktorem \textit{rating} a prediktory \textit{fkdiff}, \textit{deaths} a \textit{kills}.
{\color{red}
Jelikož by díky vysoké korelaci prediktorů vznikl problém multikolinearity, prediktor \textit{rating} ve finálních modelech není použit.
}

\newpage
\subsection{Histogramy kvantitativních prediktorů}
Histogramy kvantitativních prediktorů umožní zobrazit jejich rozdělení.

\begin{figure}[H]
    \centering
    \includegraphics{../obrazky/histogram_prediktoru.png}
    \caption{Histogram prediktorů} 
    \label{fig:histogram_prediktoru}
\end{figure}

{\color{red}
Histogram prediktorů z obrázku \ref{fig:histogram_prediktoru} ukazuje, že přediktory \textit{rating}, \textit{hs}, \textit{kills} a \textit{deaths} mají normální rozdělení
a v prediktorech se nenachází mnoho extrémních hodnot.
}
Prediktor \textit{fkdiff} má bimodální rozdělení. Prediktor \textit{assists} je 
{\color{red}
zešikmení
}
doprava, což značí velké množství odlehlých či extrémních hodnot. 
{\color{red}
Pro logistickou regresi není předpoklad normálního rozdělení prediktorů Analýza proto slouží k získání povědomí o tom, 
jakých hodnot každý prediktor nabývá a jaké je jejich rozdělení.
}

\newpage
{\color{red}
\subsection{Sloupcový graf výher přes počáteční stranu}
Každá mapa funguje a vypadá jinak, což ovlivňuje mimo jiné i možné strategie. Proto jiné mapy mohou více vyhovovat jiným týmům. Toto způsobuje, že
tým může mít vyšší procento výhry na mapě, pokud začíná na straně Counter-Terroristů.

\begin{figure}[H]
    \centering
    \includegraphics{../obrazky/sloupce_podle_strany.png}
    \caption{Procento vyhraných zápasů na dané mapě za stranu Counter-Terroristů} 
    \label{fig:sloupcovy_graf_strany}
\end{figure}

Z obrázku \ref{fig:sloupcovy_graf_strany} je patrné, že týmu Astralis nevyhovuje začínat mapu Cobblestone na straně Counter-Terroristů. Tým Sprout má největší procento
vyhraných zápasů pri počáteční straně jako Counter-Terroristé na mapě Vertigo. Nejmenší pak na mapě Mirage. Při pohledu na sloupcový graf pro všechny týmy lze vyčíst,
že celkově počáteční strana nemá vliv. Díky obrázku \ref{fig:sloupcovy_graf_strany} je vidět, že šanci na výhru ovlivňuje interakce mezi mapou a začínající stranou. 
}

\section{Predikce výhry hráče}
{\color{red}
Cílem modelu je predikovat výhru zápasu pro jednotlivé hráče a identifikovat významné prediktory na odlišných mapách.
}
Prediktory se týkají pouze výkonu jednotlivých hráčů, model tedy pro předpověď výhry hráče nepoužívá charakteristiky spoluhráčů. Pro porovnání jsou vybrané mapy Mirage a Vertigo.
Mapa Mirage je jednou z nejvíce tradičních map a mapa Vertigo je naopak nejnovější přídavek do hry. Díky rozdílným modelům bude možné zkoumat, na čem pravděpodobnost výhry na 
mapách záleží. Pro vytvoření logistického modelu je použito 80\% náhodně vybraných pozorování. Zbylých 20\% je použito pro
{\color{red}
ověření kvality modelu.
}

\subsection{Model pro mapu Mirage}

\input{kod/modely/player_model_Mirage.tex}

{\color{red}
Z tabulky \ref{tab:player_model_Mirage} je vidět, že pro model není významný prediktor \textit{hs}. Prediktor je proto odebrán a model je znovu natrénován na stejných datech.
}

\input{kod/modely/player_model_Mirage_opt.tex}

{\color{red}
Tabulka \ref{tab:player_model_Mirage_opt} představuje model se všemi významnými prediktory. Prediktory \textit{kills}, \textit{assists} a \textit{fkdiff} šanci na výhru
hráče zvyšují. S každým zabitím hráč zvyšuje šanci na výhru zhruba 1,2 krát. Naopak prediktory \textit{deaths} a \textit{starting\_ct} šanci snižují. S každou hráčovou smrtí
se šance na výhru snižuje zhruba 0,69 krát. Pokud hráči začnou mapu na straně Counter-Terroristů, jejich šance na výhru se sníží zhruba 0,82 krát. 
Model lze zapsat také pomocí logistické funkce.
}

\begin{align}
    \begin{split}
        &P(1 | X_{kills}, X_{assists}, X_{deaths}, X_{fkdiff}, X_{starting_{ct}}) = \frac{1}{1 + e^{-z}} \\
        &z = 2,399 + 0,180*X_{kills} + 0,305*X_{assists} - 0,371*X_{deaths} + \\
        &+ 0,017*X_{fkdiff} - 0,203*X_{starting_{ct}}
    \end{split}
    \label{eq:player_funkce_Mirage}
\end{align}

\subsubsection{Matice záměn pro mapu Mirage}
Predikce jsou provedené na validačním podmnožině a na optimalizovaném modelu z tabulky \ref{tab:player_model_Mirage_opt}.

\input{kod/matice/player_matice_Mirage_opt.tex}

Model predikoval správně 7 007 výher 
{\color{red}
($\sim 81,4\%$)
}
a 6 689 proher
{\color{red}
($\sim 77,2\%$). Celkově model určil správně 13 696 objektů ($\sim 79,3\%$).
}

\input{kod/matice_out/player_stats_Mirage_opt.tex}

{\color{red}
Díky vyšší specificitě modelu o zhruba $4\%$ je model vhodnější na predikci prohry hráče.
}

\subsection{Model pro mapu Vertigo}

\input{kod/modely/player_model_Vertigo.tex}

Pro hráče jsou na mapě Vertigo významné pouze prediktory \textit{kills}, \textit{assists} a \textit{deaths}. Ostatní prediktory \textit{hs}, \textit{fkdiff} a \textit{starting\_ct}
jsou pro model nevýznamné. Nevýznamnost lze interpretovat tak, že pro hráče není důležité, na jaké straně mapu začnou hrát (prediktor  \textit{starting\_ct}), jak přesně
střílí (prediktor \textit{hs})
{\color{red}
, ani jak týmově hraje na začátku kola (prediktor \textit{fkdiff}).
}
Po vyřazení nevýznamných prediktorů má model následující významné parametry:

\input{kod/modely/player_model_Vertigo_opt.tex}

{\color{red}
Jak lze očekávat, prediktory \textit{kills} a \textit{assists} šanci na výhru zvyšují, zatím co prediktor \textit{deaths} šanci na výhru sníží. S každým zabitím hráče
na mapě Vertigo se jeho šance na výhru zvýší zhruba 1,2 krát a s každou smrtí se jeho šance na výhru sníží zhruba 0,70 krát.
}
Přepis funkce je pak následující:

\begin{align}
    \begin{split}
        &P(1 | X_{kills}, X_{assists}, X_{deaths}) = \frac{1}{1 + e^{-z}} \\
        &z = 1,971 + 0,183*X_{kills} + 0,308*X_{assists} - 0,357*X_{deaths}.
    \end{split}
\end{align}

\subsubsection{Matice záměn pro mapu Vertigo}
Matice záměn pro 
{\color{red}
optimální
}
logistický model sestavený pro mapu Vertigo
{\color{red}
vypadá následovně:
}

\input{kod/matice/player_matice_Vertigo_opt.tex}

Model úspěšné predikoval 430 výher
{\color{red}
($\sim 76,6\%$)
}
a 441 proher 
{\color{red}
($\sim 77,8\%$). Celkově model úspěšně predikoval 851 objektů ($\sim 77,2\%$)
}

\input{kod/matice_out/player_stats_Vertigo_opt.tex}

Z tabulky statistik \ref{tab:player_stats_Vertigo_opt} je vidět nepatrně lepší výkon při predikci výher (statistika Senzitivita) o necelé jedno procento.
Model se proto hodí spíše na predikci výher.

\subsection{Interpretace výsledků}
Model pro mapu Mirage je spolehlivější pro předpověď prohry hráče. Model pro mapu Vertigo má menší přesnost o zhruba 
2 procentní body a hodí se spíše k predikci výher hráčů.
{\color{red}
Přehled výsledku přes ostatní mapy lze nalézt v příloze \ref{chap:ap02}
}

Pro logistický model mapy Mirage není významný prediktor \textit{hs}. Největší vliv na výhru hráče na mapě Mirage má počet jeho smrtí, kdy s každou další smrtí se šance
na výhru zmenšuje zhruba 0,69 krát. Pro mapu Vertigo jsou významné pouze prediktory \textit{kills}, \textit{assists} a \textit{deaths} a největší vliv na výhru hráče
{\color{red}
má také počet smrtí, kde s každou hráčovou smrtí šance klesne zhruba 0,7 krát.
}

Rozdíl
{\color{red}
ve významnosti prediktorů
}
by mohl být vysvětlen právě stářím jednotlivých map. Jelikož je mapa Mirage už tradiční mapou, mnoho hráčů na ní zná různé triky a strategie. To vede k tomu, že
každý malý detail hraje velikou roli a většina vybraných prediktorů je pro model významná. Naopak mapa Vertigo je relativně nová a pro hráče je nejdůležitější individuální
výkon. Jelikož hráči mapu tak dobře strategicky neznají, jsou významné pouze charakteristiky hráčů.

\newpage
\section{Predikce výhry týmu}
Cílem modelů je predikovat výhru na základě agregovaných charakteristik hráčů za tým na mapě. Charakteristiky jsou agregované buď pomocí aritmetického průměru,
nebo pomocí průměru geometrického. Ostatní charakteristiky zápasu jako počáteční strana či výherní tým nejsou nijak změněny. V modelu pro predikci týmu vystupuje
navíc prediktor \textit{team\_rank}, který ukazuje rank daného týmu na mapě. Dále je v modelu zakomponovaná interakce
mezi počáteční stranou (\textit{starting\_ct}) a hranou mapou (\textit{map}). Příklad tabulky s agregovanými charakteristikami je v příloze \ref{tab:data_agregovana}.
Pro trénink a validaci je rozdělený původní datový soubor v poměru 8:2.

\subsection{Celkový model}

\input{kod/modely/team_model_All.tex}

První vytvořený model je sestavený na celém trénovacím datovém souboru. Pro model jsou významné všechny prediktory bez interakce, jmenovitě \textit{mean\_kills},
\textit{mean\_assists}, \textit{mean\_deaths}, \textit{mean\_hs},\textit{mean\_fkdiff} a \textit{team\_rank}. Interakce mezi prediktory \textit{map} a
\textit{starting\_ct} není významná u map Vertigo, Cache a Nuke.

Agregované charakteristiky hráče \textit{mean\_kills} a \textit{mean\_assists} šanci na výhru týmu zvyšují. Pokud se průměr zabitých nepřátel za tým zvýší o jednotku,
šance na výhru týmu se zvýší zhruba 3,96 krát. Pokud se průměr smrtí hráčů za tým zvýší o jednotku, šance na výhru se sníží zhruba 0,25 krát. Všechny statisticky
významné interakce mezi prediktory \textit{map} a \textit{starting\_ct} naznačují, že je pro tým nevýhodné začínat na straně Counter-Terroristů. Jejich šance
na výhru se vždy sníží, a to nejvíce na mapě Overpass, kde se šance sníží zhruba 0,62 krát. Zajímavý je koeficient u prediktoru \textit{team\_rank}, který říká,
že s růstem ranku týmu se šance na výhru sníží zhruba 0,99 krát. To lze vysvětlit tím, že lepší týmy hrají proti lepším týmům a jejich 
šance na výhru je nižší. Model lze zapsat jako přepis funkce následovně:

\begin{align}
    \begin{split}
        P(1 | &X_{mean\_kills}, X_{mean\_assists}, X_{mean\_deaths}, X_{mean\_hs}, X_{mean\_fkdiff}, X_{team\_rank}, \\
              &X_{mapCache*starting\_ct}, X_{mapCobblestone*starting\_ct}, X_{mapDust2*starting\_ct}, X_{mapInferno*starting\_ct}, \\
              &X_{mapMirage*starting\_ct}, X_{mapNuke*starting\_ct}, X_{mapOverpass*starting\_ct}, X_{mapTrain*starting\_ct}, \\
              &X_{mapVertigo*starting\_ct}) = \frac{1}{1 + e^{-z}} \\
        z = &0,322 + 1,376*X_{mean\_kills} + 0,143*X_{mean\_assists} - 1,402*X_{mean\_deaths} - \\
            &- 0,615*X_{mean\_hs} - 0,064*X_{mean\_fkdiff} - 0,001*X_{team\_rank} - \\
            &- 0,040*X_{mapCache*starting\_ct} - 0,282*X_{mapCobblestone*starting\_ct} - \\
            &- 0,318*X_{mapDust2*starting\_ct} - 0,272*X_{mapInferno*starting\_ct} - \\
            &- 0,239*X_{mapMirage*starting\_ct} - 0,113*X_{mapNuke*starting\_ct} - \\
            &- 0,418*X_{mapOverpass*starting\_ct} - 0,177*X_{mapTrain*starting\_ct} + \\
            &+ 0,259*X_{mapVertigo*starting\_ct}
    \end{split}
\end{align}

\subsubsection{Matice záměn pro obecný model}
Model je vyhodnocen na validační podmnožině, která činní 20\% původního datového souboru.

\input{kod/matice/team_matice_All.tex}

Model úspěšně predikoval 8 274 výher ($\sim 94,1\%$) a 8 203 proher ($\sim 94,0\%$). Celkem model predikoval správně 16 477 objektů ($\sim 94,1\%$).

\input{kod/matice_out/team_stats_All.tex}

Všechny výkonnostní statistiky z matice záměn jsou identické v řádu setin procent. Z tabulky \ref{tab:team_stats_All} nelze jednoznačně určit, zda je tým
vhodnější na predikci výher či proher.

\subsection{Model pro tým Astralis}
Tým Astralis je v době extrakce dat jedním z nejlepších týmu na světě. Vyhrál několik prestižních majorů a i v roce 2022 se tým Astralis považuje za nejlepší tým
ve hře \ac{CSGO} všech dob.

\input{kod/modely/team_model_Astralis.tex}

Významné prediktory pro tým Astralis jsou pouze \textit{mean\_kills} a \textit{mean\_deaths}.  Pro tým není statistický významné, jaký je průměrný počet asistencí
(\textit{mean\_assists}), jaké je průměrné procento zabití do hlav (\textit{mean\_hs}), jaký je průměrný výkon hráčů na začátku mapy (\textit{mean\_fkdiff}) ani 
rank týmu v daném zápase (\textit{team\_rank}). Tým Astralis navíc neovlivňuje počáteční strana, mají tedy stejnou šanci na výhru bez ohledu na 
začínající stranu.

\input{kod/modely/team_model_Astralis_opt.tex}

Tabulka \ref{tab:team_model_Astralis_opt} popisuje již optimální parametry pro model týmu Astralis. S každým dalším průměrným zabitím se zvýší šance
týmu Astralis na výhru zhruba 3,85 krát. S každou další průměrnou smrtí se šance na výhru týmu sníží zhruba 0,23 krát. 
Před začátkem zápasu, tedy při nulových prediktorech \textit{mean\_kills} a \textit{mean\_deaths}, je šance na výhru týmu Astralis zhruba 9,75 větší, než jeho
prohra. Optimální model lze zapsat jako přepis rovnice.

\begin{align}
    \begin{split}
        P(1 | &X_{mean\_kills},  X_{mean\_deaths}) = \frac{1}{1 + e^{-z}} \\
        z = &2,277 + 1,349*X_{mean\_kills} - 1,457*X_{mean\_deaths}
    \end{split}
\end{align}

\subsubsection{Matice záměn pro tým Astralis}

\input{kod/matice/team_matice_Astralis_opt.tex}

Optimální model správně predikoval 54 výher ($\sim 93,1\%$) a 108 proher ($\sim 93,9\%$). Celkem optimální model predikoval správně 162 objektů ($\sim 93,6\%$).

\input{kod/matice_out/team_stats_Astralis_opt.tex}

\subsection{Model pro tým Sprout}
Tým Sprout byl v době extrakce dat čistě Německý tým a patřil k průměrným profesionálnímu týmům. Na žebříčků týmu se obvykle řadil kolem třicátého místa.

\input{kod/modely/team_model_Sprout.tex}

Z tabulky modelu \ref{tab:team_model_Sprout} je vidět, že významné prediktory pro tým jsou pouze \textit{mean\_kills} a \textit{mean\_deaths}. Pro tým není významné,
na jaké straně začíná mapu (interakce \textit{map*starting\_ct}), jaký je jeho rank (\textit{team\_rank}), jak přesně průměrné střílí hráči týmu (\textit{mean\_hs}),
jak jsou průměrné hráči dobří na začátku kola (\textit{mean\_fkdiff}) ani kolik mají průměrně asistencí (\textit{mean\_assists}).

\input{kod/modely/team_model_Sprout_opt.tex}

S každým dalším průměrným zabitím (\textit{mean\_kills}) se šance na výhru týmu zvýší zhruba 4,95 krát. S každou další průměrnou smrtí (\textit{mean\_deaths})
se šance na výhru sníží zhruba 0,23 krát. Ještě před začátkem zápasu je šance na prohru týmu zhruba 0.12 krát větší, než na jeho výhru.

\begin{align}
    \begin{split}
        P(1 | &X_{mean\_kills},  X_{mean\_deaths}) = \frac{1}{1 + e^{-z}} \\
        z = &-2,098 + 1,600*X_{mean\_kills} - 1,485*X_{mean\_deaths}
    \end{split}
\end{align}

\subsubsection{Matice záměn pro tým Sprout}

\input{kod/matice/team_matice_Sprout_opt.tex}

Optimální model korektně předpověděl 54 výher ($\sim 93,1\%$) a 62 proher ($\sim 96,9\%$). Model úspěšně předpověděl celkem 116 objektů ($\sim 95,1\%$)

\input{kod/matice_out/team_stats_Sprout_opt.tex}

Jelikož je senzitivita modelu o zhruba 4 procentní body větší, model se více hodí na předpověď výhry týmu Sprout.

\subsection{Interpretace výsledků}
Pro celkový model jsou kromě interakcí $mapCache:starting\_ct$, $mapNuke:starting\_ct$ a $mapVertigo:starting\_ct$ významné všechny prediktory. Největší vliv na výhru má
prediktor \textit{mean\_deaths}, která šanci snižuje zhruba 0,25 krát. U všech významných kombinací mezi prediktory \textit{map} a \textit{starting\_ct} je koeficient
záporný. To naznačuje, že tým má menší šanci na výhru, pokud mapu začne na straně Counter-Terroristů. Největší negativní vliv je u mapy Overpass, kde se šance
sníží až 0,65 krát.

Tým Astralis má před začátkem zápasu, tedy při nulových prediktorech, šanci na výhru zhruba 9,75 větší, než na prohru. Vzhledem k tomu, že tým byl dlouhodobě považován
za jeden z nejlepších týmu na světě a v roce 2022 je mnoha hráči považován za nejlepší tým všech dob, je výsledek očekávaný. Tým Sprout má před začátkem zápasu 
šanci na prohru zhruba 0,25 krát větší, než na výhru. Tým je průměrně umístěn na třicátém místě, díky čemuž hraje převážně proti nejlepším padesáti týmům na světě.

Oba dva referenční modely mají stejné významné prediktory, a to \textit{mean\_kills} a \textit{mean\_deaths}. Pro žádný tým není významné, na jaké mapě začíná a jeho šanci
na výhru to nijak neovlivňuje. Z modelů lze usoudit, že nejlepší tým na světě a průměrný profesionální tým má stejné významné prediktory.Rozdíl mezi referenčními modely a 
celkovým modelem by bylo možné vysvětlit např. vlivem neprofesionálních týmů. Porovnávání neprofesionálních či polo-profesionálních týmu je složitější, jelikož týmy 
hrají méně zápasů, a není zaručeno, že mají dostatek zápasu na každé mapě. Datový soubor by bylo nutné tím pádem velmi omezit.
\chapter{Závěr}
... Uzavření bakalářské práce

\section{Závěrečné vyhodnocení modelu}
... Výsledné vyhodnocení modelu pomocí všech statistik

\section{Interpretace modelu do reálného světa}
... Přenesení modelu do reálného světa

\section{Použití modelu v reálném světě}
... Použití modelu v reálném světě

\section{Místo pro budoucí vylepšení}
...


%%% Seznam použité literatury
%% Toto platí v případě použití samostatné bibliografické databáze
\printbibliography[title={Seznam použitého softwaru}, heading={bibintoc}, keyword={sw}]

\printbibliography[title={Seznam použité literatury}, heading={bibintoc}, notkeyword={ezdroj,sw}]

\printbibliography[title={Seznam elektronických zdrojů}, heading={bibintoc}, keyword={ezdroj}]


%%% Obrázky v bakalářské práci
\openright
\phantomsection
\addcontentsline{toc}{chapter}{\listfigurename}
\listoffigures

%%% Tabulky v bakalářské práci (opět nemusí být nutné uvádět)
\clearpage
\phantomsection
\addcontentsline{toc}{chapter}{\listtablename}
\listoftables

%%% Použité zkratky v bakalářské práci (opět nemusí být nutné uvádět)
\chapter*{Seznam použitých zkratek}
\addcontentsline{toc}{chapter}{Seznam použitých zkratek}

\begin{acronym}[CS:GO]
    %% esport tituly
    \acro{CS:GO}{Coutner-Strike: Global Offensive}
    %% esport žánry
    \acro{BR}{Battle Royale \textit{(hra o přežití)}}
    \acro{MOBA}{Multiplayer online battle arena}
    \acro{FPS}{First-person shooter \textit{(střílečka z pohledu první osoby)}}
    %% ostatní
    \acro{TGNS}{Twin Galaxies National Scoreboard}
\end{acronym}


%%% Přílohy k bakalářské práci
\part{Přílohy}
\appendix
\chapter{Datové soubory}
\section{Transformovaný datový soubor players.csv}
\begin{table}[H]

\caption{\label{tab:players_csv_transformovano}Záznam z transformovaného datového souboru players.csv}
\centering
\resizebox{\linewidth}{!}{
\begin{tabular}[t]{c|c|c|c|c|c|c|c|c|c}
\hline
match\_id & player\_id & team & map & kills & assists & deaths & hs & fkdiff & rating\\
\hline
2339385 & 8738 & Liquid & Overpass & 15 & 3 & 12 & 0.6 & 3 & 1.32\\
\hline
\end{tabular}}
\end{table}


\section{Transformovaný datový soubor results.csv}
\begin{table}[H]

\caption{\label{tab:results_csv_transformovano}Příklad záznamu z transformovaného datového souboru results.csv}
\centering
\resizebox{\linewidth}{!}{
\begin{tabular}[t]{c|c|c|c|c|c|c|c}
\hline
date & match\_id & team & map & map\_winner & starting\_ct & team\_rank & run\_mean\_3\_months\\
\hline
2019-11-07 & 2337454 & 100 Thieves & Nuke & 0 & 1 & 8 & 8\\
\hline
\end{tabular}}
\end{table}


\section{Datový soubor pro logistické modely}
\input{kod/data_transformovano.tex}
\chapter{Modely, matice záměn a statistiky pro individuální hráče} \label{apx:02}
\section{Mapa Cache}
\input{modely/Cache_model.tex}
\input{matice/Cache_matice.tex}
\input{statistiky/Cache_stats.tex}
\newpage

\section{Mapa Cobblestone}
\input{modely/Cobblestone_model.tex}
\input{matice/Cobblestone_matice.tex}
\input{statistiky/Cobblestone_stats.tex}
\newpage

\section{Mapa Dust2}
\input{modely/Dust2_model.tex}
\input{matice/Dust2_matice.tex}
\input{statistiky/Dust2_stats.tex}
\newpage

\section{Mapa Inferno}
\input{modely/Inferno_model.tex}
\input{matice/Inferno_matice.tex}
\input{statistiky/Inferno_stats.tex}
\newpage

\section{Mapa Nuke}
\input{modely/Nuke_model.tex}
\input{matice/Nuke_matice.tex}
\input{statistiky/Nuke_stats.tex}
\newpage

\section{Mapa Overpass}
\input{modely/Overpass_model.tex}
\input{matice/Overpass_matice.tex}
\input{statistiky/Overpass_stats.tex}
\newpage

\section{Mapa Train}
\input{modely/Train_model.tex}
\input{matice/Train_matice.tex}
\input{statistiky/Train_stats.tex}
\newpage


\end{document}
